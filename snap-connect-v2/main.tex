\documentclass[a4paper,11pt]{article}
\usepackage[utf8]{inputenc}
%

\usepackage[utf8]{inputenc}%Packages
\usepackage[T1]{fontenc}
\usepackage{fourier} 
\usepackage[english]{babel} 
\usepackage{amsmath,amsfonts,amsthm} 
\usepackage{lscape}
\usepackage{geometry}
\usepackage{amsmath}
\usepackage{algorithm}
\usepackage{algorithmic}
\usepackage{amssymb}
\usepackage{amsfonts}
\usepackage{times}
\usepackage{bm}
\usepackage{mathtools}
\usepackage{ stmaryrd }
\usepackage{ amssymb }
\usepackage{ textcomp }
\usepackage[normalem]{ulem}
% For derivation rules
\usepackage{mathpartir}
\usepackage{color}
\usepackage{a4wide}

\usepackage{stmaryrd}
\SetSymbolFont{stmry}{bold}{U}{stmry}{m}{n}

\newcommand{\distr}{\mathsf{Distr}}
\newcommand{\uniform}{\mathsf{unif}}
\newcommand{\pdf}{\mathsf{pdf}}
\newcommand{\snap}{\mathsf{Snap}}
\newcommand{\fsnap}{\mathsf{Snap}_{\mathbb{F}}}
\newcommand{\rsnap}{\mathsf{Snap}_{\mathbb{R}}}


\newcommand{\pr}[2]{\underset{#1}{\mathsf{Pr}}[#2]}
\newcommand{\projl}{\pi_1}
\newcommand{\projr}{\pi_2}
\newcommand{\supp}{\mathsf{supp}}
\newcommand{\clamp}{\mathsf{clamp}}
\newcommand{\real}{\mathbb{R}}
\newcommand{\samplel}{\xleftarrow{\$}}
\newcommand{\psup}{\mathsf{Sup}}
\newcommand{\sign}{\mathsf{sign}}

\newcommand{\lapmech}{\mathcal{L}}
\newcommand{\laplace}{\mathsf{laplce}}
\newcommand{\round}[1]{\lfloor #1 \rceil}


%for syntax:

%for programs:
\newcommand{\prog}{p}
\newcommand{\fprog}{p_{\mathbb{F}}}
\newcommand{\rprog}{p_{\mathbb{R}}}
\newcommand{\ret}{\mathsf{return}}



%expression
\newcommand{\expr}{e}
\newcommand{\fexpr}{\expr_{\mathbb{F}}}
\newcommand{\rexpr}{\expr_{\mathbb{R}}}

\newcommand{\elet}{\kw{let}}

\newcommand{\ein}{\kw{in}}

%for smaples:
\newcommand{\bernoulli}{\kw{bernoulli}}

%values
\newcommand{\fval}{c}
\newcommand{\rval}{r}
\newcommand{\valv}{v}
\newcommand{\data}{D}

%variables
\newcommand{\varx}{x}

\newcommand{\fvarx}{x}
\newcommand{\rvarx}{X}


\newcommand{\term}{t}
\newcommand{\etrue}{\kw{true}}
\newcommand{\efalse}{\kw{false}}
% \newcommand{\eflconst}{c}
% \newcommand{\erlconst}{r}
\newcommand{\precision}{\eta}
\newcommand{\floaten}{\kw{fl}}

\newcommand{\err}{err}
\newcommand{\condition}{\Phi}
\newcommand{\edistr}{\mu}

\newcommand{\fbigstep}{\Downarrow^{\mathbb{F}}}
\newcommand{\rbigstep}{\Downarrow^{\mathbb{R}}}

\newcommand{\bigstep}{\Downarrow}
\newcommand{\trsto}{\Rightarrow}


%for environments
\newcommand{\trsenv}{\Theta}

\newcommand{\evlenv}{\Gamma}

\newcommand{\fevlenv}{\Gamma^{\mathbb{F}}}

\newcommand{\revlenv}{\Gamma^{\mathbb{R}}}



\usepackage{stackengine} 

% For Operations
%binary operations
\newcommand{\bop}{*}
\newcommand{\obop}{\stackMath\mathbin{\stackinset{c}{0ex}{c}{0ex}{\text{\footnotesize{$\bop$}}}{\bigcirc}}}

\newcommand{\oexp}{\stackMath\mathbin{\stackinset{c}{0ex}{c}{0ex}{\text{\footnotesize{$\mathsf{e}$}}}{\bigcirc}}}

\newcommand{\oln}{\stackMath\mathbin{\stackinset{c}{0ex}{c}{0ex}{\text{\footnotesize{$\mathsf{ln}$}}}{\bigcirc}}}

\newcommand{\odiv}{\stackMath\mathbin{\stackinset{c}{0ex}{c}{0ex}{\text{\footnotesize{$\div$}}}{\bigcirc}}}
\newcommand{\ubar}[1]{\text{\b{$#1$}}}

%unary operations
\newcommand{\uop}{\circ}
\newcommand{\ouop}{\stackMath\mathbin{\stackinset{c}{0ex}{c}{0ex}{\text{\footnotesize{$\uop$}}}{\bigcirc}}}





\newcommand{\diam}{{\color{red}\diamond}}
\newcommand{\dagg}{{\color{blue}\dagger}}
\let\oldstar\star
\renewcommand{\star}{\oldstar}

\newcommand{\im}[1]{\ensuremath{#1}}

\newcommand{\kw}[1]{\im{\mathtt{#1}}}


\newcommand{\set}[1]{\im{\{{#1}\}}}

\newcommand{\mmax}{\ensuremath{\mathsf{max}}}

%%%%%%%%%%%%%%%%%%%%%%%%%%%%%%%%%%%%%%%%%%%%%%%%%%%%%%%%
% Comments
\newcommand{\omitthis}[1]{}

% Misc.
\newcommand{\etal}{\textit{et al.}}
\newcommand{\bump}{\hspace{3.5pt}}

% Text fonts
\newcommand{\tbf}[1]{\textbf{#1}}
%\newcommand{\trm}[1]{\textrm{#1}}

% Math fonts
\newcommand{\mbb}[1]{\mathbb{#1}}
\newcommand{\mbf}[1]{\mathbf{#1}}
\newcommand{\mrm}[1]{\mathrm{#1}}
\newcommand{\mtt}[1]{\mathtt{#1}}
\newcommand{\mcal}[1]{\mathcal{#1}}
\newcommand{\mfrak}[1]{\mathfrak{#1}}
\newcommand{\msf}[1]{\mathsf{#1}}
\newcommand{\mscr}[1]{\mathscr{#1}}

% Text mode
\newenvironment{nop}{}{}

% Math mode
\newenvironment{sdisplaymath}{
\begin{nop}\small\begin{displaymath}}{
\end{displaymath}\end{nop}\ignorespacesafterend}
\newenvironment{fdisplaymath}{
\begin{nop}\footnotesize\begin{displaymath}}{
\end{displaymath}\end{nop}\ignorespacesafterend}
\newenvironment{smathpar}{
\begin{nop}\small\begin{mathpar}}{
\end{mathpar}\end{nop}\ignorespacesafterend}
\newenvironment{fmathpar}{
\begin{nop}\footnotesize\begin{mathpar}}{
\end{mathpar}\end{nop}\ignorespacesafterend}
\newenvironment{alignS}{
\begin{nop}\begin{align}}{
\end{align}\end{nop}\ignorespacesafterend}
\newenvironment{salignS}{
\begin{nop}\small\begin{align}}{
\end{align}\end{nop}\ignorespacesafterend}
\newenvironment{falignS}{
\begin{nop}\footnotesize\begin{align*}}{
\end{align}\end{nop}\ignorespacesafterend}

% Stack formatting
\newenvironment{stackAux}[2]{%
\setlength{\arraycolsep}{0pt}
\begin{array}[#1]{#2}}{
\end{array}}
\newenvironment{stackCC}{
\begin{stackAux}{c}{c}}{\end{stackAux}}
\newenvironment{stackCL}{
\begin{stackAux}{c}{l}}{\end{stackAux}}
\newenvironment{stackTL}{
\begin{stackAux}{t}{l}}{\end{stackAux}}
\newenvironment{stackTR}{
\begin{stackAux}{t}{r}}{\end{stackAux}}
\newenvironment{stackBC}{
\begin{stackAux}{b}{c}}{\end{stackAux}}
\newenvironment{stackBL}{
\begin{stackAux}{b}{l}}{\end{stackAux}}

%APPENDIX
\newcommand{\caseL}[1]{\item[\textbf{case}] \textbf{#1}\newline}
\newcommand{\subcaseL}[1]{\item[\textbf{subcase}] \textbf{#1}\newline}

\newcommand{\todo}[1]{{\footnotesize \color{red}\textbf{[[ #1 ]]}}}


%% \makeatletter
%% \newcommand\definitionname{Lemma}
%% \newcommand\listdefinitionname{Proofs of Lemmas and Theorems}
%% \newcommand\listofdefinitions{%
%%   \section*{\listdefinitionname}\@starttoc{def}}
%% \makeatother



\newtheoremstyle{athm}{\topsep}{\topsep}%
      {\upshape}%         Body font
      {}%         Indent amount (empty = no indent, \parindent = para indent)
      {\bfseries}% Thm head font
      {}%        Punctuation after thm head
      {.8em}%     Space after thm head (\newline = linebreak)
      {\thmname{#1}\thmnumber{ #2}\thmnote{~\,(#3)}
% \addcontentsline{Lemma}{Lemma}
%   {\protect\numberline{\thechapter.\thelemma}#1}
      % \ifstrempty{#3}%
      {\addcontentsline{def}{section}{#1~#2~#3}}%
      % {\addcontentsline{def}{subsection}{\theathm~#3}}
\newline}%         Thm head spec

 \theoremstyle{athm}


% \newtheoremstyle{break}
%   {\topsep}{\topsep}%
%   {\itshape}{}%
%   {\bfseries}{}%
%   {\newline}{}%
% \theoremstyle{break}

%There are some problems with llncs documentcalss, so commenting these out until i find a solution
\newtheorem{thm}{Theorem}

%\spnewtheorem{thm1}[theorem]{Theorem}{\bfseries}{\upshape}
%\newenvironment{Theorem}[1][]{\begin{thm1}\iffirstargument[#1]\fi\quad\\}{\end{thm1}}

 \newtheorem{lem}[thm]{Lemma}
 \newtheorem{conjec}{Conjecture}
 \newtheorem{corr}[thm]{Corollary}
 \newtheorem{defn}{Definition}
 \newtheorem{prop}[thm]{Proposition}
 \newtheorem{assm}[thm]{Assumption}

\newtheorem{Eg}[thm]{Example}
\newtheorem{hypothesis}[thm]{Hypothesis}
\newtheorem{motivation}{Motivation}

% BNF symbols
\newcommand{\bnfalt}{{\bf \,\,\mid\,\,}}
\newcommand{\bnfdef}{{\bf ::=~}}

%% Highlighting
\newcommand{\hlm}[1]{\mbox{\hl{$#1$}}}

%% Provenance modes
\newcommand{\modifrcationProvenance}{{\bf MP}}
\newcommand{\updateProvenance}{{\bf UP}}

%Lemmas
\newcommand{\lemref}[1]{Lemma \ref{#1}} %name and number
\newcommand{\thmref}[1]{Theorem \ref{#1}} %name and number

\renewcommand{\labelenumii}{\theenumii}
\renewcommand{\theenumii}{\theenumi.\arabic{enumii}.}

\usepackage{enumitem}
\setenumerate{listparindent=\parindent}

\newlist{enumih}{enumerate}{3}
\setlist[enumih]{label=\alph*),before=\raggedright, topsep=1ex, parsep=0pt,  itemsep=1pt }

\newlist{enumconc}{enumerate}{3}
\setlist[enumconc]{leftmargin=0.5cm, label*= \arabic*.  , topsep=1ex, parsep=0pt,  itemsep=3pt }

\newlist{enumsub}{enumerate}{3}
\setlist[enumsub]{ leftmargin=0.7cm, label*= \textbf{subcase} \bf \arabic*: }

\newlist{enumsubsub}{enumerate}{3}
\setlist[enumsubsub]{ leftmargin=0.5cm, label*= \textbf{subsubcase} \bf \arabic*: }

\newlist{mainitem}{itemize}{3}
\setlist[mainitem]{ leftmargin=0cm , label= {\bf Case} }


\newenvironment{subproof}[1][\proofname]{%
  \renewcommand{\qedsymbol}{$\blacksquare$}%
  \begin{proof}[#1]%
}{%
  \end{proof}%
}


\newenvironment{nstabbing}
  {\setlength{\topsep}{0pt}%
   \setlength{\partopsep}{0pt}%
   \tabbing}
  {\endtabbing} 





%%% Local Variables:
%%% mode: latex
%%% TeX-master: "main"
%%% End:

\usepackage{eucal}
\usepackage{url}
\usepackage{tikz}
\usepackage{amsfonts,amsmath}
\usepackage{hyperref}
\begin{document}

\title{Verifying Snapping Mechanism - Connecting Ideal and Flopt}
\author{}

\date{}

\maketitle
In order to verify the differential privacy proeprty of
the snapping mechanism \cite{mironov2012significance},
we follow the logic rules designed from
\cite{barthe2016proving} and connecting 
it with the floating point computation semantics.
\section{Syntax}
\[\begin{array}{llll}
\mbox{Programs} & \prog & ::= & 
	%
     \varx = \expr ~|~ \varx \samplel \edistr
	%
	~|~ \prog ; \prog \\

\mbox{Expr.} & \expr & ::= & \rval 
	%
	~|~ \varx  ~|~ \expr \bop \expr
	%
	~|~ \uop (\expr) \\
%
\mbox{Binary Operation} & \bop & ::= & + ~|~ - ~|~ \times ~|~ \div \\
%
\mbox{Unary Operation} & \uop & ::= & \ln ~|~ - ~|~ \round{\cdot} 
	%
	~|~ \clamp_B(\cdot) \\
%
\mbox{Value} & \valv & ::= & \rval ~|~  \fval \\
%
\mbox{Distr.} & \edistr & ::= & \uniform(0, 1) 
%
	~|~ \uniform\{-1, 1\}\\ 
%
\mbox{Error} & \err & ::= & (\rval, \rval) \\
%
\mbox{Env.} & \trsmem & ::= & \cdot ~|~ \trsmem[x \mapsto (\fval, \err)] 
\end{array}
\]
where $\rval$ is in domain of real number $\real$, $\rval \in \real$ and $\fval$ is in domain of floating point number $\float$, $\fval \in \float$. Furthermore, the domain of floating point number $\float$ is a subset of $\real$ containing the real number that can be represented in the floating point computation.


%
\paragraph{Semantics.}
$\boxed{Env \times Expr. \trsto Value \times Value \times Value }$
%
\[
	\begin{array}{rcl}
	\sem{\expr}_{\trsmem}
	& \in &  
	\big\{(\fval, \rval_{l}, \rval_{u}) ~|~
	\exists ~~  
	\trsmem,  
	\expr \trsto (\fvalv, \rval_{l}, \rval_{u})\big\}
	\end{array}
\]
%
$
\boxed{Env \times \distr. \trsto 
\distr(Value \times Value \times Value )}
$
%
\[
	\begin{array}{rcl}
	\sem{\uniform(0, 1)}_{\trsmem}
	& \in & 
	\big\{
	(\fval, \rval_{l}, \rval_{u}) ~|~
	\fval \leftarrow \uniform(0, 1)^{\diamond}
	\land \rval_{l} = \rval_{u} = \fval
	\big\}\\
	\sem{\uniform\{-1, 1\}}_{\trsmem}
	& \in & 
	\big\{
	(-1, -1, -1), (1, 1, 1) ~|~
	each ~ w.p. ~ 0.5 
	\big\}\\	
	\end{array}
\]
%
$\boxed{Env \times prog \trsto \distr(Env)}$
\[
\begin{array}{rcl}
	%
	\sem{\varx \samplel \edistr}_{\trsmem}
	& = & 
	\elet (\fval, \rval_{l}, \rval_{u}) = \sem{\edistr}_{\trsmem}
	\ein 
	\unit{\trsmem[\varx \mapsto (\fval, \rval_{l}, \rval_{u})]}
	\\
	%
	\sem{\varx = \expr}_{\trsmem}
	& = &  
	~\unit{\trsmem[\varx \mapsto \sem{\expr}_{\trsmem}]}
	\\
	%
	\sem{\prog_1; \prog_2}_{\trsmem}
	& = &  \elet  \trsmem_1 = 
	\sem{\prog_1}_{\trsmem} \ein
	\sem{\prog_2}_{\trsmem_1} 
\end{array}
\]
%
In the semantics, 
%
$\trsmem, \expr \trsto (\fvalv, err)$ represents given an environment
%
$\trsmem$, the expression $\expr$
%
is transited to $\fvalv$ with error bound $err = (\rval_{l}, \rval_{u})$
in floating point transition semantics,
%
s.t. $\rval_{l} \leq \fvalv \leq \rval_{u}$.
%
$\trsmem, \prog \trsto \trsmem'$ represents given and environment $\trsmem$,
%
the program $\prog$ is transited to a new environment $\trsmem'$.
%
%
%
\section{Judgement and Validity}
%
\begin{defn}
[$(\epsilon, \delta)$ - lifting \cite{barthe2016proving}]
Two sub-distributions $\mu_1 \in \distr(U_1)$, $\mu_2 \in \distr(U_2)$are related by the $(\epsilon, \delta)$ - dilation lifting of $\Psi \subseteq U_1 \times U_2$, written $\mu_1 \Psi^{\#(\epsilon, \delta)} \mu_2$, if there exist two witness sub-distributions $\mu_L \in \distr(U_1 \times U_2)$ and $\mu_R \in \distr(U_1, U_2)$ s.t.:
\begin{enumerate}
	\item $\projl(\mu_L) = \mu_1$ and $\projr(\mu_R) = \mu_2$;
	\item $\supp(\mu_L) \subseteq \Psi$ and $\supp(\mu_R) \subseteq \Psi$; and
	\item $\Delta_{\epsilon}(\mu_L, \mu_R) \leq \delta$.
\end{enumerate}
\end{defn}
%
\begin{defn}[tagged variable]
Let $\mathcal{X}\langle 1 \rangle$ and $\mathcal{X}\langle 2 \rangle$ be the sets of tagged variables, finite sets of variable names tagged with $\langle 1 \rangle$ or $\langle 2 \rangle$ respectively:
\[
	\mathcal{X}\langle 1 \rangle = \{\varx\langle 1 \rangle ~|~ x \in \mathcal{X}\}
	~~
	\text{and}
	~~
	\mathcal{X}\langle 2 \rangle = \{\varx\langle 2 \rangle ~|~ x \in \mathcal{X}\},
\]
where $\mathcal{X}$ is a finite set of variable names.
\end{defn}
%
\paragraph{Assertion.} We consider a set $\mathcal{A}$ of assertions (predicates) from first order logic by the following grammar:
\[
\begin{array}{llll}
%
\mbox{Bool Expression} & \bexp & :: = & 
\pi_i(\expr\langle 1 / 2 \rangle) = \pi_i(\expr\langle 1 / 2 \rangle)  
 ~|~ \pi_i(\expr\langle 1 / 2 \rangle) < \pi_i(\expr\langle 1 / 2 \rangle) 
 ~|~ \pi_i(\expr\langle 1 / 2 \rangle) \leq \pi_i(\expr\langle 1 / 2 \rangle)
	\\
%
\mbox{Assert.} & \mathcal{A} & ::= & \top ~|~ \bot ~|~ \bexp 
	~|~ \bexp \land \bexp ~|~ \bexp \lor \bexp ~|~ \neg \bexp
\end{array}
\]
%
We typically use capital Greek letters ($\Phi, \Psi, \cdots$) for predicates. 
%
$\expr\langle 1 / 2 \rangle$ denotes an expression where program variables are tagged with $\langle 1 \rangle$ or $\langle 2 \rangle$.
%
$\pi_i(\expr\langle 1 / 2 \rangle)$ represents an expression where program variables are projected to the $i^{th}$ value from its triples, where $i \in \{1, 2, 3\}$.
%
\paragraph{Assertion Interpretation.} Assertions are interpreted as set of product environments. Let $\Phi$ be an assertion,
%
for example, when $\Phi \triangleq \pi_1(\expr_1\langle 1 \rangle) < \pi_1(\expr_2\langle 2 \rangle)$:
\[
	\sem{\Phi} = \{(\trsmem_1, \trsmem_2)
	~|~ \pi_1(\sem{\expr_1}_{\trsmem_1}) 
	< \pi_1(\sem{\expr_2}_{\trsmem_2}) \}.
\]
%
%
\paragraph{Judgment.}
The judgments are defined in following form:
\[
	\prog_1 \sim_{\epsilon} \prog_2: \Phi \Rightarrow \Psi.
\]
Here, $\prog_1$ and $\prog_2$ are programs and $\Phi$ and $\Psi$ are assertions on pairs of memories. Each assertions can refer to two copies $x\langle 1 \rangle, x\langle 2 \rangle$ of each program variable $x$, where these tagged variables refer to the value of x in the execution of $\prog_1$ and $\prog_2$ respectively.
%
\\
A judgment is valid, written $\vdash \prog_1 \sim_{\epsilon, \delta} \prog_2: \Phi_0 \Rightarrow \Phi$, 
if for any two environments $\trsmem_1$ and $\trsmem_2$ satisfying precondition $\Phi_0$, 
i.e., $(\trsmem_1, \trsmem_2) \in \sem{\Phi_0}$, there exists a lifting of $\Phi$ relating the output distributions: 
%
$(\sem{\prog_1}_{\trsmem_1})$ 
$\sem{\Phi}^{\#(\epsilon, \delta)}$ 
$(\sem{\prog_2}_{\trsmem_2})$.
%
\\
%
Fig. \ref{fig:aprhl} presents the main rules from apRHL+ \cite{barthe2016proving} excluding the while and condition rules which is not defined in out syntax, as well as the sampling rule, which we generalized in extended apRHL.
The rule in Fig. \ref{fig:aprhlplus} represents the lifting proved in soundness theorem.
%
\begin{figure}[ht]
\boxed{\vdash: prog \times prog \times \real \times Assert \times Assert, ~~ \Phi: Env \times Env \to bool}\\
\boxed{\pi_i: Value \times Value \times Value \to Value, ~~ i = 1, 2, 3}
\begin{mathpar}
\inferrule*[right = Unif]
{
\empty
}
{
	\vdash
	\varx_1 \samplel \uniform(0, 1) 	
	\sim_{\epsilon} 
	\varx_2 \samplel \uniform(0, 1)
	: 
	\top \Rightarrow  
	% (P \Rightarrow   Q)
	\rvalL_1 \leq \pi_1(\varx_2\langle 2 \rangle) \leq \rvalR_1 
	\land 
	\rvalL_2 \leq \pi_1 (\varx_1\langle 1 \rangle) \leq \rvalR_2
	\land
	e^{-\epsilon} \leq 
	\frac{\rvalR_1 - \rvalL_1}{\rvalL_2 - \rvalR_2}
	\leq e^{\epsilon}
}
\and
\inferrule
{
\empty
}
{
	\vdash 
	\varx_1 \samplel \uniform(0, 1) 	
	\sim_{0} 
	\varx_2 \samplel \uniform(0, 1) 
	: \top \Rightarrow 
	\pi_1 (\varx_2\langle 2 \rangle) 
	= \pi_1 (\varx_1\langle 1 \rangle )
	\land \pi_2 (\varx_2\langle 2 \rangle) = \pi_2 (\varx_1\langle 1 \rangle)
	\land \pi_3 (\varx_2\langle 2 \rangle) = \pi_3 (\varx_1\langle 1 \rangle)
}~\textbf{null1}
\and
\inferrule
{
\empty
}
{
	\vdash 
	\varx_1 \samplel \uniform\{-1, 1\} 	
	\sim_{0} 
	\varx_2 \samplel \uniform\{-1, 1\}  
	: \top \Rightarrow 
	\pi_1 (\varx_2\langle 2 \rangle) = \pi_1 (\varx_1\langle 1 \rangle )
	\land \pi_2 (\varx_2\langle 2 \rangle) = \pi_2 (\varx_1\langle 1 \rangle)
	\land \pi_3 (\varx_2\langle 2 \rangle) = \pi_3 (\varx_1\langle 1 \rangle)
}~\textbf{null}
\and
\inferrule*[right = round]
{
\empty
}
{
	\vdash 
	\varx_1 \samplel \round{\vary_1}_{\Lambda}	
	\sim_{0} 
	\varx_2 \samplel \round{\vary_2}_{\Lambda}
	: \vary_1 \langle 1 \rangle \lameq \vary_2 \langle 2 \rangle
	\Rightarrow 
	\pi_1 (\varx_2 \langle 2 \rangle) = \pi_1 (\varx_1 \langle 1 \rangle) = k
}
\end{mathpar}
\caption{Rules Extended from apRHL+}
\label{fig:aprhlplus}
\end{figure}
%
\begin{figure}[ht]
\begin{mathpar}
\inferrule*[right = Assn]
{
\empty
}
{
	\vdash 
	\varx_1 = \expr_1  
	\sim_{0} 
	\varx_2 = \expr_2  
	: \Phi[\expr_1/\varx_1\langle 1 \rangle]
	[\expr_2/\varx_2\langle 2 \rangle]  \Rightarrow \Phi
}
~~~
\inferrule*[right = Seq]
{
\prog_1 \sim_{\epsilon} \prog_2 : \Phi_1 \Rightarrow \Phi'_1
\\
\prog'_1 \sim_{\epsilon'} \prog'_2 : \Phi'_1 \Rightarrow \Phi_2
}
{
	\vdash 
	\prog_1; \prog'_1  
	\sim_{\epsilon + \epsilon'} 
	\prog_2; \prog'_2
	: \Phi_1  \Rightarrow  \Phi_2
}
\and
\inferrule*[right = Conseq]
{
\prog_1 \sim_{\epsilon} \prog_2 : \Phi'_1 \Rightarrow \Phi'_2
\and
\Phi_1 \Rightarrow \Phi'_1
\and 
\Phi'_2 \Rightarrow \Phi_2
\and 
\epsilon \leq \epsilon'
}
{
\prog_1 \sim_{\epsilon'} \prog_2 : 
\Phi_1 \Rightarrow \Phi_2
}
\end{mathpar}
\caption{Proving Rules from apRHL}
\label{fig:aprhl}
\end{figure}
%
%
\newpage
\begin{thm}
\label{thm:unif_coupling}
Let $\edistr_1 \in \distr(\float \times \real \times \real)$, $\edistr_2 \in \distr(\float \times \real \times \real)$ are defined:
\[
	\edistr_1(\fvarx, \rvarx_l, \rvarx_u) 
	= \sem{\uniform(0, 1)}_{[]},
~~
	\edistr_1(\fvary, \rvary_l, \rvary_u) 
	= \sem{\uniform(0, 1)}_{[]}
\]
where $\uniform$ is uniform distribution over $[0, 1)$ whoes $\pdf.$ is defined as:
\[
	\pdf_{\uniform}(\fvarx, \rvarx_l, \rvarx_u) = 
	\begin{cases}
	1 & \fvarx \in [0, 1)^{\mathbb{F}}\\
	0       & o.w.
	\end{cases},
	~~
	\pdf_{\uniform}(\fvary, \rvary_l, \rvary_u) = 
	\begin{cases}
	1 & \fvary \in [0, 1)^{\mathbb{F}}\\
	0       & o.w.
	\end{cases}.
\]
Then, $	\sem{\varx_1 \samplel \edistr_1}_{\trsmem_1} 
		~ \sem{\Phi}^{\#(\epsilon, 0)} ~
		\sem{\varx_2 \samplel \edistr_2}_{\trsmem_2}$, 
where
\[
\begin{array}{rl}
	\sem{\Phi} = &
	\{((\fval_1, \rval_{1l}, \rval_{1u}), (\fval_2, \rval_{2l}, \rval_{2u})) 
	\in (\float \times \real \times \real) \times (\float \times \real \times \real)\\
	& ~|~
	\exists ~ \rvalL_1, \rvalL_2, \rvalR_1, \rvalR_2. ~
	\rvalL_1 \leq \fval_1 \leq \rvalR_1 
	\land 
	\rvalL_2 \leq \fval_1 \leq \rvalR_2
	\land
	e^{-\epsilon} \leq 
	\frac{\rvalR_1 - \rvalL_1}{\rvalL_2 - \rvalR_2}
	\leq e^{\epsilon}
	\}
\end{array}
\]
\end{thm}	
\begin{proof}[Proof of Theorem \ref{thm:unif_coupling}]
%
%
Let $\mu_L, \mu_R \in \distr((\float \times \real \times \real) \times (\float \times \real \times \real))$ be the two witness distribution s.t.:
\[
	{\mu_L}(x, y) = 
	\begin{cases}
	{\uniform}(x) & \pi_1(x) \cdot e^{-\epsilon} = \pi_1(y) \land \pi_1(x) \in [0, 1)\\
	0       & o.w.
	\end{cases},
~~~
	{\mu_R}(x, y) = 
	\begin{cases}
	{\uniform}(y) & \pi_1(x) \cdot e^{-\epsilon} = \pi_1(y) \land \pi_1(y) \in [0, 1)\\
	0       & o.w.
	\end{cases},
\]
%
%
where $\varx = (\fvarx, \rvarx_l, \rvarx_u) \in (\float \times \real \times \real)$, $\vary = (\fvary, \rvary_l, \rvary_u ) \in (\float \times \real \times \real)$ and their $\pdf.$ are defined as:
\[
	\pdf_{\mu_L}(x, y) = 
	\begin{cases}
	\pdf_{\uniform}(x) & \pi_1(x) \cdot e^{-\epsilon} = \pi_1(y) \land \pi_1(x) \in [0, 1)\\
	0       & o.w.
	\end{cases},
\]
\[
	\pdf_{\mu_R}(x, y) = 
	\begin{cases}
	\pdf_{\uniform}(y) & \pi_1(x) \cdot e^{-\epsilon} = \pi_1(y) \land \pi_1(y) \in [0, 1)\\
	0       & o.w.
	\end{cases}.
\]
It is enough to show the 2 witnesses $\edistr_L(x, y)$ and $\edistr_R(x, y)$ are satisfying following three requirements:
\begin{itemize}
	\item $\supp(\mu_L) \in \Psi \land \supp(\mu_R) \in \Psi$

	\begin{itemize}
		\item $\supp(\mu_L) \subseteq \Psi$ 
		%
		\\
		%
		Let $\rvalL_1 = \fvarx$, $\rvalR_1 = 2\fvarx$, $\rvalL_2 = \fvary$ and $\rvalR_2 = 2\fvary$. 
		By definition of the $\pdf$ of $\mu_L$, we have: 
		$\rvalL_1 \leq \fvarx \leq \rvalR_1 
		\land \rvalL_2 \leq \fvary \leq \rvalR_2
		\land e^{-\epsilon} \leq 
		\frac{\rvalR_1 - \rvalL_1}{\rvalL_2 - \rvalR_2}
		\leq e^{\epsilon} \in \Psi$ where $\pdf_{\mu_R}(x, y) = \pdf_{\uniform}(x) > 0$.
		%
		\\%
		%
		Then we can derive $\supp(\mu_L) \in \Psi$
		%
		\item $\supp(\mu_R) \subseteq \Psi$
		%
		\\
		%
		Let $\rvalL_1 = x$, $\rvalR_1 = 2x$, $\rvalL_2 = y$ and $\rvalR_2 = 2y$. 
		By definition of the $\pdf$ of $\mu_L$, we have: 
		$\rvalL_1 \leq x \leq \rvalR_1 
		\land \rvalL_2 \leq y \leq \rvalR_2
		\land e^{-\epsilon} \leq 
		\frac{\rvalR_1 - \rvalL_1}{\rvalL_2 - \rvalR_2}
		\leq e^{\epsilon} \in \Psi$ when $\pdf_{\mu_R}(x, y) = \pdf_{\uniform}(y) > 0$.
		%
		\\
		%
		Then we can derive $\supp(\mu_L) \in \Psi$
		%
	\end{itemize}		
%
	\item $\projl(\mu_L) = \mu_1 \land \pi_2(\mu_R) = \mu_2$
	
	\begin{itemize}
		\item $\projl(\mu_L) = \mu_1$ 

		% Equivalent to show $\pdf_{\projl(\mu_L)}  = \pdf_{\mu_1}$.

		By definition of the $\projl$ and $\pdf$ of $\mu_L$, we have $\forall x = (\fvarx, \rvarx_l, \rvarx_u) \in \float \times \real \times \real$:
		\[
			\pdf_{\projl(\mu_L)}(x) = 
			\begin{cases}
			\int_{y}\pdf_{\uniform}(x) 
			& \fvarx \cdot e^{-\epsilon} = \pi_1(\fvary) 
			\land \fvarx \in [0, 1)\\
			0       & o.w.
			\end{cases} 
			= 
			\begin{cases}
			\pdf_{\uniform}(x) & \fvarx \in [0, 1)\\
			0       & o.w.
			\end{cases}
			=
			\pdf_{\mu_1}(x)
		\]

		\item $\projl(\mu_R) = \mu_2$ 

		Equivalent to show$\pdf_{\projr(\mu_R)}  = \pdf_{\mu_2}$.

		By definition of the $\projr$ and $\pdf$ of $\mu_R$, we have $\forall y = (\fvary, \rvary_l, \rvary_u) \in \float \times \real \times \real$:
		\[
			\pdf_{\projr(\mu_R)}(y) = 
			\begin{cases}
			\int_{x}\pdf_{\uniform}(y) 
			& \fvarx \cdot e^{-\epsilon} = \pi_1(\fvary) 
			\land \fvary \in [0, 1)\\
			0       & o.w.
			\end{cases} 
			= 
			\begin{cases}
			\pdf_{\uniform}(y) & \fvary \in [0, 1)\\
			0       & o.w.
			\end{cases}
			=
			\pdf_{\mu_2}(y)
		\]
	\end{itemize}	

	\item $\Delta_{\epsilon}(\mu_L, \mu_R) \leq 0$

	By definition of $\epsilon-$DP divergence, we have:
	 \[
	 \begin{array}{ll}
	 \Delta_{\epsilon}(\mu_L, \mu_R) 
	 & = \underset{S}{\psup}
	 \Big(
	 \pr{(x,y) \samplel \mu_L}{(x,y) \in S} - e^{\epsilon} \pr{(x,y) \samplel \mu_R}{(x,y) \in S}
	 \Big) \\
	 & =\underset{S}{\psup}
	 \Big(
	 \int_{(x,y) \in S} \pdf_{\mu_L}(x, y) - e^{\epsilon} \int_{(x,y) \in S} \pdf_{\mu_R}(x, y)
	 \Big)	 
	 \end{array}
	 \]
	 \begin{itemize}
	 	\item[{\bf case}]
	 	%
	 	$S \subseteq \{(x, y) ~|~ 
	 	\pi_1(x) \in [0, 1) \land \pi_1(x) \cdot e^{-\epsilon} 
	 	= \pi_1(y)\}$:
	 	%
		\[
		 \begin{array}{ll}
		 \Delta_{\epsilon}(\mu_L, \mu_R) 
		 & = 
		 \int_{(x,y) \in S} \pdf_{\uniform}(x) - e^{\epsilon} \int_{(x,y) \in S} \pdf_{\uniform}(y)\\
		 & = 
		 \int_{(x,y) \in S} \pdf_{\uniform}(x) - e^{\epsilon} \int_{(x,y) \in S} \pdf_{\uniform}(x * e^{-\epsilon})\\ 
		 & = 
		 \int_{(x,y) \in S} \pdf_{\uniform}(x) - e^{\epsilon}* e^{-\epsilon} \int_{(x,y) \in S} \pdf_{\uniform}(x) 
		 = 0 
		 \end{array}
		 \]
		 %
	 	\item[{\bf case}] $S \subseteq \{(x, y) 
	 	~|~ \pi_1(x) \in [1, e^{\epsilon}) 
	 	\land \pi_1(x) \cdot e^{-\epsilon} = \pi_1(y)\}$:
	 	%
		 \[
		 \Delta_{\epsilon}(\mu_L, \mu_R) 
		 = 
		 0 - e^{\epsilon} \int_{(x,y) \in S} \pdf_{\uniform}(y) <0
		 \]
	 	\item[{\bf case}] o.w.:
	 	%
		 \[
		 \Delta_{\epsilon}(\mu_L, \mu_R) = 0 - 0 =  0 
		 \]	 	

	 \end{itemize}

\end{itemize}
\end{proof}
%
%
%
%
%
%
\clearpage
\begin{thm}[Soundness]
 $\forall \prog_1$, $\prog_2$,  $ \vdash \prog_1	
\sim_{\epsilon, \delta} 
\prog^2 :
\Phi_0 \Rightarrow \Phi $,    $\forall \trsmem_1$, $\trsmem_2$ 
s.t $\Phi_0$: 
$\trsmem_1 ~ \sem{\Phi_0} ~ \trsmem_2$,
then
$$ 
(\sem{\prog_1}_{\trsmem_1})  
\sem{\Phi}^{\#(\epsilon, \delta)} 
(\sem{\prog_2}_{\trsmem_2}) 
$$.
\end{thm}



\begin{proof}
By induction on the transition judgement $\vdash \prog_1	
\sim_{\epsilon, \delta} 
\prog_2 :
\Phi_0 \Rightarrow \Phi $.
\begin{itemize}
\caseL{[\textsc{Unif}]}
	In this case, 
	$\sem{\Phi} = 
	\{((\fval_1, \rval_{1l}, \rval_{1u}), (\fval_2, \rval_{2l}, \rval_{2u})) 
	\in (\float \times \real \times \real) \times (\float \times \real \times \real)
	~|~
	\expr^{-\epsilon} \rval_{2l} \leq \rval_{1u}
	\land
	\expr^{\epsilon} \rval_{1l} \leq \rval_{2u}
	\}$
	%
	\\
	%
	By the semantics of sampling, we have:\\
	%
	$\sem{\varx_1 \samplel \edistr_1}_{\trsmem_1} = 
	\elet (\fval_1, \rval_{1l}, \rval_{1u}) 
	= \sem{\edistr_1}_{\trsmem_1} 
	\ein 
	\unit{\trsmem[\varx \mapsto (\fval_1, \rval_{1l}, \rval_{1u})]}
	$ and 
	%
	$\sem{\varx_2 \samplel \edistr_2}_{\trsmem_2} = 
	\elet (\fval_2, \rval_{2l}, \rval_{2u}) 
	= \sem{\edistr_2}_{\trsmem_2} 
	\ein 
	\unit{\trsmem[\varx \mapsto (\fval_2, \rval_{2l}, \rval_{2u})]}
	$. \\
	%
	Then by the semantics of $\sem{\uniform(0, 1)}_{\trsmem}$ we have:
	\\
	%
	$ \rval_{1l} = \fval_1 = \rval_{1u} ~ (1)$ 
	and
	$ \rval_{2l} = \fval_2 = \rval_{2u} ~ (2)$.
	%
	\\
	%
	Then, it is enough to show:
	${\edistr_1}(x) ~ \Psi^{\#(\epsilon, 0)} ~ {\edistr_1}(y)$, where
	\[
		\Psi = \{(\fval_1, \fval_2) \in \float \times \float
		| 
		\fval_1 e^{-\epsilon} 
		\leq \fval_2
		\leq \fval_1 e^{\epsilon} \}.
	\]
	%
	Which is proved in the Theorem \ref{thm:unif_coupling}.
\caseL{[\textsc{Null}] }
	In this case, 
	$\sem{\Phi} = 
	\{((\fval_1, \rval_{1l}, \rval_{1u}), (\fval_2, \rval_{2l}, \rval_{2u})) 
	\in (\float \times \real \times \real) \times (\float \times \real \times \real)
	|
	\rval_{2l} = \rval_{1l}
	\land
	\rval_{2u} = \rval_{1u}
	\land
	\fval_1 = \fval_2
	\}$
	%
	\\
	%
	By the semantics of sampling, we have:\\
	%
	$\sem{\varx_1 \samplel \edistr_1}_{\trsmem_1} = 
	\elet (\fval_1, \rval_{1l}, \rval_{1u}) 
	= \sem{\edistr_1}_{\trsmem_1} 
	\ein 
	\unit{\trsmem[\varx \mapsto (\fval_1, \rval_{1l}, \rval_{1u})]}
	$ and 
	%
	$\sem{\varx_2 \samplel \edistr_2}_{\trsmem_2} = 
	\elet (\fval_2, \rval_{2l}, \rval_{2u}) 
	= \sem{\edistr_2}_{\trsmem_2} 
	\ein 
	\unit{\trsmem[\varx \mapsto (\fval_2, \rval_{2l}, \rval_{2u})]}
	$. \\
	%
	Then by the semantics of $\sem{\uniform(0, 1)}_{\trsmem}$ we have:
	\\
	%
	$ \rval_{1l} = \fval_1 = \rval_{1u} ~ (1)$ 
	and
	$ \rval_{2l} = \fval_2 = \rval_{2u} ~ (2)$.
	%
	\\
	%
	Then, it is enough to show:
	${\edistr_1}(x) ~ \Psi^{\#(0, 0)} ~ {\edistr_1}(y)$, where
	\[
		\Psi = \{(\fval_1, \fval_2) \in \float \times \float
		~| ~
		\fval_1 = \fval_2 \}.
	\]
	%
	Which is obviously.
	% \caseL{[\textsc{Assn}] }
	% \caseL{[\textsc{Seq}] }
	% \caseL{[\textsc{ConSeq}] }
	\end{itemize}
\end{proof}


\newpage
\section{Snapping Mechanism}

\begin{defn}
[$\snap(a) : A \to \distr(\real)$]
Given privacy parameter $\epsilon$, the Snapping mechanism $\snap(a)$ is defined as:
\[
	\varu \samplel \uniform(0,1); s \samplel \uniform\{-1, 1\};
	\varx = f(a) + \frac{1}{\epsilon} \times s \times \ln (\varu);
	\vary = \round{\varx}_{\Lambda};
	\varz = \clamp_B (\vary)
\]
where $f(a)$ represents a value that the query function $f$ be evaluated over input database $a \in A$, $\epsilon$ is the privacy parameter, $B$ is the clamping argument and $\Lambda$ is the rounding argument satisfying $\lambda = 2^k$ where $2^k$ is the smallest power of 2 greater or equal to the $\frac{1}{\epsilon}$.
\end{defn}

\begin{defn}[$\Lambda$ equivalent]
Given two floating point values $\valv_1$ and $\valv_2$, if for some floating point value $\fvalv$ which is a multiple of $\Lambda$:
\[
	\fvalv - \frac{\Lambda}{2} \leq \valv_1 < \fvalv + \frac{\Lambda}{2}
	~~
	\land
	~~
	\fvalv - \frac{\Lambda}{2} \leq \valv_2 < \fvalv + \frac{\Lambda}{2},	
\]
then $\valv_1$ and $\valv_2$ are $\Lambda$ equivalent, i.e., 
$\valv_1 \lameq \valv_2 \lameq \fvalv$.
\end{defn}
%
% \begin{lem}[]
% \label{lem:xisLameq}
% Given the environment $\trsmem_1 = [\varu_1 \to (\fvalv_1, \rvalv_1, \rvalv_1)]$, $\trsmem_1 = [\varu_2 \to (\fvalv_2, \rvalv_2, \rvalv_2)]$ and $f(a) + 1 = f(a')$, if 
% \begin{itemize}
% 	\item $\fvalv < 0$, 
% 	$e^{\epsilon 
% 		\big((\valv - \frac{\Lambda}{2})(1 + \eta)^3 
% 		- f(a) (1 + \eta)^2) \big)}
% 	\leq \fvalv_1 \leq 
% 	e^{\epsilon 
% 		(\frac{(\frac{\valv + \frac{\Lambda}{2}}{1 + \eta} - f(a))}{(1 + \eta)^2})}$,
% 	$e^{\epsilon 
% 		\big((\valv - \frac{\Lambda}{2})(1 + \eta)^3 
% 		- f(a') (1 + \eta)^2) \big)}
% 	\leq \fvalv_2 \leq 
% 	e^{\epsilon 
% 		(\frac{(\frac{\valv + \frac{\Lambda}{2}}{1 + \eta} - f(a))}{(1 + \eta)^2})}$, 
% 	Or
% 	\item
% 	$\fvalv = 0$, 
% 	$e^{\epsilon 
% 		\big((\valv - \frac{\Lambda}{2})(1 + \eta)
% 		- f(a) (1 + \eta)^2) \big)}
% 	\leq \fvalv_1 \leq 
% 	e^{\epsilon 
% 		((\frac{\valv + \frac{\Lambda}{2}}{1 + \eta} - \frac{f(a))}{(1 + \eta)^2})}$,
% 	$e^{\epsilon 
% 		\big((\valv - \frac{\Lambda}{2})(1 + \eta) 
% 		- f(a') (1 + \eta)^2) \big)}
% 	\leq \fvalv_2 \leq 
% 	e^{\epsilon 
% 		((\frac{\valv + \frac{\Lambda}{2}}{1 + \eta} - \frac{f(a))}{(1 + \eta)^2})}$, 
% \end{itemize}
% then we have:
% 	\[
% 	\pi_1(\sem{\varx_1 = f(a) + \frac{1}{\epsilon}\ln(\varu_1)}_{\trsmem_1}(\varx_1))
% 	\lameq 
% 	\pi_1(\sem{\varx_2 = f(a') + \frac{1}{\epsilon}\ln(\varu_2)}_{\trsmem_2}(\varx_2))	
% 	\]	
% \end{lem}

\begin{thm}[The $\snap$ mechanism is 
$\epsilon-$differentially private]
\end{thm}
\begin{proof}
Induction on the output space of $\snap(a)$ mechanism, let $\valv$ be the output of $\snap(a)$, we have following cases:
\begin{itemize}
	\caseL{$\boldsymbol{\valv = -B}$}
%
Let $b$ be the largest number rounded by $\Lambda$ that is smaller than $B$, $b' = b + \Lambda / 2$.
%
\\
Let $\rvalL_1 = \rvalL_2 = 0$, 
$\rvalR_1 = e^{\epsilon 
		(\frac{(\frac{-b'}{1 + \eta} - f(a))}{(1 + \eta)^2})}$
and $\rvalR_2 = e^{\epsilon 
		(\frac{(\frac{-b'}{1 + \eta} - f(a'))}{(1 + \eta)^2})}$.
By the calculation in the Flopt Version proof, we have:
\[
	e^0 < \frac{\rvalR_1 - \rvalL_1}{\rvalR_2 - \rvalL_2}
	< e^{\epsilon(10.1 \eta B + 1 + 2.1\eta)} < e^{\epsilon(23 \eta B + 1)}.
\]
Then, we have following derivation for this case:

{\tiny
\begin{mathpar}
\inferrule*[right = Seq]
{
	\inferrule*[right = Seq]
	{
	\inferrule[Unif]
	{
	\Phi = 
	\rvalL_1 \leq \pi_1(\varu_1) \leq \rvalR_1 
	\land 
	\rvalL_2 \leq \pi_1(\varu_2) \leq \rvalR_2
	}
	{
		\varu_1 \samplel \uniform(0,1)
		\sim_{\epsilon(23 \eta B + 1)} 
		\varu_2 \samplel \uniform(0,1)
		: 
		\top
		\Rightarrow 
		\Phi
	}
	~~
	\inferrule[Null]
	{
	\empty
	}
	{
		s_1 = \samplel \uniform\{-1, 1\}
		\sim_{0} 
		s_2 = \samplel \uniform\{-1, 1\}
		: 
		\Phi
		\Rightarrow 
		\Phi \land \pi_1(s_1) = \pi_1(s_2)
	}
	}
	{
		\varu_1 \samplel \uniform(0,1); s_1 = \samplel \uniform\{-1, 1\}
		\sim_{\epsilon(23 \eta B + 1)} 
		\varu_2 \samplel \uniform(0,1); s_2 = \samplel \uniform\{-1, 1\}
		: 
		\top \Rightarrow \Phi \land \pi_1(s_1) = \pi_1(s_2) = \valv
	}
\\
\Pi_R
}
{
	\vdash 
	\varu_1 \samplel \uniform(0,1); s_1 = \samplel \uniform\{-1, 1\};
	\varx_1 = f(a) + \frac{1}{\epsilon} \times s_1 \times \ln (\varu_1);
	\vary_1 = \round{\varx_1}_{\Lambda};
	\varz_1 = \clamp_B (\vary_1)
	\\
	\sim_{\epsilon(23 \eta B + 1)} 
	\varu_2 \samplel \uniform(0,1); s_2 = \samplel \uniform\{-1, 1\};
	\varx_2 = f(a') + \frac{1}{\epsilon} \times s_2 \times \ln (\varu_2)
	\vary_2 = \round{\varx_2}_{\Lambda};
	\varz_2 = \clamp_B (\vary_2)
	: \\
	\valv = -B \land f(a) + 1 = f(a')  \land
	\Rightarrow  \pi_1(\varx_1) = \pi_1(\varx_2) = \valv
}
\\
\Pi_R:
\\
\inferrule*[right = Seq]
{
\inferrule*[right = ConSeq]
{	
\inferrule*[right = Assn]
	{
		\empty
	}
	{
		\varx_1 = 
		f(a) + \frac{1}{\epsilon} \times s_1 \times \ln (\varu_1)
		\sim_{0} 
		\varx_2 =
		f(a') + \frac{1}{\epsilon} \times s_2 \times \ln (\varu_2)
		:
		\Phi \land \pi_1(s_1) = \pi_1(s_2)
		\Rightarrow
		\Phi'
	}
\and
\Phi' \Rightarrow \pi_1(\varx_1) \lameq \pi_1(\varx_2) 
\\
\Phi \land \pi_1(s_1) = \pi_1(s_2)
\Rightarrow \Phi \land \pi_1(s_1) = \pi_1(s_2)
}
{
	\varx_1 = f(a) + \frac{1}{\epsilon} \times s_1 \times \ln (\varu_1)
	\sim_{0} 
	\varx_2 = f(a') + \frac{1}{\epsilon} \times s_2 \times \ln (\varu_2)
	:
	f(a) + 1 = f(a') \land \Phi \land \pi_1(s_1) = \pi_1(s_2)
	\Rightarrow \pi_1(\varx_1) \lameq \pi_1(\varx_2)
}
\and
\Delta_R
}
{
	\varx_1 = f(a) + \frac{1}{\epsilon} \times s_1 \times \ln (\varu_1);
	\vary_1 = \round{\varx_1}_{\Lambda};
	\varz_1 = \clamp(\vary_1)
	\sim_{0} 
	\varx_2 = f(a') + \frac{1}{\epsilon} \times s_2 \times \ln (\varu_2);
	\vary_2 = \round{\varx_2}_{\Lambda};
	\varz_2 = \clamp(\vary_2)
	:
	f(a) + 1 = f(a') \land \Phi \land \pi_1(s_1) = \pi_1(s_2)
	\Rightarrow \pi_1(\varx_1) = \pi_1(\varx_2)
}
\\
\Delta_R:
\\
\inferrule*[right = Seq]
{
	\inferrule*[right = round]
	{
		\empty
	}
	{
		\vary_1 = \round{\varx_1}_{\Lambda}
		\sim_{0}
		\vary_2 = \round{\varx_2}_{\Lambda}
		: \pi_1(\varx_1) \lameq \pi_1(\varx_2) 
		\Rightarrow \pi_1(\varx_1) = \pi_1(\varx_2)
	}
	\and
	\inferrule*[right = null]
	{
		\empty
	}
	{
		\varz_1 = \clamp(\vary_1)
		\sim_{0}
		\varz_2 = \clamp(\vary_2)
		: \pi_1(\varx_1) = \pi_1(\varx_2) 
		\Rightarrow \pi_1(\varx_1) = \pi_1(\varx_2)
	}
}
{
	\vary_1 = \round{\varx_1}_{\Lambda};
	\varz_1 = \clamp(\vary_1)
	\sim_{0} 
	\vary_2 = \round{\varx_2}_{\Lambda};
	\varz_2 = \clamp(\vary_2)
	:
	\pi_1(\varx_1) \lameq \pi_1(\varx_2)
	\Rightarrow \pi_1(\varx_1) = \pi_1(\varx_2)
}
\end{mathpar}
}
%
The $\Phi' \Rightarrow \pi_1(\varx_1) \lameq \pi_1(\varx_2)$ is proved as following:
%
\\
By the program semantics, we have:
\\
{\tiny
\[
\begin{array}{ll}
	& \sem{\varu_1 \samplel \uniform(0,1); s_1 = \samplel \uniform\{-1, 1\};
		\varx_1 = f(a) + \frac{1}{\epsilon} \times s_1 \times \ln (\varu_1)}_{[]} =
	\\
	&
	 \elet (\fval_1, \rval_1, \rval_1 ) = \sem{\uniform(0,1)}_{[]} 
	 \ein \elet (1, 1, 1) = \sem{\uniform\{-1, 1\}} \ein
	 [\varu_1 \mapsto (\fval_1, \rval_1, \rval_1); 
	 s_1 \mapsto (1, 1, 1); 
	 \varx_1 \mapsto \bigg(
				f(a) + \frac{1}{\epsilon} \times \ln(\fval_1),
				%
				 (f(a) + 
				(\frac{1}{\epsilon} \times \ln(\rval_1))
				(1 + \eta)^2)
				{(1 + \eta)},
				%
				\frac{(
				f(a) + \frac{\frac{1}{\epsilon} 
				\times \ln(\rval_1)}
				{(1 + \eta)^2}
				)}
				{(1 + \eta)}
				\bigg)]
	\\
	\sim
	&
	\sem{\varu_2 \samplel \uniform(0,1); s_2 = \samplel \uniform\{-1, 1\};
		\varx_2 = f(a') + \frac{1}{\epsilon} \times s_2 \times \ln (\varu_2)}
	\\
	&
	\elet (\fval_2, \rval_2, \rval_2) = \sem{\uniform(0,1)}_{[]}
	\ein \elet (1, 1, 1) = \sem{\uniform\{-1, 1\}} \ein
	[\varu_2 \mapsto (\fval_2, \rval_2, \rval_2);
	 s_2 \mapsto (1, 1, 1); 
	 \varx_2 \mapsto \bigg(
				f(a) + \frac{1}{\epsilon} \times \ln(\fval_2),
						%
				\big( (f(a') + 
				(\frac{1}{\epsilon} \times \ln(\rval_2))
				(1 + \eta)^2)
				{(1 + \eta)},
				%
				\frac{(
				f(a') + \frac{\frac{1}{\epsilon} 
				\times \ln(\rval_2)}
				{(1 + \eta)^2}
				)}
				{(1 + \eta)}
				 \big)
				\bigg)]	
	\end{array}
\]
}
%
 We also have $\pi_1(\varu_1) = \pi_2(\varu_1) = \pi_3(\varu_1)$, 
	$\pi_1(\varu_2) = \pi_2(\varu_2) = \pi_3(\varu_2)$,
	$\pi_2(\varx_1) \leq \pi_1(\varx_1) \leq \pi_3(\varx_1)$ and 
	$\pi_2(\varx_2) \leq \pi_1(\varx_2) \leq \pi_3(\varx_2)$ by the semantics, 
	since $\Phi' = 	
\rvalL_1 \leq \pi_1(\varu_1) < \rvalR_1 
	\land 
	\rvalL_2 \leq \pi_1(\varu_2) < \rvalR_2$, 
	we can get:
{\tiny
\[
\begin{array}{rcl}
	(f(a) + (\frac{1}{\epsilon} \times \ln(\rvalL_1))
	(1 + \eta)^2){(1 + \eta)}
	\leq \pi_1(\varx_1) \leq 
	\frac{(f(a) + \frac{\frac{1}{\epsilon} 
	\times \ln(\rvalR_1)}{(1 + \eta)^2})}{(1 + \eta)}
	& ~ \sim ~ &
	(f(a') + (\frac{1}{\epsilon} \times \ln(\rvalL_2))
	(1 + \eta)^2){(1 + \eta)} 
	\leq \pi_1(\varx_2) \leq 
	\frac{(f(a') + \frac{\frac{1}{\epsilon} 
	\times \ln(\rvalR_2)}{(1 + \eta)^2})}{(1 + \eta)}
	\\
	0 \leq \pi_1(\varx_1) \leq  b + \frac{\Lambda}{2}
	& ~ \sim ~ &
	0 \leq \pi_1(\varx_2) \leq b + \frac{\Lambda}{2}
	\\
	\pi_1(\varx_1) & ~ \lameq ~ & \pi_1(\varx_2)
\end{array}
\]
}
%
%
\caseL{
	$\boldsymbol{\valv \in (-B, \round{f(a)}_{\Lambda})} ~ (\star) $
	}
	%
	\subcaseL{
	$\boldsymbol{\round{f(a)}_{\Lambda} \leq 0 
	\lor \bigg( \round{f(a)}_{\Lambda} > 0 \land \valv \in (-B, 0) \bigg) } ~ (\star_1) $
	}
	%
	Let $\valv_1 = \valv - (\frac{\Lambda}{2})$,
		$\valv_2 = \valv + (\frac{\Lambda}{2})$, 
		we know $\valv_1 < 0$, $\valv_2 < 0$.
\\
Let $\rvalL_1 = e^{\epsilon 
				\big( (\valv_1(1 + \eta) - f(a)) (1 + \eta)^2) \big)}$,
$\rvalL_2 = e^{\epsilon 
				\big( (\valv_1(1 + \eta) - f(a')) (1 + \eta)^2) \big)}$, 
$\rvalR_1 = e^{\epsilon 
		(\frac{(\frac{\valv_2}{1 + \eta} - f(a))}{(1 + \eta)^2})}$
and $\rvalR_2 = e^{\epsilon 
		(\frac{(\frac{\valv_2}{1 + \eta} - f(a'))}{(1 + \eta)^2})}$.
By the calculation in the Flopt Version proof, we have:
\[
	e^0 < \frac{\rvalR_1 - \rvalL_1}{\rvalR_2 - \rvalL_2}
	< e^{\epsilon(23 \eta B + 1)}.
\]
Then, we have following derivation for this case:

\tiny{
\begin{mathpar}
\inferrule*[right = Seq]
{
	\inferrule*[right = Seq]
	{
	\inferrule[Unif]
	{
	\Phi = 
	\rvalL_1 \leq \fval_1 \leq \rvalR_1 
	\land 
	\rvalL_2 \leq \fval_1 \leq \rvalR_2
	}
	{
		\varu_1 \samplel \uniform(0,1)
		\sim_{\epsilon(23 \eta B + 1)} 
		\varu_2 \samplel \uniform(0,1)
		: 
		\top
		\Rightarrow 
		\Phi
	}
	~~
	\inferrule[Null]
	{
	\empty
	}
	{
		s_1 = \samplel \uniform\{-1, 1\}
		\sim_{0} 
		s_2 = \samplel \uniform\{-1, 1\}
		: 
		\Phi
		\Rightarrow 
		\Phi \land \pi_1(s_1) = \pi_1(s_2)
	}
	}
	{
		\varu_1 \samplel \uniform(0,1); s_1 = \samplel \uniform\{-1, 1\}
		\sim_{\epsilon(23 \eta B + 1)} 
		\varu_2 \samplel \uniform(0,1); s_2 = \samplel \uniform\{-1, 1\}
		: 
		\top \Rightarrow \Phi \land \pi_1(s_1) = \pi_1(s_2) = \valv
	}
\\
\Pi_R
}
{
	\vdash 
	\varu_1 \samplel \uniform(0,1); s_1 = \samplel \uniform\{-1, 1\};
	\varx_1 = f(a) + \frac{1}{\epsilon} \times s_1 \times \ln (\varu_1);
	\vary_1 = \round{\varx_1}_{\Lambda};
	\varz_1 = \clamp_B (\vary_1)
	\\
	\sim_{\epsilon(23 \eta B + 1)} 
	\varu_2 \samplel \uniform(0,1); s_2 = \samplel \uniform\{-1, 1\};
	\varx_2 = f(a') + \frac{1}{\epsilon} \times s_2 \times \ln (\varu_2)
	\vary_2 = \round{\varx_2}_{\Lambda};
	\varz_2 = \clamp_B (\vary_2)
	: \\
	\valv = -B \land f(a) + 1 = f(a')  \land
	\Rightarrow  \pi_1(\varx_1) = \pi_1(\varx_2) = \valv
}
\\
\Pi_R:
\\
\inferrule*[right = Seq]
{
\inferrule*[right = ConSeq]
{	
\inferrule*[right = Assn]
	{
		\empty
	}
	{
		\varx_1 = 
		f(a) + \frac{1}{\epsilon} \times s_1 \times \ln (\varu_1)
		\sim_{0} 
		\varx_2 =
		f(a') + \frac{1}{\epsilon} \times s_2 \times \ln (\varu_2)
		:
		\Phi \land \pi_1(s_1) = \pi_1(s_2)
		\Rightarrow
		\Phi'
	}
\and
\Phi' \Rightarrow \pi_1(\varx_1) \lameq \pi_1(\varx_2) 
\\
\Phi \land \pi_1(s_1) = \pi_1(s_2)
\Rightarrow \Phi \land \pi_1(s_1) = \pi_1(s_2)
}
{
	\varx_1 = f(a) + \frac{1}{\epsilon} \times s_1 \times \ln (\varu_1)
	\sim_{0} 
	\varx_2 = f(a') + \frac{1}{\epsilon} \times s_2 \times \ln (\varu_2)
	:
	f(a) + 1 = f(a') \land \Phi \land \pi_1(s_1) = \pi_1(s_2)
	\Rightarrow \pi_1(\varx_1) \lameq \pi_1(\varx_2)
}
\and
\Delta_R
}
{
	\varx_1 = f(a) + \frac{1}{\epsilon} \times s_1 \times \ln (\varu_1);
	\vary_1 = \round{\varx_1}_{\Lambda};
	\varz_1 = \clamp(\vary_1)
	\sim_{0} 
	\varx_2 = f(a') + \frac{1}{\epsilon} \times s_2 \times \ln (\varu_2);
	\vary_2 = \round{\varx_2}_{\Lambda};
	\varz_2 = \clamp(\vary_2)
	:
	f(a) + 1 = f(a') \land \Phi \land \pi_1(s_1) = \pi_1(s_2)
	\Rightarrow \pi_1(\varx_1) = \pi_1(\varx_2)
}
\\
\Delta_R:
\\
\inferrule*[right = Seq]
{
	\inferrule*[right = round]
	{
		\empty
	}
	{
		\vary_1 = \round{\varx_1}_{\Lambda}
		\sim_{0}
		\vary_2 = \round{\varx_2}_{\Lambda}
		: \pi_1(\varx_1) \lameq \pi_1(\varx_2) 
		\Rightarrow \pi_1(\varx_1) = \pi_1(\varx_2)
	}
	\and
	\inferrule*[right = null]
	{
		\empty
	}
	{
		\varz_1 = \clamp(\vary_1)
		\sim_{0}
		\varz_2 = \clamp(\vary_2)
		: \pi_1(\varx_1) = \pi_1(\varx_2) 
		\Rightarrow \pi_1(\varx_1) = \pi_1(\varx_2)
	}
}
{
	\vary_1 = \round{\varx_1}_{\Lambda};
	\varz_1 = \clamp(\vary_1)
	\sim_{0} 
	\vary_2 = \round{\varx_2}_{\Lambda};
	\varz_2 = \clamp(\vary_2)
	:
	\pi_1(\varx_1) \lameq \pi_1(\varx_2)
	\Rightarrow \pi_1(\varx_1) = \pi_1(\varx_2)
}
\end{mathpar}
}
%
	\subcaseL{
	$\boldsymbol{\round{f(a)}_{\Lambda} > 0 \land \valv \in (0, \round{f(a)}_{\Lambda}) } ~ (\star_1)$}

	%
	\subcaseL{$\boldsymbol{\round{f(a)}_{\Lambda} > 0 \land \valv = 0 } $}
	%
	\caseL{$\boldsymbol{\valv = \round{f(a)}_{\Lambda}}$}
	%
	\subcaseL{$s = 1$}
	%
	\subcaseL{$s = -1$}
	%
	\caseL{
	$\boldsymbol{\valv \in (\round{f(a)}_{\Lambda}, \round{f(a')}_{\Lambda})}$
	}
	%
	\caseL{$\boldsymbol{\valv = \round{f(a')}_{\Lambda}}$}
	%
	\subcaseL{$s = 1$}
	%
	\subcaseL{$s = -1$}
	%
	\caseL{$\boldsymbol{\valv \in  (\round{f(a')}_{\Lambda}, B)}$}
	%
	\subcaseL{
	$\boldsymbol{\round{f(a')}_{\Lambda} > 0 \lor \round{f(a')}_{\Lambda} < 0 \land \valv \in  (0, B)}$}
	%
	\subcaseL{$\boldsymbol{\round{f(a')}_{\Lambda} < 0 \land \valv \in  (\round{f(a')}_{\Lambda}, 0)}$}
	%
	\subcaseL{$\boldsymbol{\round{f(a')}_{\Lambda} < 0 \land \valv = 0}$}
	%
	\caseL{$\boldsymbol{\valv = B}$}
	%
\end{itemize}


\end{proof}

\newpage
\bibliographystyle{plain}
\bibliography{verifysnap.bib}

\end{document}



