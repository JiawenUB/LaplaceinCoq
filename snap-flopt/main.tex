\documentclass[a4paper,11pt]{article}
\usepackage[utf8]{inputenc}
%

\usepackage[utf8]{inputenc}%Packages
\usepackage[T1]{fontenc}
\usepackage{fourier} 
\usepackage[english]{babel} 
\usepackage{amsmath,amsfonts,amsthm} 
\usepackage{lscape}
\usepackage{geometry}
\usepackage{amsmath}
\usepackage{algorithm}
\usepackage{algorithmic}
\usepackage{amssymb}
\usepackage{amsfonts}
\usepackage{times}
\usepackage{bm}
\usepackage{mathtools}
\usepackage{ stmaryrd }
\usepackage{ amssymb }
\usepackage{ textcomp }
\usepackage[normalem]{ulem}
% For derivation rules
\usepackage{mathpartir}
\usepackage{color}
\usepackage{a4wide}

\usepackage{stmaryrd}
\SetSymbolFont{stmry}{bold}{U}{stmry}{m}{n}

\newcommand{\distr}{\mathsf{Distr}}
\newcommand{\uniform}{\mathsf{unif}}
\newcommand{\pdf}{\mathsf{pdf}}
\newcommand{\snap}{\mathsf{Snap}}
\newcommand{\fsnap}{\mathsf{Snap}_{\mathbb{F}}}
\newcommand{\rsnap}{\mathsf{Snap}_{\mathbb{R}}}


\newcommand{\pr}[2]{\underset{#1}{\mathsf{Pr}}[#2]}
\newcommand{\projl}{\pi_1}
\newcommand{\projr}{\pi_2}
\newcommand{\supp}{\mathsf{supp}}
\newcommand{\clamp}{\mathsf{clamp}}
\newcommand{\real}{\mathbb{R}}
\newcommand{\samplel}{\xleftarrow{\$}}
\newcommand{\psup}{\mathsf{Sup}}
\newcommand{\sign}{\mathsf{sign}}

\newcommand{\lapmech}{\mathcal{L}}
\newcommand{\laplace}{\mathsf{laplce}}
\newcommand{\round}[1]{\lfloor #1 \rceil}


%for syntax:

%for programs:
\newcommand{\prog}{p}
\newcommand{\fprog}{p_{\mathbb{F}}}
\newcommand{\rprog}{p_{\mathbb{R}}}
\newcommand{\ret}{\mathsf{return}}



%expression
\newcommand{\expr}{e}
\newcommand{\fexpr}{\expr_{\mathbb{F}}}
\newcommand{\rexpr}{\expr_{\mathbb{R}}}

\newcommand{\elet}{\kw{let}}

\newcommand{\ein}{\kw{in}}

%for smaples:
\newcommand{\bernoulli}{\kw{bernoulli}}

%values
\newcommand{\fval}{c}
\newcommand{\rval}{r}
\newcommand{\valv}{v}
\newcommand{\data}{D}

%variables
\newcommand{\varx}{x}

\newcommand{\fvarx}{x}
\newcommand{\rvarx}{X}


\newcommand{\term}{t}
\newcommand{\etrue}{\kw{true}}
\newcommand{\efalse}{\kw{false}}
% \newcommand{\eflconst}{c}
% \newcommand{\erlconst}{r}
\newcommand{\precision}{\eta}
\newcommand{\floaten}{\kw{fl}}

\newcommand{\err}{err}
\newcommand{\condition}{\Phi}
\newcommand{\edistr}{\mu}

\newcommand{\fbigstep}{\Downarrow^{\mathbb{F}}}
\newcommand{\rbigstep}{\Downarrow^{\mathbb{R}}}

\newcommand{\bigstep}{\Downarrow}
\newcommand{\trsto}{\Rightarrow}


%for environments
\newcommand{\trsenv}{\Theta}

\newcommand{\evlenv}{\Gamma}

\newcommand{\fevlenv}{\Gamma^{\mathbb{F}}}

\newcommand{\revlenv}{\Gamma^{\mathbb{R}}}



\usepackage{stackengine} 

% For Operations
%binary operations
\newcommand{\bop}{*}
\newcommand{\obop}{\stackMath\mathbin{\stackinset{c}{0ex}{c}{0ex}{\text{\footnotesize{$\bop$}}}{\bigcirc}}}

\newcommand{\oexp}{\stackMath\mathbin{\stackinset{c}{0ex}{c}{0ex}{\text{\footnotesize{$\mathsf{e}$}}}{\bigcirc}}}

\newcommand{\oln}{\stackMath\mathbin{\stackinset{c}{0ex}{c}{0ex}{\text{\footnotesize{$\mathsf{ln}$}}}{\bigcirc}}}

\newcommand{\odiv}{\stackMath\mathbin{\stackinset{c}{0ex}{c}{0ex}{\text{\footnotesize{$\div$}}}{\bigcirc}}}
\newcommand{\ubar}[1]{\text{\b{$#1$}}}

%unary operations
\newcommand{\uop}{\circ}
\newcommand{\ouop}{\stackMath\mathbin{\stackinset{c}{0ex}{c}{0ex}{\text{\footnotesize{$\uop$}}}{\bigcirc}}}





\newcommand{\diam}{{\color{red}\diamond}}
\newcommand{\dagg}{{\color{blue}\dagger}}
\let\oldstar\star
\renewcommand{\star}{\oldstar}

\newcommand{\im}[1]{\ensuremath{#1}}

\newcommand{\kw}[1]{\im{\mathtt{#1}}}


\newcommand{\set}[1]{\im{\{{#1}\}}}

\newcommand{\mmax}{\ensuremath{\mathsf{max}}}

%%%%%%%%%%%%%%%%%%%%%%%%%%%%%%%%%%%%%%%%%%%%%%%%%%%%%%%%
% Comments
\newcommand{\omitthis}[1]{}

% Misc.
\newcommand{\etal}{\textit{et al.}}
\newcommand{\bump}{\hspace{3.5pt}}

% Text fonts
\newcommand{\tbf}[1]{\textbf{#1}}
%\newcommand{\trm}[1]{\textrm{#1}}

% Math fonts
\newcommand{\mbb}[1]{\mathbb{#1}}
\newcommand{\mbf}[1]{\mathbf{#1}}
\newcommand{\mrm}[1]{\mathrm{#1}}
\newcommand{\mtt}[1]{\mathtt{#1}}
\newcommand{\mcal}[1]{\mathcal{#1}}
\newcommand{\mfrak}[1]{\mathfrak{#1}}
\newcommand{\msf}[1]{\mathsf{#1}}
\newcommand{\mscr}[1]{\mathscr{#1}}

% Text mode
\newenvironment{nop}{}{}

% Math mode
\newenvironment{sdisplaymath}{
\begin{nop}\small\begin{displaymath}}{
\end{displaymath}\end{nop}\ignorespacesafterend}
\newenvironment{fdisplaymath}{
\begin{nop}\footnotesize\begin{displaymath}}{
\end{displaymath}\end{nop}\ignorespacesafterend}
\newenvironment{smathpar}{
\begin{nop}\small\begin{mathpar}}{
\end{mathpar}\end{nop}\ignorespacesafterend}
\newenvironment{fmathpar}{
\begin{nop}\footnotesize\begin{mathpar}}{
\end{mathpar}\end{nop}\ignorespacesafterend}
\newenvironment{alignS}{
\begin{nop}\begin{align}}{
\end{align}\end{nop}\ignorespacesafterend}
\newenvironment{salignS}{
\begin{nop}\small\begin{align}}{
\end{align}\end{nop}\ignorespacesafterend}
\newenvironment{falignS}{
\begin{nop}\footnotesize\begin{align*}}{
\end{align}\end{nop}\ignorespacesafterend}

% Stack formatting
\newenvironment{stackAux}[2]{%
\setlength{\arraycolsep}{0pt}
\begin{array}[#1]{#2}}{
\end{array}}
\newenvironment{stackCC}{
\begin{stackAux}{c}{c}}{\end{stackAux}}
\newenvironment{stackCL}{
\begin{stackAux}{c}{l}}{\end{stackAux}}
\newenvironment{stackTL}{
\begin{stackAux}{t}{l}}{\end{stackAux}}
\newenvironment{stackTR}{
\begin{stackAux}{t}{r}}{\end{stackAux}}
\newenvironment{stackBC}{
\begin{stackAux}{b}{c}}{\end{stackAux}}
\newenvironment{stackBL}{
\begin{stackAux}{b}{l}}{\end{stackAux}}

%APPENDIX
\newcommand{\caseL}[1]{\item[\textbf{case}] \textbf{#1}\newline}
\newcommand{\subcaseL}[1]{\item[\textbf{subcase}] \textbf{#1}\newline}

\newcommand{\todo}[1]{{\footnotesize \color{red}\textbf{[[ #1 ]]}}}


%% \makeatletter
%% \newcommand\definitionname{Lemma}
%% \newcommand\listdefinitionname{Proofs of Lemmas and Theorems}
%% \newcommand\listofdefinitions{%
%%   \section*{\listdefinitionname}\@starttoc{def}}
%% \makeatother



\newtheoremstyle{athm}{\topsep}{\topsep}%
      {\upshape}%         Body font
      {}%         Indent amount (empty = no indent, \parindent = para indent)
      {\bfseries}% Thm head font
      {}%        Punctuation after thm head
      {.8em}%     Space after thm head (\newline = linebreak)
      {\thmname{#1}\thmnumber{ #2}\thmnote{~\,(#3)}
% \addcontentsline{Lemma}{Lemma}
%   {\protect\numberline{\thechapter.\thelemma}#1}
      % \ifstrempty{#3}%
      {\addcontentsline{def}{section}{#1~#2~#3}}%
      % {\addcontentsline{def}{subsection}{\theathm~#3}}
\newline}%         Thm head spec

 \theoremstyle{athm}


% \newtheoremstyle{break}
%   {\topsep}{\topsep}%
%   {\itshape}{}%
%   {\bfseries}{}%
%   {\newline}{}%
% \theoremstyle{break}

%There are some problems with llncs documentcalss, so commenting these out until i find a solution
\newtheorem{thm}{Theorem}

%\spnewtheorem{thm1}[theorem]{Theorem}{\bfseries}{\upshape}
%\newenvironment{Theorem}[1][]{\begin{thm1}\iffirstargument[#1]\fi\quad\\}{\end{thm1}}

 \newtheorem{lem}[thm]{Lemma}
 \newtheorem{conjec}{Conjecture}
 \newtheorem{corr}[thm]{Corollary}
 \newtheorem{defn}{Definition}
 \newtheorem{prop}[thm]{Proposition}
 \newtheorem{assm}[thm]{Assumption}

\newtheorem{Eg}[thm]{Example}
\newtheorem{hypothesis}[thm]{Hypothesis}
\newtheorem{motivation}{Motivation}

% BNF symbols
\newcommand{\bnfalt}{{\bf \,\,\mid\,\,}}
\newcommand{\bnfdef}{{\bf ::=~}}

%% Highlighting
\newcommand{\hlm}[1]{\mbox{\hl{$#1$}}}

%% Provenance modes
\newcommand{\modifrcationProvenance}{{\bf MP}}
\newcommand{\updateProvenance}{{\bf UP}}

%Lemmas
\newcommand{\lemref}[1]{Lemma \ref{#1}} %name and number
\newcommand{\thmref}[1]{Theorem \ref{#1}} %name and number

\renewcommand{\labelenumii}{\theenumii}
\renewcommand{\theenumii}{\theenumi.\arabic{enumii}.}

\usepackage{enumitem}
\setenumerate{listparindent=\parindent}

\newlist{enumih}{enumerate}{3}
\setlist[enumih]{label=\alph*),before=\raggedright, topsep=1ex, parsep=0pt,  itemsep=1pt }

\newlist{enumconc}{enumerate}{3}
\setlist[enumconc]{leftmargin=0.5cm, label*= \arabic*.  , topsep=1ex, parsep=0pt,  itemsep=3pt }

\newlist{enumsub}{enumerate}{3}
\setlist[enumsub]{ leftmargin=0.7cm, label*= \textbf{subcase} \bf \arabic*: }

\newlist{enumsubsub}{enumerate}{3}
\setlist[enumsubsub]{ leftmargin=0.5cm, label*= \textbf{subsubcase} \bf \arabic*: }

\newlist{mainitem}{itemize}{3}
\setlist[mainitem]{ leftmargin=0cm , label= {\bf Case} }


\newenvironment{subproof}[1][\proofname]{%
  \renewcommand{\qedsymbol}{$\blacksquare$}%
  \begin{proof}[#1]%
}{%
  \end{proof}%
}


\newenvironment{nstabbing}
  {\setlength{\topsep}{0pt}%
   \setlength{\partopsep}{0pt}%
   \tabbing}
  {\endtabbing} 





%%% Local Variables:
%%% mode: latex
%%% TeX-master: "main"
%%% End:

\usepackage{eucal}
\usepackage{url}
\usepackage{tikz}
\usepackage{amsfonts,amsmath}
\begin{document}

\title{Verifying Snapping Mechanism - Floating Point Implementation Version}
\author{Jiawen Liu}

\date{\today}

\maketitle
In order to verify the differential privacy property of an implementation of the snapping mechanism \cite{mironov2012significance}, we follow the logic rules designed from \cite{barthe2016proving} and the floating point error semantics from \cite{Ramananandro2016unified,Martel2006higher,Becker2018verified,Moscato2017Automatic}.

\section{Preliminary Definitions}
\begin{defn}
[Laplace mechanism \cite{dwork2006calibrating}]
Let $\epsilon > 0$. The Laplace mechanism  $\lapmech_{\epsilon}$: $\real \to \distr(\real)$ is defined by $\lapmech(t) = t + v$, where $v \in \real$ is drawn from the Laplace distribution $\laplace(\frac{1}{\epsilon})$.
\end{defn}
%
%
%
\section{Syntax - V1}
Following are the syntax of our system:
%
\[\begin{array}{llll}
\mbox{Arithmetic Expr.} & \expr & ::= & \varx 
	~|~ \rval ~|~ \fval
	%
	~|~ F(\data) ~|~ \expr \bop \expr
	%
	~|~ \uop (\expr) ~|~ \samplel \edistr \\
%
\mbox{Binary Operation} & \bop & ::= & + ~|~ - ~|~ \times ~|~ \div \\
%
\mbox{Unary Operation} & \uop & ::= & \ln ~|~ - ~|~ \round{\cdot} 
	%
	~|~ \clamp_B(\cdot) \\
%
\mbox{Value} & \valv & ::= & \rval ~|~  \fval \\
%
\mbox{Distribution} & \edistr & ::= & \laplace ~|~ \uniform ~|~ \bernoulli \\ 
%
\mbox{Error} & \err & ::= & (\expr, \expr) \\
%
\end{array}
\]


\newpage
\section{Semantics - V1}

The transition semantics with relative floating point computation error are shown in Figure. \ref{fig_trans_semantics_exp}. The semantics are $\expr \trsto (\err)$, which means a real expression $\expr$ can be transited in floating point computation with error bound $\err$, $\eta$ is the machine epsilon.

We assume the SAMPLE and F(D) semantics for floating point and real computation are the same. $\edistr \bigstep_{\$} \valv$ represents $\valv$ is sampled from the distribution $\edistr$.

\begin{figure}
\begin{mathpar}
\inferrule*[right = val]
{
	\rval \geq 0
}
{
	\rval
	\trsto
	\big( \frac{\rval}{(1 + \eta)}, \rval(1 + \eta) \big)
}
%
\and
%
\inferrule*[right = val-neg]
{
	\rval < 0
}
{
	\rval
	\trsto
	\big( \rval(1 + \eta), \frac{\rval}{(1 + \eta)} \big)
}
%
\and
%
\inferrule*[right = val-eq]
{
	\rval = \floaten(\rval)
}
{
	\rval
	\trsto
	( \rval, \rval )
}
%
\and
%
\inferrule*[right = f(d)]
{
	\empty
}
{
	f(D)
	\trsto
	( f(D), f(D) )
}
%
\and
%
\inferrule*[right = sample]
{
	\empty
}
{
	\samplel \edistr
	\trsto
	( \samplel \edistr, \samplel \edistr )
}
%
\and
%
\inferrule*[right = bop]
{
	\expr^1 \trsto (\ubar{\expr^1}, \bar{\expr^1})
	\and
	\expr^2 \trsto (\ubar{\expr^2}, \bar{\expr^2})
	\and
	\expr^1 * \expr^2 \geq 0
}
{
    \expr^1 * \expr^2
    \trsto 
    \big(
    \frac{\ubar{\expr^1} * \ubar{\expr^2}}{(1 + \eta)}, 
    (\bar{\expr^1} * \bar{\expr^2})(1 + \eta)
    \big)
}
%
\inferrule*[right = bop-neg]
{
	\expr^1 \trsto  (\ubar{\expr^1}, \bar{\expr^1})
	\and
	\expr^2 \trsto  (\ubar{\expr^2}, \bar{\expr^2})
	\and
	\expr^1 * \expr^2 < 0
}
{
    \expr^1 * \expr^2
    \trsto 
    \big(
    (\bar{\expr^1} * \bar{\expr^2})(1 + \eta),
    \frac{\ubar{\expr^1} * \ubar{\expr^2}}{(1 + \eta)}
    \big)
}%
\and
%
\inferrule*[right = uop]
{
	\expr \trsto (\ubar{\expr}, \bar{\expr})
	\and
	\uop (\expr) \geq 0
}
{
    \uop(\expr)
    \trsto 
    \big(
    \frac{\uop(\ubar{\expr})}{(1 + \eta)}, 
    (\uop(\bar{\expr}))(1 + \eta)
    \big)
}
%
\inferrule*[right = uop-neg]
{
	\expr \trsto  (\ubar{\expr}, \bar{\expr})
	\and
	\uop (\expr) < 0
}
{
    \uop(\expr)
    \trsto 
    \big(
    (\uop(\ubar{\expr}))(1 + \eta),
    \frac{\uop(\bar{\expr})}{(1 + \eta)}
    \big)
}
\end{mathpar}
\caption{Semantics of Transition for Expressions with Relative Floating Point Error}
\label{fig_trans_semantics_exp}
\end{figure}



\begin{figure}
\begin{mathpar}
\inferrule*[right = fval]
{
	\floaten(\rval) = \fval
}
{
	\rval
	\bigstep
	\fval
}
%
\and
%
\inferrule*[right = fbop]
{
	\expr^1 \bigstep \fval^1
	\and
	\expr^2 \bigstep \fval^2
	\and
	\floaten(\fval^1 \bop \fval^2) = \fval
}
{
    \expr^1 \bop \expr^2 \bigstep \fval
}
%
\and
%
\inferrule*[right = fuop]
{
	\expr \bigstep \fval',
	\and
	\floaten(\uop (\fval')) = \fval
}
{
    \uop(\expr) \bigstep \fval
}
\end{mathpar}
\caption{Semantics of Evaluation in Floating Point Computation}
\label{fig_real_semantics_exp}
\end{figure}

\begin{figure}
\begin{mathpar}
\inferrule*[right = rval]
{
	\empty
}
{
	\rval
	\bigstep
	\rval
}
%
\and
%
\inferrule*[right = rbop]
{
	\expr^1 \bigstep \rval^1
	\and
	\expr^2 \bigstep \rval^2
	\and
	\rval^1 \bop \rval^2 = \rval
}
{
    \expr^1 \bop \expr^2 \bigstep \rval
}
%
\and
%
\inferrule*[right = ruop]
{
	\expr \bigstep \rval',
	\and
	\uop (\rval') = \rval
}
{
    \uop(\expr) \bigstep \rval
}
%
\and
%
\inferrule*[right = sample]
{
	 \fval \leftarrow \edistr^{\diamond}
}
{
	\samplel \edistr \bigstep \fval
}
%
\and
%
\inferrule*[right = f(d)]
{
	f(D) = \fval
}
{
	f(D) \bigstep \fval
}
\end{mathpar}
\caption{Semantics of Evaluation in Real Computation}
\label{fig_real_semantics_exp}
\end{figure}


\begin{thm}[Soundness Theorem]
Given $\expr$ where the transition 
$\expr \trsto (\ubar{\expr}, \bar{\expr})$ holds, 
then if $\expr$ evaluates to $c$ in floating point computation and 
$\ubar{\expr}$ and $\bar{\expr}$ evaluates to $\ubar{r}$ and $\bar{r}$ in real computation, we have: 
\[
\ubar{r} \leq c \leq \bar{r}
% ~~ \land ~~ 
% \frac{r^1}{r^2} \leq \frac{c}{r} \leq \frac{r^2}{r^1}.
\]
\end{thm}
% \begin{proof}
% Induction on $\expr$, we have following cases:

% \end{proof}


\newpage
\section{Syntax - V2 (with Programs)}
\[\begin{array}{llll}
\mbox{Programs} & \prog & ::= & 
	%
     \varx = \expr ~|~ \varx \samplel \edistr
	%
	~|~ \prog ; \prog \\

\mbox{Expr.} & \expr & ::= & \rval ~|~  \fval
	%
	~|~ \varx ~|~ f(\data) ~|~ \expr \bop \expr
	%
	~|~ \uop (\expr) \\
%
\mbox{Binary Operation} & \bop & ::= & + ~|~ - ~|~ \times ~|~ \div \\
%
\mbox{Unary Operation} & \uop & ::= & \ln ~|~ - ~|~ \round{\cdot} 
	%
	~|~ \clamp_B(\cdot) \\
%
\mbox{Value} & \valv & ::= & \rval ~|~  \fval \\
%
\mbox{Distribution} & \edistr & ::= & \laplace ~|~ \uniform ~|~ \bernoulli \\ 
%
\mbox{Error} & \err & ::= & (\expr, \expr) \\
%
\mbox{Transaction Env.} & \trsenv & ::= & \cdot ~|~ \trsenv[x \mapsto (\expr, \err)] \\
\end{array}
\]

\newpage
\section{Semantics - V2 (with Programs)}

The transition semantics with relative floating point computation error are shown in Figure. \ref{fig_trans_semantics_exp} for programs. The semantics are $\trsenv, \prog \trsto \trsenv'$, which means a real computation programs $\prog$ with environment $\trsenv$ can be transited in floating point computation with error bound for all variables in $\trsenv'$, $\eta$ is the machine epsilon.

\begin{figure}
\begin{mathpar}
\inferrule*[right = var]
{
	\trsenv(\varx) 
	= (\expr, ( \ubar{\expr}, \bar{\expr} ))
}
{
	\trsenv, \varx
	\trsto
	(\expr, ( \ubar{\expr}, \bar{\expr} ))
}
%
\and
%
\inferrule*[right = var-non]
{
	\varx \notin dom(\trsenv)
}
{
	\trsenv, \varx
	\trsto
	(\varx, (\varx, \varx ))
}
%
\and
%
\inferrule*[right = val]
{
	\rval \geq 0
}
{
	\trsenv, \rval
	\trsto
	\big( \frac{r}{(1 + \eta)}, r(1 + \eta) \big)
}
%
\and
%
\inferrule*[right = val-neg]
{
	\fval = \floaten(\rval)
	\and
	\rval < 0
}
{
	\trsenv, \rval
	\trsto
	\big( r(1 + \eta), \frac{r}{(1 + \eta)} \big)
}
%
\and
%
\inferrule*[right = val-eq]
{
	\rval = \floaten(\rval)
}
{
	\trsenv, \rval
	\trsto
	( \rval, \rval )
}
%
\and
%
\inferrule*[right = f(d)]
{
	\empty
}
{
	\trsenv, f(D)
	\trsto
	(f(D), ( f(D), f(D) ))
}
%
\and
%
\inferrule*[right = bop]
{
	\trsenv, \expr^1 \trsto (\ubar{\expr^1}, \bar{\expr^1})
	\and
	\trsenv, \expr^2 \trsto (\ubar{\expr^2}, \bar{\expr^2})
	\and
	\expr^1 * \expr^2 \geq 0
}
{
    \trsenv, \expr^1 * \expr^2
    \trsto
    \big(
    \expr^1 * \expr^2,
    (\frac{\ubar{\expr^1} * \ubar{\expr^2}}{(1 + \eta)}, 
        (\bar{\expr^1} * \bar{\expr^2})(1 + \eta))
    \big)
}
%
\inferrule*[right = bop-neg]
{
	\trsenv, \expr^1 \trsto (\ubar{\expr^1}, \bar{\expr^1})
	\and
	\trsenv, \expr^2 \trsto (\ubar{\expr^2}, \bar{\expr^2})
	\and
	\expr^1 * \expr^2 < 0
}
{
    \trsenv, \expr^1 * \expr^2
    \trsto
    \big(\expr^1 * \expr^2,
    (\bar{\expr^1} * \bar{\expr^2})(1 + \eta),
    \frac{\ubar{\expr^1} * \ubar{\expr^2}}{(1 + \eta)}
    \big)
}%
\and
%
\inferrule*[right = uop]
{
	\trsenv, \expr \trsto (\ubar{\expr}, \bar{\expr})
	\and
	\uop (\expr) \geq 0
}
{
    \trsenv, \uop(\expr)
    \trsto
    \big(\uop(\expr),
    \frac{\uop(\ubar{\expr})}{(1 + \eta)}, 
    (\uop(\bar{\expr}))(1 + \eta)
    \big)
}
%
\inferrule*[right = uop-neg]
{
	\trsenv, \expr \trsto (\ubar{\expr}, \bar{\expr})
	\and
	\uop (\expr) < 0
}
{
    \trsenv, \uop(\expr)
    \trsto 
    \big(\uop(\expr),
    (\uop(\ubar{\expr}))(1 + \eta),
    \frac{\uop(\bar{\expr})}{(1 + \eta)}
    \big)
}
\end{mathpar}
\caption{Semantics of Transition for Expressions with Relative Floating Point Error}
\label{fig_trans_semantics_exp}
\end{figure}

\begin{figure}
\begin{mathpar}
\inferrule*[right = asg]
{
	\trsenv, \expr \trsto (\expr, \err )
}
{
	\trsenv, \rvarx = \expr
	\trsto
	\trsenv[\rvarx \mapsto (\expr, \err ) x]
}
%
\inferrule*[right = consq]
{
	\trsenv, \prog_1  \trsto \trsenv_1
	\and
	\trsenv_2, \prog_2  \trsto \trsenv_2
}
{
	\trsenv, \prog_1; \prog_2
	\bigstep
	\trsenv_2
}
%
\inferrule*[right = sample]
{
	\empty
}
{
	\trsenv, \varx \samplel \edistr
	\bigstep
	\trsenv[\varx \mapsto \samplel \edistr]
}
\end{mathpar}
\caption{Semantics of Transition with Relative Floating Point Error Propagation for Programs}
\label{fig_semantics_prog}
\end{figure}


\begin{figure}
\begin{mathpar}
\inferrule*[right = fval]
{
	\floaten(\rval) = \fval
}
{
	\rval
	\bigstep
	\fval
}
%
\and
%
\inferrule*[right = fbop]
{
	\expr^1 \bigstep \fval^1
	\and
	\expr^2 \bigstep \fval^2
	\and
	\floaten(\fval^1 \bop \fval^2) = \fval
}
{
    \expr^1 \bop \expr^2 \bigstep \fval
}
%
\and
%
\inferrule*[right = fuop]
{
	\expr \bigstep \fval',
	\and
	\floaten(\uop (\fval')) = \fval
}
{
    \uop(\expr) \bigstep \fval
}
\end{mathpar}
\caption{Semantics of Evaluation in Floating Point Computation}
\label{fig_real_semantics_exp}
\end{figure}

\begin{figure}
\begin{mathpar}
\inferrule*[right = rval]
{
	\empty
}
{
	\rval
	\bigstep
	\rval
}
%
\and
%
\inferrule*[right = rbop]
{
	\expr^1 \bigstep \rval^1
	\and
	\expr^2 \bigstep \rval^2
	\and
	\rval^1 \bop \rval^2 = \rval
}
{
    \expr^1 \bop \expr^2 \bigstep \rval
}
%
\and
%
\inferrule*[right = ruop]
{
	\expr \bigstep \rval',
	\and
	\uop (\rval') = \rval
}
{
    \uop(\expr) \bigstep \rval
}
%
\and
%
\inferrule*[right = sample]
{
	 \fval \leftarrow \edistr^{\diamond}
}
{
	\varx \samplel \edistr \bigstep \fval
}
%
\and
%
\inferrule*[right = f(d)]
{
	f(D) = \fval
}
{
	f(D) \bigstep \fval
}
\end{mathpar}
\caption{Semantics of Evaluation in Real Computation}
\label{fig_real_semantics_exp}
\end{figure}


\begin{thm}[Soundness Theorem]
For any $\prog$, if the transition 
$\cdot, \prog \trsto \trsenv$ holds, 
then $\forall x \in dom(\trsenv)$ s.t. $\trsenv(x) =(\expr, (\ubar{\expr}, \bar{\expr})) $, we have
if $\expr$ evaluates to $c$ in floating point computation and 
$\ubar{\expr}$ and $\bar{\expr}$ evaluates to $\ubar{r}$ and $\bar{r}$ in real computation, then: 
\[
\ubar{r} \leq c \leq \bar{r}
% ~~ \land ~~ 
% \frac{r^1}{r^2} \leq \frac{c}{r} \leq \frac{r^2}{r^1}.
\]
\end{thm}
\begin{proof}
Induction on $\expr$, we have following cases:

\end{proof}

\newpage
\section{Snapping Mechanism}

\begin{defn}
[$\rsnap(a) : A \to \distr(\real)$]
Given privacy parameter $\epsilon$, the Snapping mechanism $\rsnap(a)$ is defined as:
\[
	U \samplel \edistr; S \samplel \{-1, 1\};
	z = \clamp_B \big(
	\round{F(a) + S \times \ln (U) \div \epsilon}_{\Lambda}
	\big)
\]
where $F$ is a primitive query function over input database $a \in A$, $\epsilon$ is the privacy budget, $B$ is the clamping bound and $\Lambda$ is the rounding argument satisfying $\lambda = 2^k$ where $2^k$ is the smallest power of 2 greater or equal to the $\frac{1}{\epsilon}$.
%

%
Let $\rsnap'(a, U, S)$ be the same as $\fsnap(a)$ given $U, S$ without rounding and clamping steps.
\end{defn}


\begin{defn}
[$\fsnap(a) : A \to \distr(\real)$]
Given privacy parameter $\epsilon$, the floating point implemented
Snapping mechanism $\fsnap(a)$ is defined as (where all parameters are defined the same as above):
\[
	u_{\mathbb{F}} \xleftarrow{\$} \edistr;
	s_{\mathbb{F}} \samplel \{-1, 1\};
	z = \clamp_B \big(
	\round{f(a) \oplus s \otimes \oln (u) \odiv \epsilon}_{\Lambda}
	\big)
\]
Let $\fsnap'(a, u, s)$ be the same as $\fsnap(a)$ without rounding and clamping steps given $u, s$.
\end{defn}



\newpage
\section{Main Theorem}

\begin{thm}[The $\snap$ mechanism is $\epsilon-$differentially private]
Consider $\snap(a)$ defined as before, if $\snap(a) = x$ given database $a$ and privacy parameter $\epsilon$, then its actual privacy loss is bounded by $\epsilon + 12 x \epsilon \eta + 2\eta$
\end{thm}

\begin{proof}

Given $\fsnap(a) = x$ and parameter $\epsilon$, we consider $a'$ be the adjacent database of $a$ satisfying $|f(a) - f(a')| \leq 1$.
Without loss of generalization, we assume $f(a) + 1 = f(a') ~ (\diamond)$.
The proof is developed by cases of the output of $\fsnap(a)$ mechanism.
%

%
Consider the $\rsnap(a)$ outputting the same result $x$, let $(L, R)$ be the range where $\forall u \in (L, R)$ and some $s$, $\rsnap'(a, u, s) = x$, we have $\Pr[\rsnap(a)] = R - L$. Given the $\rsnap$ is $\epsilon-$dp, we have:
\[
	e^{-\epsilon} \leq \frac{\Pr[\rsnap(a)]}{\Pr[\rsnap(a)]} = \frac{R - L}{R' - L'} \leq e^{\epsilon}
\]
%

%
Let $(l, r)$ be the range where $\forall u \in (l, r)$ and some $s$, $\fsnap'(a, u, s) = x$, we estimated the $|r - l|$ in terms of floating point relative error and $|R - L|$ through our semantics in order to verify the privacy loss of $\fsnap$.
	%
	\begin{itemize}
		%
		% \item[\textbf{case}] $\boldsymbol{x = -B}$
		\caseL{$\boldsymbol{x = -B}$}
		%
		Let $b$ be the largest number rounded by $\Lambda$ that is smaller than $B$.
		We know $s = 1$, $L = l = 0$ and $R = -b$, so we only need to estimate the right side range $r$ in this case. The derivation of this case given $\fsnap'(a, R, 1) = \fsnap'(a', R, 1) = x$ is shown as following:
		% \begin{figure}
		\begin{mathpar}
		\inferrule[ln]
		{
			\inferrule*[right = val-eq]
			{
				\empty
			}
			{
				R 
				\bigstep
				r,
				({R}, {R})
			}
		}
		{
			\inferrule[op]
			{
				\ln(R) 
				\bigstep
				\oln(r),
				(\ln({R})(1 + \eta),
				\frac{\ln({R})}{(1 + \eta)})
			}
			{
				\inferrule[op]
				{
					\frac{1}{\epsilon} \times \ln(R) 
					\bigstep
					\frac{1}{\epsilon} \otimes \oln(r)
					,
					((\frac{1}{\epsilon} \times \ln({R}))(1 + \eta)^2,
					%
					\frac{\frac{1}{\epsilon} \times \ln({R})}{(1 + \eta)^2})
				}
				{
					\inferrule[id]
					{
						f(a) + \frac{1}{\epsilon} \times \ln(R)
						\bigstep
						f(a) \oplus \frac{1}{\epsilon} \otimes \oln(r)
						,
						\bigg(
						\big( f(a) + 
						(\frac{1}{\epsilon} \times \ln({R}))
						(1 + \eta)^2 \big)
						{(1 + \eta)},
						%
						\frac{(
						f(a) + \frac{\frac{1}{\epsilon} \times \ln({R})}
						{(1 + \eta)^2}
						)}
						{(1 + \eta)}
						\bigg)
					}
					{
						\rsnap'(a, R, 1)
						\bigstep
						\fsnap'(a, r, 1)
						,
						\bigg(
						\big( f(a) + 
						(\frac{1}{\epsilon} \times \ln({R}))
						(1 + \eta)^2 \big)
						{(1 + \eta)},
						%
						\frac{(
						f(a) + \frac{\frac{1}{\epsilon} \times \ln({R})}
						{(1 + \eta)^2}
						)}
						{(1 + \eta)}
						\bigg)
					}
				}
			}
		}
		\end{mathpar}
		% 
		%
		In the same way, we have the derivation for $\fsnap'(a', r, 1)$:
		\begin{mathpar}
		\inferrule
		{
			\dots
		}
		{
			\rsnap'(a', R', 1)
			\bigstep
			\fsnap'(a', r, 1)
			,
			\bigg(
			\big( f(a') + 
			(\frac{1}{\epsilon} \times \ln({R'}))
			(1 + \eta)^2 \big)
			{(1 + \eta)},
			%
			\frac{(
			f(a') + \frac{\frac{1}{\epsilon} \times \ln({R'})}
			{(1 + \eta)^2}
			)}
			{(1 + \eta)}
			\bigg)
		}
		\end{mathpar}
		%
		%
		Given $\fsnap(a) = \fsnap(a') = x = -b$, we have the lower and upper bounds for $R$ and $R'$, which are $\ubar{R}, \bar{R}, \ubar{R'}$ and $\bar{R'}$:
		%
		%
		\[
		\begin{array}{c}
		\ubar{R} = e^{\epsilon 
		\big( (x(1 + \eta) - f(a)) (1 + \eta)^2) \big)},
		%
		\bar{R} = e^{\epsilon 
		\frac{(\frac{x}{1 + \eta} - f(a))}{(1 + \eta)^2}}
		\\
		%
		\ubar{R'} = e^{\epsilon 
		\big((x(1 + \eta) - f(a'))(1 + \eta)^2 \big)},
		%
		\bar{R'} = e^{\epsilon 
		(\frac{(\frac{x}{1 + \eta} - f(a'))}{(1 + \eta)^2})}
		\end{array}
		\]
		%		
		The privacy loss of $\fsnap(a)$ in this case is bounded by:
		%
		\[
		\begin{array}{ll}
		\frac
		{\frac{1}{2}(\bar{R} - 0)}
		{\frac{1}{2}(\ubar{R'} - 0)}
		& = e^{\epsilon
		\bigg(
		\frac{(\frac{x}{1 + \eta} - f(a))}{(1 + \eta)^2}
		-
		\big((x(1 + \eta) - f(a'))(1 + \eta)^2 \big)
		\bigg)}\\
		& = e^{\epsilon
		\bigg(
		\frac{x}{(1 + \eta)^3} - \frac{f(a)}{(1 + \eta)^2}
		-
		x(1 + \eta)^3 + f(a')(1 + \eta)^2 
		\bigg)} ~ (\star)
		\end{array}
		\]
		%
		Since $ (1 + \eta)^3 > 1 + 3\eta$,  $\frac{1}{(1 + \eta)^3} < \frac{1}{1 + 3\eta} $, $(1 + \eta)^2 < 1 + 2.1\eta$ and $\frac{1}{(1 + \eta)^2} > 1 - 2 \eta$, we have:
		%
		\[
		\begin{array}{ll}
		(\star) & < e^{\epsilon \big( 
		-\frac{9\eta + 6}{1 + 3\eta} x
		+ 4.1 \eta f(a)
		+ (1 + 2.1\eta) 
		\big)}\\
		%
		& < e^{\epsilon(10.1 \eta B + 1 + 2.1\eta)}
		\end{array}
		\]
		%
		%
		%
		%
		\caseL{$\boldsymbol{x \in (-B, \round{f(a)}_{\Lambda})}$}
		%
		\subcaseL{$\boldsymbol{\round{f(a)}_{\Lambda} \leq 0 \lor \bigg( \round{f(a)}_{\Lambda} > 0 \land x \in (-B, 0) \bigg) } $}
		%
		Let $y_1 = x - (\frac{\Lambda}{2})$, $y_2 = x + (\frac{\Lambda}{2})$, we know $y_1 < 0$, $y_2 < 0$, $S = s = 1$, $L = e^{\epsilon(y_1 - f(a))}$ and $R = e^{\epsilon(y_2 - f(a))}$ in this case. The derivations of estimating $l$ and $r$ are shown as following:
		%
		\begin{mathpar}
		\inferrule[ln]
		{
			\inferrule*[right = val-eq]
			{
				\empty
			}
			{
				R 
				\bigstep
				r,
				({R}, {R})
			}
		}
		{
			\inferrule[op]
			{
				\ln(R) 
				\bigstep
				\oln(r),
				(\ln({R})(1 + \eta),
				\frac{\ln({R})}{(1 + \eta)})
			}
			{
				\inferrule[op]
				{
					\frac{1}{\epsilon} \times \ln(R) 
					\bigstep
					\frac{1}{\epsilon} \otimes \oln(r)
					,
					((\frac{1}{\epsilon} \times \ln({R}))(1 + \eta)^2,
					%
					\frac{\frac{1}{\epsilon} \times \ln({R})}{(1 + \eta)^2})
				}
				{
					\inferrule[id]
					{
						f(a) + \frac{1}{\epsilon} \times \ln(R)
						\bigstep
						f(a) \oplus \frac{1}{\epsilon} \otimes \oln(r)
						,
						\bigg(
						\big( f(a) + 
						(\frac{1}{\epsilon} \times \ln({R}))
						(1 + \eta)^2 \big)
						{(1 + \eta)},
						%
						\frac{(
						f(a) + \frac{\frac{1}{\epsilon} \times \ln({R})}
						{(1 + \eta)^2}
						)}
						{(1 + \eta)}
						\bigg)
					}
					{
						\rsnap'(a, R, 1)
						\bigstep
						\fsnap'(a, r, 1)
						,
						(
						\expr^1,
						%
						\expr^2)
					}
				}
			}
		}
		\end{mathpar}
		From soundness theorem, we have  $\expr^1 \leq y_2 \leq \expr^2$, where we can get $\ubar{R} \leq r \leq \bar{R}$.\\
		%
		Taking the lower bound,  we have: 
		$\ubar{R} = e^{\epsilon 
				\big( (y_1(1 + \eta) - f(a)) (1 + \eta)^2) \big)}$.\\
		%
		Taking the upper bound, we have: 
		$\bar{R} = e^{\epsilon 
				\frac{(\frac{y_1}{1 + \eta} - f(a))}{(1 + \eta)^2}}$.
		%
		
		\begin{mathpar}
		\inferrule
		{
			\dots
		}
		{
			\inferrule
			{
				\rsnap'(a, L, 1)
				\bigstep
				\fsnap'(a, l, 1)
				,
				(
				\frac{f(a) + 
				(\frac{1}{\epsilon} \times \ln(\ubar{L}))
				(1 + \eta)^2}
				{1 + \eta},
				%
				(
				f(a) + \frac{\frac{1}{\epsilon} \times \ln(\bar{L})}
				{(1 + \eta)^2}
				)(1 + \eta)
				)
			}
			{
				\rsnap'(a, L, 1)
				\bigstep
				\fsnap'(a, l, 1)
				,
				(\err_1, \err_2)
				%
			}
		}
		\end{mathpar}
		%
		From soundness theorem, we have  $\err_1 \leq y_2 \leq \err_2$.\\
		%
		%
		Taking the lower bound , we have:
		$\ubar{L} = e^{\epsilon 
				\big( (y_2(1 + \eta) - f(a)) (1 + \eta)^2) \big)}$.\\
		%
		Taking the upper bound, we have: 
		$\bar{L} = e^{\epsilon 
				\frac{(\frac{y_2}{1 + \eta} - f(a))}{(1 + \eta)^2}}$.

		In the same way, we have the bound of $l, r$ for adjacent data set $a'$:

		$$\ubar{R'} = e^{\epsilon 
				\big( (y_1(1 + \eta) - f(a')) (1 + \eta)^2) \big)},  ~
		\bar{R'} = e^{\epsilon 
				\frac{(\frac{y_1}{1 + \eta} - f(a'))}{(1 + \eta)^2}}.$$
		%
		%
		$$ 
		\ubar{L'} = e^{\epsilon 
				\big( (y_2(1 + \eta) - f(a')) (1 + \eta)^2) \big)}, ~ 
		\bar{L'} = e^{\epsilon 
				\frac{(\frac{y_2}{1 + \eta} - f(a'))}{(1 + \eta)^2}}$$

		Then, we have the privacy loss is bounded by:
		\[
		\frac{|\bar{R} - \ubar{L}|}{|\ubar{R'} - \bar{L'}|}.		
		\]
		%
		We also have:
		\[
		\begin{array}{ll}
		\frac{\bar{R}}{R} 
		& = e^{\epsilon
		\big(\frac{y_1}{(1 + \eta)^3} - \frac{f(a)}{(1 + \eta) ^2} 
		- y_1 + f(a) \big)}
		 \leq e^{\epsilon
		\big(- \frac{3 \eta}{1 +3 \eta}y_1 + 2 \eta f(a) \big)}
		 \leq e^{\epsilon
		\big(\frac{3 \eta}{1 +3 \eta}B + 2 \eta B \big)}
		\leq e^{5 \epsilon B \eta}\\
%
		\frac{\ubar{L}}{L} 
		& = e^{\epsilon
		\big({y_2}{(1 + \eta)^3} - {f(a)}{(1 + \eta) ^2} 
		- y_2 + f(a) \big)}
		 \geq e^{\epsilon
		\big({3 \eta}y_1 - 2 \eta f(a) \big)}
		\geq e^{-5 \epsilon B \eta}\\
		\end{array}
		\]
		%
		Then, we can derive:
		%
		\[
		\begin{array}{ll}
		|\bar{R} - \ubar{L}| 
		& \leq e^{5 \epsilon B \eta}R - e^{-5 \epsilon B \eta} L \\
		& = L \big(  e^{ \Lambda\epsilon + 5 \epsilon B \eta} 
		- e^{-5 \epsilon B \eta} \big)\\
		& = L \big( e^{ \Lambda\epsilon} e^{5 \epsilon B \eta}
		-e^{5 \epsilon B \eta}
		+  e^{5 \epsilon B \eta}
		- e^{-5 \epsilon B \eta} \big)\\
		& = L \big( e^{ \Lambda\epsilon} e^{5 \epsilon B \eta}
		- e^{5 \epsilon B \eta}
		+  \frac{1}{(e^{ \Lambda\epsilon} - 1)}
		(e^{ \Lambda\epsilon} - 1)e^{5 \epsilon B \eta}
		- e^{-5 \epsilon B \eta} \big)\\
		& \leq L \big( e^{ \Lambda\epsilon} e^{5 \epsilon B \eta}
		- e^{5 \epsilon B \eta}
		+  \frac{1}{(e - 1)}
		(e^{ \Lambda\epsilon} - 1)e^{5 \epsilon B \eta}
		- e^{-5 \epsilon B \eta} \big) ~( by 1 \leq \Lambda \epsilon < 2)\\ 
		& = L  \frac{e}{(e - 1)} \big( e^{ \Lambda\epsilon} e^{5 \epsilon B \eta}
		- e^{5 \epsilon B \eta}
		- e^{-5 \epsilon B \eta} \big)\\
		& < L  \frac{e}{(e - 1)} \big( e^{ \Lambda\epsilon} e^{5 \epsilon B \eta}
		- e^{5 \epsilon B \eta} \big)\\
		& = L (e^{ \Lambda\epsilon} -  1) e^{\ln(\frac{e}{(e - 1)}) + 5 \epsilon B \eta}\\
		& < L (e^{ \Lambda\epsilon} -  1)e^{11 \epsilon B \eta}~ (by ~ ( \frac{1}{\epsilon} < B < 2^{42} \frac{1}{\epsilon} )) \\
		& = (R - L)e^{11 \epsilon B \eta}
		\end{array}
		\]
		In the same way, we can derive:
		\[
		|\ubar{R} - \bar{L}| > e^{-5 \epsilon B\eta} R - e^{5 \epsilon B\eta} L
		> (R - L)e^{- 12 \epsilon B \eta}
		\]
		Then we have:
		\[
		\frac{|\bar{R} - \ubar{L}|}{|\ubar{R'} - \bar{L'}|}
		< e^{(23 \epsilon B \eta + \epsilon)}.		
		\]
%
		\subcaseL{$\boldsymbol{\round{f(a)}_{\Lambda} > 0 \land x \in (0, \round{f(a)}_{\Lambda}) } $}
		%

		Let $y_1 = x - (\frac{\Lambda}{2})$, $y_2 = x + (\frac{\Lambda}{2})$, we know $y_1 > 0$, $y_2 > 0$,  $S = s = 1$, $L = e^{\epsilon(y_1 - f(a))}$ and $R = e^{\epsilon(y_2 - f(a))}$ in this case. The derivations of estimating $l$ and $r$ are shown as following:
		% in Figure. \ref{fig_der_snap2}
		% \begin{figure}
		\begin{mathpar}
		\inferrule
		{
			L 
			\bigstep
			l,
			(\ubar{L}, \bar{L})
		}
		{
			\inferrule
			{
				\ln(L) 
				\bigstep
				\oln(l),
				(\ln(\ubar{L})(1 + \eta),
				\frac{\ln(\bar{L})}{(1 + \eta)})
			}
			{
				\inferrule
				{
					\frac{1}{\epsilon} \times \ln(L) 
					\bigstep
					\frac{1}{\epsilon} \otimes \oln(l)
					,
					((\frac{1}{\epsilon} \times \ln(\ubar{L}))(1 + \eta)^2,
					%
					\frac{\frac{1}{\epsilon} \times \ln(\bar{L})}{(1 + \eta)^2})
				}
				{
					\inferrule
					{
						f(a) + \frac{1}{\epsilon} \times \ln(L) 
						\bigstep
						f(a) \oplus \frac{1}{\epsilon} \otimes \oln(l)
						,
						(
						\frac{f(a) + 
						(\frac{1}{\epsilon} \times \ln(\ubar{L}))
						(1 + \eta)^2}
						{1 + \eta},
						%
						(
						f(a) + \frac{\frac{1}{\epsilon} \times \ln(\bar{L})}
						{(1 + \eta)^2}
						)(1 + \eta)
						)
					}
					{
					\rsnap'(a, L, 1)
					\bigstep
					\fsnap'(a, l, 1)
					,
					(
					\err_1,
					%
					\err_2
					)
					}
				}
			}
		}
		\end{mathpar}
		%
		%
		From soundness theorem, we have  $\err_1 \leq y_1 \leq \err_2$.
		%

		Taking the lower bound (i.e. $\err_1 = y_1$), we get:
		$\ubar{L} = e^{(y_1 / (1 + \eta) - f(a))(1 + \eta)^2\epsilon}$.
		Taking the upper bound (i.e. $\err_2 = y_1$), we get:
		$\bar{L} = e^{(y_1 (1 + \eta) - f(a))\epsilon/(1 + \eta)^2}$.
		%
		%
		\begin{mathpar}
		\inferrule
		{
			\dots
		}
		{
			\inferrule
			{
				\rsnap'(a, R, 1)
				\bigstep
				\fsnap'(a, r, 1)
				,
				(
				\frac{f(a) + 
				(\frac{1}{\epsilon} \times \ln(\ubar{R}))
				(1 + \eta)^2}
				{1 + \eta},
				%
				(
				f(a) + \frac{\frac{1}{\epsilon} \times \ln(\bar{R})}
				{(1 + \eta)^2}
				)(1 + \eta)
				)
			}
			{
				\rsnap'(a, R, 1)
				\bigstep
				\fsnap'(a, r, 1)
				,
				(\err_1, \err_2)
				%
			}
		}
		\end{mathpar}
		%
		From soundness theorem, we have  $\err_1 \leq y_2 \leq \err_2$.
		%

		%
		Taking the lower bound (i.e. $\err_1 = y_2$), we have:
		$\ubar{R} = e^{(y_2 / (1 + \eta) - f(a))(1 + \eta)^2\epsilon}$.
		Taking the upper bound (i.e. $\err_2 = y_1$), we have:
		$\bar{R} = e^{(y_2 (1 + \eta) - f(a))\epsilon/(1 + \eta)^2}$.
		%

		%
		In the same way, we have the derivation for $\fsnap'(a', l, 1)$ and $\fsnap'(a', r, 1)$:
		\begin{mathpar}
		\inferrule
		{
			\dots
		}
		{
			\rsnap'(a', L', 1)
			\bigstep
			\fsnap'(a', l', 1)
			,
			(
			\frac{f(a') + 
			(\frac{1}{\epsilon} \times \ln(\ubar{L'}))
			(1 + \eta)^2}
			{1 + \eta},
			%
			(
			f(a') + \frac{\frac{1}{\epsilon} \times \ln(\bar{L'})}
			{(1 + \eta)^2}
			)(1 + \eta)
			)
		}
		\end{mathpar}
		%
		From soundness theorem, we have  $\err_1 \leq y_2 \leq \err_2$.
		%

		Taking the lower bound (i.e. $\err_1 = y_1$), we get:
		$\ubar{L} = e^{(y_1 / (1 + \eta) - f(a'))(1 + \eta)^2\epsilon}$.
		%
		Taking the upper bound (i.e. $\err_2 = y_1$), we get:
		$\bar{L} = e^{(y_1 (1 + \eta) - f(a'))\epsilon/(1 + \eta)^2}$.
		%
		%
		\begin{mathpar}
		\inferrule
		{
			\dots
		}
		{
			\rsnap'(a', R', 1)
			\bigstep
			\fsnap'(a', r', 1)
			,
			(
			\frac{f(a') + 
			(\frac{1}{\epsilon} \times \ln(\ubar{R'}))
			(1 + \eta)^2}
			{1 + \eta},
			%
			(
			f(a') + \frac{\frac{1}{\epsilon} \times \ln(\bar{R'})}
			{(1 + \eta)^2}
			)(1 + \eta)
			)
		}
		\end{mathpar}
		From soundness theorem, we have  $\err_1 \leq y_2 \leq \err_2$.
		%

		%
		Taking the lower bound (i.e. $\err_1 = y_2$), we have:
		$\ubar{R} = e^{(y_2 / (1 + \eta) - f(a'))(1 + \eta)^2\epsilon}$.
		Taking the upper bound (i.e. $\err_2 = y_1$), we have:
		$\bar{R} = e^{(y_2 (1 + \eta) - f(a'))\epsilon/(1 + \eta)^2}$.
		%

		%
		%
		%
		The privacy loss is bounded by:
		%
		\[
		\frac{|\bar{R} - \ubar{L}|}{|\ubar{R'} - \bar{L'}|}
		\]
		%

		%
		Since the following bound can be proved by using $1 - 2\eta < (1 + \eta)^2 < 1 + 2.1\eta, y_1 > -B, y_2 > -B $ and simple approximation:
		\[
		\bar{R} - \ubar{L} < (R - L) e^{(5B\eta\epsilon)}, 
		\ubar{R'} - \bar{L'} > (R'  - L') e^{-7B\eta \epsilon}
		\]

		We also have the $\rsnap(a)$ is $\epsilon$-dp:
		\[
		\frac{|R - L|}{|R' - L'|} = e^{\epsilon}
		\]
		So we can get:
		\[
		\frac{|\bar{R} - \ubar{L}|}{|\ubar{R'} - \bar{L'}|}
		< \frac{|R - L|}{|R' - L'|} e^{(12B\eta\epsilon)}
		= e^{(1 + 12B\eta)\epsilon}
		\]		
%
%
		\subcaseL{$\boldsymbol{\round{f(a)}_{\Lambda} > 0 \land x = 0 } $}

		Let $y_1 = x - (\frac{\Lambda}{2})$, $y_2 = x + (\frac{\Lambda}{2})$, we know $y_1 < 0$, $y_2 > 0$,  $S = s = 1$, $L = e^{\epsilon(y_1 - f(a))}$ and $R = e^{\epsilon(y_2 - f(a))}$ in this case. We have the derivation as:
%
		\begin{mathpar}
		\inferrule
		{
			\dots
		}
		{
			\inferrule
			{
				\rsnap'(a, L, 1)
				\bigstep
				\fsnap'(a, l, 1)
				,
				(
				\frac{f(a) + 
				(\frac{1}{\epsilon} \times \ln(\ubar{L}))
				(1 + \eta)^2}
				{1 + \eta},
				%
				(
				f(a) + \frac{\frac{1}{\epsilon} \times \ln(\bar{L})}
				{(1 + \eta)^2}
				)(1 + \eta)
				)
			}
			{
				\rsnap'(a, L, 1)
				\bigstep
				\fsnap'(a, l, 1)
				,
				(\err_1, \err_2)
				%
			}
		}
		\end{mathpar}
		%
		From soundness theorem, we have  $\err_1 \leq y_2 \leq \err_2$.\\
		%
		%
		Taking the lower bound , we have:
		$\ubar{L} = e^{\epsilon 
				\big( (y_2(1 + \eta) - f(a)) (1 + \eta)^2) \big)}$.\\
		%
		Taking the upper bound, we have: 
		$\bar{L} = e^{\epsilon 
				\frac{(\frac{y_2}{1 + \eta} - f(a))}{(1 + \eta)^2}}$.  
		%
		%
		\begin{mathpar}
		\inferrule
		{
			\dots
		}
		{
			\inferrule
			{
				\rsnap'(a, R, 1)
				\bigstep
				\fsnap'(a, r, 1)
				,
				(
				\frac{f(a) + 
				(\frac{1}{\epsilon} \times \ln(\ubar{R}))
				(1 + \eta)^2}
				{1 + \eta},
				%
				(
				f(a) + \frac{\frac{1}{\epsilon} \times \ln(\bar{R})}
				{(1 + \eta)^2}
				)(1 + \eta)
				)
			}
			{
				\rsnap'(a, R, 1)
				\bigstep
				\fsnap'(a, r, 1)
				,
				(\err_1, \err_2)
				%
			}
		}
		\end{mathpar}
		%
		From soundness theorem, we have  $\err_1 \leq y_2 \leq \err_2$.
		%
		%
		Taking the lower bound (i.e. $\err_1 = y_2$), we have:
		$\ubar{R} = e^{(y_2 / (1 + \eta) - f(a))(1 + \eta)^2\epsilon}$.
		Taking the upper bound (i.e. $\err_2 = y_1$), we have:
		$\bar{R} = e^{(y_2 (1 + \eta) - f(a))\epsilon/(1 + \eta)^2}$.
		%
		Using the bound we proved before, we have the folloing bound on $|\bar{R} - \ubar{L}|$ and $|\ubar{R} - \bar{L}|$:
		%
		\[
		\begin{array}{ll}
		\bar{R} - \ubar{L} 
		& < 
		e^{(2B\eta \epsilon)}R - e^{-5B\eta \epsilon} L < 
		(R - L) e^{6B\eta \epsilon}\\
		\ubar{R} - \bar{L}
		& > e^{(-3B\eta \epsilon)}R - e^{5B\eta \epsilon} L > (R - L) e^{-8B\eta \epsilon},
		\end{array}
		\]

		and privacy loss is bounded by:
		%
		\[
		\frac{|\bar{R} - \ubar{L}|}{|\ubar{R'} - \bar{L'}|}
		< e^{14B\eta \epsilon + \epsilon}
		\]

		%
		\caseL{$\boldsymbol{x = \round{f(a)}_{\Lambda}}$}
		%
		%
		This case can also be split into 3 subcases by: $\round{f(a)}_{\Lambda} < 0$, $\round{f(a)}_{\Lambda} = 0$ and $\round{f(a)}_{\Lambda} > 0$. 
		%
		Without loss of generalization, we consider the worst case where the error propagate in the same direction, i.e. $\round{f(a)}_{\Lambda} < 0$.\\
		%
		From this assumption, let $y_1 = x - (\frac{\Lambda}{2})$, $y_2 = x + (\frac{\Lambda}{2})$, we know $y_1 < 0$, $y_2 < 0$.
		%
		Since $f(a) + 1 = f(a')$, we also have $\round{f(a)} < \round{f(a')}$.
		So, we know $s$ can only be $1$ for input $a'$ but $s$ can be $1$ or $-1$ for input $a$.
		%
		\\
		%
		For input $a$, when $s = 1$, we have following derivations:
		%
		%
		\begin{mathpar}
		\inferrule
		{
		 R
		 \bigstep
		 r, ( R, R )
		}
		{
		 \inferrule
		 {
		  \ln(R)
		  \bigstep
		  \oln (r), 
		  (\ln(R)(1 + \eta), \frac{\ln(R)}{1+\eta})
		 }
		 {
		  \inferrule
		  {
		   \frac{1}{\epsilon}\ln(R)
		   \bigstep
		   \frac{1}{\epsilon} \otimes \oln (r), 
		   \big(
		   \frac{1}{\epsilon}\ln(R)(1 + \eta)^2, 
		   \frac{1}{\epsilon}\frac{\ln(R)}{(1+\eta)^2}
		   \big)
		  }
		  {
		   \inferrule
		   {
		    f(a) + \frac{1}{\epsilon}\ln(R)
			\bigstep
			f(a) \oplus \frac{1}{\epsilon} \otimes \oln (r), 
			\bigg(
			(f(a) + \frac{1}{\epsilon}\ln(R)(1 + \eta)^2)(1 + \eta), 
			(f(a) + \frac{1}{\epsilon}\frac{\ln(R)}{(1+\eta)^2}) / (1 + \eta)
			\bigg)
		   }
		   {
		   \rsnap'(a, 1, R) \bigstep \fsnap'(a, 1, r), 
		   (\err_1, \err_2)
		   }
		  }
		 }
		}
		\end{mathpar}
		%
		%
		From soundness theorem, we have  $\err_1 \leq y_2 \leq \err_2$. Then we can get following bounds for $r$:\\
		%
		%
		$\ubar{R_+} = e^{\epsilon 
				\big( (y_2(1 + \eta) - f(a)) (1 + \eta)^2) \big)}$, 
		%
		$\bar{R_+} = e^{\epsilon 
				\frac{(\frac{y_2}{1 + \eta} - f(a))}{(1 + \eta)^2}}$.  
		%
		\\
		%
		Since $y_2 = \round{f(a)} + \frac{\Lambda}{2}$, we have $e^{\epsilon 
				\big( (y_2 - f(a))) \big)} > 1$, so actually we know $R = r = 1$.
		%
		% 
		\\
		%
		We can also derive the bound for $l$ in the same way as:\\
		$\ubar{L_+} = e^{\epsilon 
				\big( (y_1(1 + \eta) - f(a)) (1 + \eta)^2) \big)}$, 
		%
		$\bar{L_+} = e^{\epsilon 
				\frac{(\frac{y_1}{1 + \eta} - f(a))}{(1 + \eta)^2}}$.
		%

		% 
		When $s = -1$, we can derive following bounds in the same way for $l$ and $r$:\\
		%
		$\ubar{L_-} = e^{\epsilon 
				\big( (f(a) - y_2(1 + \eta)) (1 + \eta)^2) \big)}$,
		%
		$\bar{L_-} = e^{\epsilon 
				\frac{(f(a) - \frac{y_2}{1 + \eta})}{(1 + \eta)^2}}$.\\
		%
		% 
		$\ubar{R_2} = e^{\epsilon 
				\big( (f(a) - y_1(1 + \eta)) (1 + \eta)^2) \big)}$, 
		%
		$\bar{R_2} = e^{\epsilon 
				\frac{(f(a) - \frac{y_1}{1 + \eta})}{(1 + \eta)^2}}$.\\
		%
		Since $y_1 = \round{f(a)} - \frac{\Lambda}{2}$, 
		%
		we have $e^{\epsilon \big( (f(a) - y_1)) \big)} > 1$, so actually we know $R' = r' = 1$.
		%
		\\
		%
		For input $a'$, we have only one case where $s = 1$, the following bound can be derived:
		%
		\\
		%
		$\ubar{R'} = e^{\epsilon 
				\big( (y_2(1 + \eta) - f(a')) (1 + \eta)^2) \big)}$, 
		%
		$\bar{R'} = e^{\epsilon 
				\frac{(\frac{y_2}{1 + \eta} - f(a'))}{(1 + \eta)^2}}$.  
		% 
		\\
		%
		$\ubar{L'} = e^{\epsilon 
				\big( (y_1(1 + \eta) - f(a')) (1 + \eta)^2) \big)}$, 
		%
		$\bar{L'} = e^{\epsilon 
				\frac{(\frac{y_1}{1 + \eta} - f(a'))}{(1 + \eta)^2}}$.

		% \begin{mathpar}
		% \end{mathpar}

		We have following bounds on their ratios:
		%
		\[
		\frac{\ubar{R_+}}{R_+} = e^{\epsilon 
		\big(
		(1 + \eta)^3 y_2 - (1 + \eta)^2f(a) - y_2 + f(a)
		\big)}
		> e^{-3\epsilon B \eta},
		\frac{\bar{R_+}}{R_+} = e^{\epsilon 
		\big(
		frac{y_2}{(1 + \eta)^3} - \frac{f(a)}{(1 + \eta)^2} - y_2 + f(a)
		\big)}
		< e^{3\epsilon B \eta},
		\]
		%
		The same bound for $L_+$ by substituting $y_2$ with $y_1$, and similar bound for $L', R'$.
		%
		\[
		\frac{\ubar{R'}}{R} = e^{\epsilon 
		\big(
		(1 + \eta)^2f(a) - (1 + \eta)^3 y_2 - f(a) + y_2
		\big)}
		> e^{-2\epsilon B \eta},
		%
		\frac{\bar{R'}}{R'} = e^{\epsilon 
		\big(
		\frac{f(a)}{(1 + \eta)^2} - frac{y_2}{(1 + \eta)^3} - f(a) + y_2
		\big)}
		< e^{2\epsilon B \eta},
		\]
		%
		Using the bound on their ratios, we can get following bounds on $|\bar{R_+} - \ubar{L_+}|$ and $|\ubar{R'} - \bar{L'}|$:
		%
		\[
		|\bar{R_+} - \ubar{L_+}| < e^{3\epsilon B \eta} R - e^{-3\epsilon B \eta}L < (R - L) e^{7\epsilon B \eta},
		%
		|\ubar{R'} - \bar{L'}| > e^{-2\epsilon B \eta} R - e^{2\epsilon B \eta}L > (R' - L') e^{-5\epsilon B \eta}
		\]
		%
		Then we have the following bounds on privacy loss:
		%
		\[
		\frac{2 - (\ubar{L_+} + \ubar{L_-})}{\ubar{R'} - \bar{L'}}
		< \frac{\bar{R_+} - \ubar{L_+}}{\ubar{R'} - \bar{L'}}
		< \frac{e^{7\epsilon B \eta}(R_+ - L_+)}
		{e^{-5\epsilon B \eta} (R' - L')}
		= e^{12\epsilon B \eta + \epsilon}
		\]
		%
		%
		%
		\caseL{$\boldsymbol{x \in (\round{f(a)}_{\Lambda}, \round{f(a')}_{\Lambda})}$}
		%
		Since the output set $(\round{f(a)}_{\Lambda}, \round{f(a')}_{\Lambda})$ is empty when $\Lambda \geq 1$, so we consider the situation where $\Lambda < 1$.
		%
		There are two subcases in this case : $x >0 and x < 0$. Without loss of generalization, we consider the worst case where error propagate in the same direction, i.e., $\round{f(a')}_{\Lambda}) < 0$. The bounds derived for $l, r$ and $l', r'$ under input $a$ and $a'$ are as follows:
		%
		\\
		%
		For input $a$:
		%
		\\
		%
		$\ubar{R} = e^{\epsilon 
				\big( (f(a) - y_2(1 + \eta)) (1 + \eta)^2) \big)}$, 
		%
		$\bar{R} = e^{\epsilon 
				\frac{(f(a) - \frac{y_2}{1 + \eta})}{(1 + \eta)^2}}$.  
		% 
		\\
		%
		$\ubar{L} = e^{\epsilon 
				\big( (f(a) - y_1(1 + \eta)) (1 + \eta)^2) \big)}$, 
		%
		$\bar{L} = e^{\epsilon 
				\frac{f(a) - \frac{y_1}{1 + \eta}}{(1 + \eta)^2}}$.
		%
		\\
		For input $a'$:
		%
		\\
		% 
		$\ubar{R'} = e^{\epsilon 
				\big( (y_2(1 + \eta) - f(a')) (1 + \eta)^2) \big)}$, 
		%
		$\bar{R'} = e^{\epsilon 
				\frac{ \frac{y_2}{1 + \eta} - f(a') }{(1 + \eta)^2}}$.
		%
		\\
		%
		%
		$\ubar{L'} = e^{\epsilon 
				\big( (y_1(1 + \eta) - f(a') ) (1 + \eta)^2) \big)}$,
		%
		$\bar{L'} = e^{\epsilon 
				\frac{\frac{y_1}{1 + \eta} - f(a') }{(1 + \eta)^2}}$.
		%
		\\
		%
		The bounds on their ratio are as follows:
		%
		\[
		\frac{\ubar{R}}{R} > e^{-5B\eta \epsilon}, 
		~ \frac{\bar{R}}{R} < e^{5B\eta \epsilon};
		~~~~
		\frac{\ubar{R'}}{R'} > e^{-5B\eta \epsilon}, 
		~ \frac{\bar{R'}}{R'} < e^{5B\eta \epsilon}.
		\]
		%
		And the bounds on $|\ubar{R} - \bar{L}|$ and $|\bar{R'} - \ubar{L'}|$ are as follows:
		%
		\[
		|\ubar{R} - \bar{L}| > e^{-12 B\eta \epsilon}|R - L|, 
		~ |\bar{R'} - \ubar{L'}| < e^{11 B\eta \epsilon}|R' - L'|
		\]
		%
		So we have the privacy loss is bounded by:
		%
		\[
		\frac{|\ubar{R} - \bar{L}|}{|\bar{R'} - \ubar{L'}|}
		> \frac{e^{-12 B\eta \epsilon}|R - L|}{e^{11 B\eta \epsilon}|R' - L'|}
		= e^{-23B \eta \epsilon - \epsilon}
		\]
		%
		%
		%
		\caseL{$\boldsymbol{x = \round{f(a')}_{\Lambda}}$}
		%
		%
		%
		This case is symmetric with the case where $\boldsymbol{x = \round{f(a')}_{\Lambda}}$.
		%
		It can also be split into 3 subcases by: $\round{f(a')}_{\Lambda} < 0$, $\round{f(a')}_{\Lambda} = 0$ and $\round{f(a')}_{\Lambda} > 0$. 
		%
		Without loss of generalization, we consider the worst case where the error propagate in the same direction, i.e. $\round{f(a')}_{\Lambda} < 0$.\\
		%
		From this assumption, let $y_1 = x - (\frac{\Lambda}{2})$, $y_2 = x + (\frac{\Lambda}{2})$, we know $y_1 < 0$, $y_2 < 0$.
		%
		Since $f(a) + 1 = f(a')$, we also have $\round{f(a)} < \round{f(a')} <0$.
		So, we know $s$ can only be $-1$ for input $a$ but $s$ can be $1$ or $-1$ for input $a'$.
		%
		\\
		%
		For input $a'$, when $s = 1$, we have following derivations:
		%
		%
		\begin{mathpar}
		\inferrule
		{
		 R_+'
		 \bigstep
		 r_+, ( R_+', R_+' )
		}
		{
		 \inferrule
		 {
		  \ln(R_+')
		  \bigstep
		  \oln (r_+), 
		  (\ln(R_+')(1 + \eta), \frac{\ln(R_+')}{1+\eta})
		 }
		 {
		  \inferrule
		  {
		   \frac{1}{\epsilon}\ln(R)
		   \bigstep
		   \frac{1}{\epsilon} \otimes \oln (r_+), 
		   \big(
		   \frac{1}{\epsilon}\ln(R_+')(1 + \eta)^2, 
		   \frac{1}{\epsilon}\frac{\ln(R_+')}{(1+\eta)^2}
		   \big)
		  }
		  {
		   \inferrule
		   {
		    f(a) + \frac{1}{\epsilon}\ln(R_+')
			\bigstep
			f(a) \oplus \frac{1}{\epsilon} \otimes \oln (r_+), 
			\bigg(
			(f(a) + \frac{1}{\epsilon}\ln(R_+')(1 + \eta)^2)(1 + \eta), 
			(f(a) + \frac{1}{\epsilon}\frac{\ln(R_+')}{(1+\eta)^2}) / (1 + \eta)
			\bigg)
		   }
		   {
		   \rsnap'(a, 1, R_+') \bigstep \fsnap'(a, 1, r_+), 
		   (\err_1, \err_2)
		   }
		  }
		 }
		}
		\end{mathpar}
		%
		%
		From soundness theorem, we have  $\err_1 \leq y_2 \leq \err_2$. Then we can get following bounds for $r$:\\
		%
		%
		$\ubar{R_+'} = e^{\epsilon 
				\big( (y_2(1 + \eta) - f(a')) (1 + \eta)^2) \big)}$, 
		%
		$\bar{R_+'} = e^{\epsilon 
				\frac{(\frac{y_2}{1 + \eta} - f(a'))}{(1 + \eta)^2}}$.  
		%
		\\
		%
		Since $y_2 = \round{f(a)} + \frac{\Lambda}{2}$, we have 
		$e^{\epsilon 
				\big( (y_2 - f(a))) \big)} > 1$, so actually we know $R_+' = r_+' = 1$.
		%
		% 
		\\
		%
		We can also derive the bound for $l$ in the same way as:\\
		$\ubar{L_+'} = e^{\epsilon 
				\big( (y_1(1 + \eta) - f(a')) (1 + \eta)^2) \big)}$, 
		%
		$\bar{L_+'} = e^{\epsilon 
				\frac{(\frac{y_1}{1 + \eta} - f(a'))}{(1 + \eta)^2}}$.
		%

		% 
		When $s = -1$, we can derive following bounds in the same way for $l$ and $r$:\\
		%
		$\ubar{L_-'} = e^{\epsilon 
				\big( (f(a') - y_2(1 + \eta)) (1 + \eta)^2) \big)}$,
		%
		$\bar{L_-'} = e^{\epsilon 
				\frac{(f(a') - \frac{y_2}{1 + \eta})}{(1 + \eta)^2}}$.\\
		%
		% 
		$\ubar{R_-'} = e^{\epsilon 
				\big( (f(a') - y_1(1 + \eta)) (1 + \eta)^2) \big)}$, 
		%
		$\bar{R_-'} = e^{\epsilon 
				\frac{(f(a') - \frac{y_1}{1 + \eta})}{(1 + \eta)^2}}$.\\
		%
		Since $y_1 = \round{f(a')} - \frac{\Lambda}{2}$, 
		%
		we have $e^{\epsilon \big( (f(a') - y_1)) \big)} > 1$, so actually we know $R_-' = r_-' = 1$.
		%
		\\
		%
		For input $a$, we have only one case where $s = -1$, the following bound can be derived:
		%
		\\
		%
		$\ubar{R} = e^{\epsilon 
				\big( f(a) - (y_2(1 + \eta)) (1 + \eta)^2) \big)}$, 
		%
		$\bar{R} = e^{\epsilon 
				\frac{(f(a) - \frac{y_2}{1 + \eta})}{(1 + \eta)^2}}$.  
		% 
		\\
		%
		$\ubar{L} = e^{\epsilon 
				\big( (f(a) - y_1(1 + \eta)) (1 + \eta)^2) \big)}$, 
		%
		$\bar{L} = e^{\epsilon 
				\frac{f(a) - \frac{y_1}{1 + \eta}}{(1 + \eta)^2}}$.

		% \begin{mathpar}
		% \end{mathpar}

		We have following bounds on their ratios:
		%
		\[
		\frac{\ubar{R_+'}}{R_+'} = e^{\epsilon 
		\big(
		(1 + \eta)^3 y_2 - (1 + \eta)^2f(a) - y_2 + f(a)
		\big)}
		> e^{-3\epsilon B \eta},
		\frac{\bar{R_+'}}{R_+'} = e^{\epsilon 
		\big(
		frac{y_2}{(1 + \eta)^3} - \frac{f(a)}{(1 + \eta)^2} - y_2 + f(a)
		\big)}
		< e^{3\epsilon B \eta},
		\]
		%
		The same bound for $L_+'$ by substituting $y_2$ with $y_1$, and similar bound for $L, R$.
		%
		\[
		\frac{\ubar{R}}{R} = e^{\epsilon 
		\big(
		(1 + \eta)^2f(a) - (1 + \eta)^3 y_2 - f(a) + y_2
		\big)}
		> e^{-2\epsilon B \eta},
		%
		\frac{\bar{R}}{R} = e^{\epsilon 
		\big(
		\frac{f(a)}{(1 + \eta)^2} - frac{y_2}{(1 + \eta)^3} - f(a) + y_2
		\big)}
		< e^{2\epsilon B \eta},
		\]
		%
		Using the bound on their ratios, we can get following bounds on $|\bar{R_-'} - \ubar{L_-'}|$ and $|\ubar{R} - \bar{L}|$:
		%
		\[
		|\bar{R_-'} - \ubar{L_-'}| < e^{3\epsilon B \eta} R - e^{-3\epsilon B \eta}L < (R_-' - L_-') e^{7\epsilon B \eta},
		%
		|\ubar{R} - \bar{L}| > e^{-2\epsilon B \eta} R - e^{2\epsilon B \eta}L > (R - L) e^{-5\epsilon B \eta}
		\]
		%
		Then we have the following bounds on privacy loss:
		%
		\[
		\frac{\ubar{R} - \bar{L}}{2 - (\ubar{L_+'} + \ubar{L_-'})}
		> \frac{\ubar{R} - \bar{L}}{\bar{R_-'} - \ubar{L_-'}}
		> \frac{e^{-5\epsilon B \eta} (R - L)}
		{e^{7\epsilon B \eta}(R_-' - L_-')}
		= e^{-12\epsilon B \eta - \epsilon}
		\]
		%
		%
		%
		\caseL{$\boldsymbol{x \in  (\round{f(a')}_{\Lambda}, B)}$}
		%
		This case can also be split into 3 subcases symmetric with the case where $\boldsymbol{x \in  (-B, \round{f(a)}_{\Lambda})}$:
		%
		%
		%
		\subcaseL{$\boldsymbol{\round{f(a')}_{\Lambda} > 0 \lor \round{f(a')}_{\Lambda} < 0 \land x \in  (0, B)}$}
		%
		%
		let $y_1 = x - \frac{\Lambda}{2}$, $y_2 = x + \frac{\Lambda}{2}$, we have $y_1, y_2 > 0$. The bounds derived for $l, r$ and $l', r'$ under input $a$ and $a'$ in this case are as follows:
		%
		\\
		For input $a'$:
		%
		\\
		% 
		$\ubar{R'} = e^{\epsilon 
				\big( (f(a') - \frac{y_2}{1 + \eta}) (1 + \eta)^2) \big)}$, 
		%
		$\bar{R'} = e^{\epsilon 
				\frac{(f(a') - y_2(1 + \eta))}{(1 + \eta)^2}}$.
		%
		\\
		%
		%
		$\ubar{L'} = e^{\epsilon 
				\big( (f(a') - \frac{y_1}{1 + \eta})) (1 + \eta)^2) \big)}$,
		%
		$\bar{L'} = e^{\epsilon 
				\frac{(f(a') - y_1(1 + \eta)}{(1 + \eta)^2}}$.
		%
		\\
		%
		For input $a$:
		%
		\\
		%
		$\ubar{R} = e^{\epsilon 
				\big( (f(a) - \frac{y_2}{1 + \eta}) (1 + \eta)^2) \big)}$, 
		%
		$\bar{R} = e^{\epsilon 
				\frac{(f(a) - y_2(1 + \eta))}{(1 + \eta)^2}}$.  
		% 
		\\
		%
		$\ubar{L} = e^{\epsilon 
				\big( (f(a) - \frac{y_1}{1 + \eta}) (1 + \eta)^2) \big)}$, 
		%
		$\bar{L} = e^{\epsilon 
				\frac{f(a) - y_1(1 + \eta)}{(1 + \eta)^2}}$.
		%
		%
		The bounds on their ratio are as follows:
		%
		\[
		\frac{\ubar{R}}{R} > e^{-3B\eta \epsilon}, 
		~ \frac{\bar{R}}{R} < e^{3B\eta \epsilon}
		\]
		%
		And the bounds on $|\ubar{R} - \bar{L}|$ and $|\bar{R'} - \ubar{L'}|$ are as follows:
		%
		\[
		|\ubar{R} - \bar{L}| > e^{-7 B\eta \epsilon}|R - L|, 
		~ |\bar{R'} - \ubar{L'}| < e^{7 B\eta \epsilon}|R' - L'|
		\]
		%
		So we have the privacy loss is bounded by:
		%
		\[
		\frac{|\ubar{R} - \bar{L}|}{|\bar{R'} - \ubar{L'}|}
		> \frac{e^{-7 B\eta \epsilon}|R - L|}{e^{7 B\eta \epsilon}|R' - L'|}
		= e^{-14B \eta \epsilon - \epsilon}
		\]
		%
		\subcaseL{$\boldsymbol{\round{f(a')}_{\Lambda} < 0 \land x \in  (\round{f(a')}_{\Lambda}, 0)}$}
		%		
		let $y_1 = x - \frac{\Lambda}{2}$, $y_2 = x - \frac{\Lambda}{2}$, we have $y_1, y_2 < 0$. The bounds derived for $l, r$ in this case are as follows:
		%
		\\
		For input $a'$:
		%
		\\
		% 
		$\ubar{R'} = e^{\epsilon 
				\big( (f(a') - y_2(1 + \eta)) (1 + \eta)^2) \big)}$, 
		%
		$\bar{R'} = e^{\epsilon 
				\frac{(f(a') - \frac{y_2}{1 + \eta})}{(1 + \eta)^2}}$.
		%
		\\
		%
		%
		$\ubar{L'} = e^{\epsilon 
				\big( (f(a') - y_1(1 + \eta) ) (1 + \eta)^2) \big)}$,
		%
		$\bar{L'} = e^{\epsilon 
				\frac{(f(a') - \frac{y_1}{1 + \eta}}{(1 + \eta)^2}}$.
		%
		\\
		%
		For input $a$:
		%
		\\
		%
		$\ubar{R} = e^{\epsilon 
				\big( (f(a) - y_2(1 + \eta)) (1 + \eta)^2) \big)}$, 
		%
		$\bar{R} = e^{\epsilon 
				\frac{(f(a) - \frac{y_2}{1 + \eta})}{(1 + \eta)^2}}$.  
		% 
		\\
		%
		$\ubar{L} = e^{\epsilon 
				\big( (f(a) - y_1(1 + \eta)) (1 + \eta)^2) \big)}$, 
		%
		$\bar{L} = e^{\epsilon 
				\frac{f(a) - \frac{y_1}{1 + \eta}}{(1 + \eta)^2}}$.
		%
		\\
		%
		The bounds on their ratio are as follows:
		%
		\[
		\frac{\ubar{R}}{R} > e^{-5B\eta \epsilon}, 
		~ \frac{\bar{R}}{R} < e^{5B\eta \epsilon}
		\]
		%
		And the bounds on $|\ubar{R} - \bar{L}|$ and $|\bar{R'} - \ubar{L'}|$ are as follows:
		%
		\[
		|\ubar{R} - \bar{L}| > e^{-12 B\eta \epsilon}|R - L|, 
		~ |\bar{R'} - \ubar{L'}| < e^{11 B\eta \epsilon}|R' - L'|
		\]
		%
		So we have the privacy loss is bounded by:
		%
		\[
		\frac{|\ubar{R} - \bar{L}|}{|\bar{R'} - \ubar{L'}|}
		> \frac{e^{-12 B\eta \epsilon}|R - L|}{e^{11 B\eta \epsilon}|R' - L'|}
		= e^{-23B \eta \epsilon - \epsilon}
		\]
		%
		%
		\subcaseL{$\boldsymbol{\round{f(a')}_{\Lambda} < 0 \land x = 0}$}
		%
		let $y_1 = x - \frac{\Lambda}{2}$, $y_2 = x - \frac{\Lambda}{2}$, we have $y_1 < 0$ and $y_2 > 0$. The bounds derived for $l, r$ in this case are as follows:
		%
		\\
		For input $a'$:
		%
		\\
		% 
		$\ubar{R'} = e^{\epsilon 
				\big( (f(a') - \frac{y_2}{1 + \eta}) (1 + \eta)^2) \big)}$, 
		%
		$\bar{R'} = e^{\epsilon 
				\frac{(f(a') - y_2(1 + \eta))}{(1 + \eta)^2}}$.
		%
		\\
		%
		%
		$\ubar{L'} = e^{\epsilon 
				\big( (f(a') - y_1(1 + \eta)) (1 + \eta)^2) \big)}$, 
		%
		$\bar{L'} = e^{\epsilon 
				\frac{f(a') - \frac{y_1}{1 + \eta}}{(1 + \eta)^2}}$.
		%
		\\
		%
		For input $a$:
		%
		\\
		%
		$\ubar{R} = e^{\epsilon 
				\big( (f(a) - \frac{y_2}{1 + \eta}) (1 + \eta)^2) \big)}$, 
		%
		$\bar{R} = e^{\epsilon 
				\frac{(f(a) - y_2(1 + \eta))}{(1 + \eta)^2}}$.  
		% 
		\\
		%
		$\ubar{L} = e^{\epsilon 
				\big( (f(a) - y_1(1 + \eta)) (1 + \eta)^2) \big)}$, 
		%
		$\bar{L} = e^{\epsilon 
				\frac{f(a) - \frac{y_1}{1 + \eta}}{(1 + \eta)^2}}$.
		%
		\\
		%
		The bounds on their ratio are as follows:
		%
		\[
		\frac{\ubar{R}}{R} > e^{-3B\eta \epsilon}, 
		~ \frac{\bar{R}}{R} < e^{3B\eta \epsilon}
		\frac{\ubar{L}}{L} > e^{-5B\eta \epsilon}, 
		~ \frac{\bar{L}}{L} < e^{5B\eta \epsilon}
		\]
		%
		And the bounds on $|\ubar{R} - \bar{L}|$ and $|\bar{R'} - \ubar{L'}|$ are as follows:
		%
		\[
		|\ubar{R} - \bar{L}| > e^{-8 B\eta \epsilon}|R - L|, 
		~ |\bar{R'} - \ubar{L'}| < e^{8 B\eta \epsilon}|R' - L'|
		\]
		%
		So we have the privacy loss is bounded by:
		%
		\[
		\frac{|\ubar{R} - \bar{L}|}{|\bar{R'} - \ubar{L'}|}
		> \frac{e^{-8 B\eta \epsilon}|R - L|}{e^{8 B\eta \epsilon}|R' - L'|}
		= e^{-16B \eta \epsilon - \epsilon}
		\]
		%
		%
		%
		\caseL{$\boldsymbol{x = B}$}
		%
		%
		We know $s = -1$, $L = l = 0$ and $R = b$, so we only need to estimate the right side range $r$ in this case. The bounds derived for $r, r'$ are as following:
		\[
		\begin{array}{c}
		\ubar{R} = e^{\epsilon 
		\big( (f(a) -  \frac{x}{1 + \eta}) (1 + \eta)^2) \big)},
		%
		\bar{R} = e^{\epsilon 
		\frac{(f(a) - x(1 + \eta))}{(1 + \eta)^2}}
		\\
		%
		\ubar{R'} = e^{\epsilon 
		\big( (f(a') -  \frac{x}{1 + \eta}) (1 + \eta)^2) \big)},
		%
		\bar{R'} = e^{\epsilon 
		\frac{(f(a') - x(1 + \eta))}{(1 + \eta)^2}}
		\end{array}
		\]
		%		
		The privacy loss of $\fsnap(a)$ in this case is bounded by:
		%
		\[
		\begin{array}{ll}
		\frac
		{\frac{1}{2}(\ubar{R} - 0)}
		{\frac{1}{2}(\bar{R'} - 0)}
		& = e^{\epsilon
		\bigg(
		\big( (f(a) -  \frac{x}{1 + \eta}) (1 + \eta)^2) \big)
		-
		\frac{(f(a') - x(1 + \eta))}{(1 + \eta)^2}
		\bigg)}\\
		& = e^{\epsilon
		\bigg(
		f(a)(1 + \eta)^2 - x(1 + \eta) 
		- \frac{f(a)}{(1 + \eta)^2} + \frac{x}{(1 + \eta)}   
		\bigg)} ~ (\star)
		\end{array}
		\]
		%
		Since $ 1 + 2.1\eta > (1 + \eta)^2 > 1 + 2\eta$ and $\frac{1}{(1 + \eta)^2} > 1 - 2 \eta$, we have:
		%
		\[
		\begin{array}{ll}
		(\star) & > e^{\epsilon \big(
		(1 + 2\eta) f(a) - \frac{\eta(\eta + 2)}{1 + \eta} x
		- \frac{1}{1 + 2\eta}(f(a) + 1)
		\big)}\\
		%
		& = e^{\epsilon\big(
		\frac{4\eta(\eta + 1)}{1 + 2\eta} f(a) 
		- \frac{\eta(\eta + 2)}{1 + \eta} x
		- \frac{1}{1 + 2\eta}		
		\big)}\\
		%
		& > e^{\epsilon\big( -B \eta
		\frac{4(\eta + 1)}{1 + 2\eta} + \frac{(\eta + 2)}{1 + \eta} x - 1	
		\big)}\\
%
		& > e^{\epsilon(-6 \eta B - 1)}
		\end{array}
		\]
	%	
	\end{itemize}



\end{proof}


\newpage
\bibliographystyle{plain}
\bibliography{verifysnap.bib}



\end{document}















