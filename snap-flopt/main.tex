\documentclass[a4paper,11pt]{article}
\usepackage[utf8]{inputenc}
%

\usepackage[utf8]{inputenc}%Packages
\usepackage[T1]{fontenc}
\usepackage{fourier} 
\usepackage[english]{babel} 
\usepackage{amsmath,amsfonts,amsthm} 
\usepackage{lscape}
\usepackage{geometry}
\usepackage{amsmath}
\usepackage{algorithm}
\usepackage{algorithmic}
\usepackage{amssymb}
\usepackage{amsfonts}
\usepackage{times}
\usepackage{bm}
\usepackage{mathtools}
\usepackage{ stmaryrd }
\usepackage{ amssymb }
\usepackage{ textcomp }
\usepackage[normalem]{ulem}
% For derivation rules
\usepackage{mathpartir}
\usepackage{color}
\usepackage{a4wide}

\usepackage{stmaryrd}
\SetSymbolFont{stmry}{bold}{U}{stmry}{m}{n}

\newcommand{\distr}{\mathsf{Distr}}
\newcommand{\uniform}{\mathsf{unif}}
\newcommand{\pdf}{\mathsf{pdf}}
\newcommand{\snap}{\mathsf{Snap}}
\newcommand{\fsnap}{\mathsf{Snap}_{\mathbb{F}}}
\newcommand{\rsnap}{\mathsf{Snap}_{\mathbb{R}}}


\newcommand{\pr}[2]{\underset{#1}{\mathsf{Pr}}[#2]}
\newcommand{\projl}{\pi_1}
\newcommand{\projr}{\pi_2}
\newcommand{\supp}{\mathsf{supp}}
\newcommand{\clamp}{\mathsf{clamp}}
\newcommand{\real}{\mathbb{R}}
\newcommand{\samplel}{\xleftarrow{\$}}
\newcommand{\psup}{\mathsf{Sup}}
\newcommand{\sign}{\mathsf{sign}}

\newcommand{\lapmech}{\mathcal{L}}
\newcommand{\laplace}{\mathsf{laplce}}
\newcommand{\round}[1]{\lfloor #1 \rceil}


%for syntax:

%for programs:
\newcommand{\prog}{p}
\newcommand{\fprog}{p_{\mathbb{F}}}
\newcommand{\rprog}{p_{\mathbb{R}}}
\newcommand{\ret}{\mathsf{return}}



%expression
\newcommand{\expr}{e}
\newcommand{\fexpr}{\expr_{\mathbb{F}}}
\newcommand{\rexpr}{\expr_{\mathbb{R}}}

\newcommand{\elet}{\kw{let}}

\newcommand{\ein}{\kw{in}}

%for smaples:
\newcommand{\bernoulli}{\kw{bernoulli}}

%values
\newcommand{\fval}{c}
\newcommand{\rval}{r}
\newcommand{\valv}{v}
\newcommand{\data}{D}

%variables
\newcommand{\varx}{x}

\newcommand{\fvarx}{x}
\newcommand{\rvarx}{X}


\newcommand{\term}{t}
\newcommand{\etrue}{\kw{true}}
\newcommand{\efalse}{\kw{false}}
% \newcommand{\eflconst}{c}
% \newcommand{\erlconst}{r}
\newcommand{\precision}{\eta}
\newcommand{\floaten}{\kw{fl}}

\newcommand{\err}{err}
\newcommand{\condition}{\Phi}
\newcommand{\edistr}{\mu}

\newcommand{\fbigstep}{\Downarrow^{\mathbb{F}}}
\newcommand{\rbigstep}{\Downarrow^{\mathbb{R}}}

\newcommand{\bigstep}{\Downarrow}
\newcommand{\trsto}{\Rightarrow}


%for environments
\newcommand{\trsenv}{\Theta}

\newcommand{\evlenv}{\Gamma}

\newcommand{\fevlenv}{\Gamma^{\mathbb{F}}}

\newcommand{\revlenv}{\Gamma^{\mathbb{R}}}



\usepackage{stackengine} 

% For Operations
%binary operations
\newcommand{\bop}{*}
\newcommand{\obop}{\stackMath\mathbin{\stackinset{c}{0ex}{c}{0ex}{\text{\footnotesize{$\bop$}}}{\bigcirc}}}

\newcommand{\oexp}{\stackMath\mathbin{\stackinset{c}{0ex}{c}{0ex}{\text{\footnotesize{$\mathsf{e}$}}}{\bigcirc}}}

\newcommand{\oln}{\stackMath\mathbin{\stackinset{c}{0ex}{c}{0ex}{\text{\footnotesize{$\mathsf{ln}$}}}{\bigcirc}}}

\newcommand{\odiv}{\stackMath\mathbin{\stackinset{c}{0ex}{c}{0ex}{\text{\footnotesize{$\div$}}}{\bigcirc}}}
\newcommand{\ubar}[1]{\text{\b{$#1$}}}

%unary operations
\newcommand{\uop}{\circ}
\newcommand{\ouop}{\stackMath\mathbin{\stackinset{c}{0ex}{c}{0ex}{\text{\footnotesize{$\uop$}}}{\bigcirc}}}





\newcommand{\diam}{{\color{red}\diamond}}
\newcommand{\dagg}{{\color{blue}\dagger}}
\let\oldstar\star
\renewcommand{\star}{\oldstar}

\newcommand{\im}[1]{\ensuremath{#1}}

\newcommand{\kw}[1]{\im{\mathtt{#1}}}


\newcommand{\set}[1]{\im{\{{#1}\}}}

\newcommand{\mmax}{\ensuremath{\mathsf{max}}}

%%%%%%%%%%%%%%%%%%%%%%%%%%%%%%%%%%%%%%%%%%%%%%%%%%%%%%%%
% Comments
\newcommand{\omitthis}[1]{}

% Misc.
\newcommand{\etal}{\textit{et al.}}
\newcommand{\bump}{\hspace{3.5pt}}

% Text fonts
\newcommand{\tbf}[1]{\textbf{#1}}
%\newcommand{\trm}[1]{\textrm{#1}}

% Math fonts
\newcommand{\mbb}[1]{\mathbb{#1}}
\newcommand{\mbf}[1]{\mathbf{#1}}
\newcommand{\mrm}[1]{\mathrm{#1}}
\newcommand{\mtt}[1]{\mathtt{#1}}
\newcommand{\mcal}[1]{\mathcal{#1}}
\newcommand{\mfrak}[1]{\mathfrak{#1}}
\newcommand{\msf}[1]{\mathsf{#1}}
\newcommand{\mscr}[1]{\mathscr{#1}}

% Text mode
\newenvironment{nop}{}{}

% Math mode
\newenvironment{sdisplaymath}{
\begin{nop}\small\begin{displaymath}}{
\end{displaymath}\end{nop}\ignorespacesafterend}
\newenvironment{fdisplaymath}{
\begin{nop}\footnotesize\begin{displaymath}}{
\end{displaymath}\end{nop}\ignorespacesafterend}
\newenvironment{smathpar}{
\begin{nop}\small\begin{mathpar}}{
\end{mathpar}\end{nop}\ignorespacesafterend}
\newenvironment{fmathpar}{
\begin{nop}\footnotesize\begin{mathpar}}{
\end{mathpar}\end{nop}\ignorespacesafterend}
\newenvironment{alignS}{
\begin{nop}\begin{align}}{
\end{align}\end{nop}\ignorespacesafterend}
\newenvironment{salignS}{
\begin{nop}\small\begin{align}}{
\end{align}\end{nop}\ignorespacesafterend}
\newenvironment{falignS}{
\begin{nop}\footnotesize\begin{align*}}{
\end{align}\end{nop}\ignorespacesafterend}

% Stack formatting
\newenvironment{stackAux}[2]{%
\setlength{\arraycolsep}{0pt}
\begin{array}[#1]{#2}}{
\end{array}}
\newenvironment{stackCC}{
\begin{stackAux}{c}{c}}{\end{stackAux}}
\newenvironment{stackCL}{
\begin{stackAux}{c}{l}}{\end{stackAux}}
\newenvironment{stackTL}{
\begin{stackAux}{t}{l}}{\end{stackAux}}
\newenvironment{stackTR}{
\begin{stackAux}{t}{r}}{\end{stackAux}}
\newenvironment{stackBC}{
\begin{stackAux}{b}{c}}{\end{stackAux}}
\newenvironment{stackBL}{
\begin{stackAux}{b}{l}}{\end{stackAux}}

%APPENDIX
\newcommand{\caseL}[1]{\item[\textbf{case}] \textbf{#1}\newline}
\newcommand{\subcaseL}[1]{\item[\textbf{subcase}] \textbf{#1}\newline}

\newcommand{\todo}[1]{{\footnotesize \color{red}\textbf{[[ #1 ]]}}}


%% \makeatletter
%% \newcommand\definitionname{Lemma}
%% \newcommand\listdefinitionname{Proofs of Lemmas and Theorems}
%% \newcommand\listofdefinitions{%
%%   \section*{\listdefinitionname}\@starttoc{def}}
%% \makeatother



\newtheoremstyle{athm}{\topsep}{\topsep}%
      {\upshape}%         Body font
      {}%         Indent amount (empty = no indent, \parindent = para indent)
      {\bfseries}% Thm head font
      {}%        Punctuation after thm head
      {.8em}%     Space after thm head (\newline = linebreak)
      {\thmname{#1}\thmnumber{ #2}\thmnote{~\,(#3)}
% \addcontentsline{Lemma}{Lemma}
%   {\protect\numberline{\thechapter.\thelemma}#1}
      % \ifstrempty{#3}%
      {\addcontentsline{def}{section}{#1~#2~#3}}%
      % {\addcontentsline{def}{subsection}{\theathm~#3}}
\newline}%         Thm head spec

 \theoremstyle{athm}


% \newtheoremstyle{break}
%   {\topsep}{\topsep}%
%   {\itshape}{}%
%   {\bfseries}{}%
%   {\newline}{}%
% \theoremstyle{break}

%There are some problems with llncs documentcalss, so commenting these out until i find a solution
\newtheorem{thm}{Theorem}

%\spnewtheorem{thm1}[theorem]{Theorem}{\bfseries}{\upshape}
%\newenvironment{Theorem}[1][]{\begin{thm1}\iffirstargument[#1]\fi\quad\\}{\end{thm1}}

 \newtheorem{lem}[thm]{Lemma}
 \newtheorem{conjec}{Conjecture}
 \newtheorem{corr}[thm]{Corollary}
 \newtheorem{defn}{Definition}
 \newtheorem{prop}[thm]{Proposition}
 \newtheorem{assm}[thm]{Assumption}

\newtheorem{Eg}[thm]{Example}
\newtheorem{hypothesis}[thm]{Hypothesis}
\newtheorem{motivation}{Motivation}

% BNF symbols
\newcommand{\bnfalt}{{\bf \,\,\mid\,\,}}
\newcommand{\bnfdef}{{\bf ::=~}}

%% Highlighting
\newcommand{\hlm}[1]{\mbox{\hl{$#1$}}}

%% Provenance modes
\newcommand{\modifrcationProvenance}{{\bf MP}}
\newcommand{\updateProvenance}{{\bf UP}}

%Lemmas
\newcommand{\lemref}[1]{Lemma \ref{#1}} %name and number
\newcommand{\thmref}[1]{Theorem \ref{#1}} %name and number

\renewcommand{\labelenumii}{\theenumii}
\renewcommand{\theenumii}{\theenumi.\arabic{enumii}.}

\usepackage{enumitem}
\setenumerate{listparindent=\parindent}

\newlist{enumih}{enumerate}{3}
\setlist[enumih]{label=\alph*),before=\raggedright, topsep=1ex, parsep=0pt,  itemsep=1pt }

\newlist{enumconc}{enumerate}{3}
\setlist[enumconc]{leftmargin=0.5cm, label*= \arabic*.  , topsep=1ex, parsep=0pt,  itemsep=3pt }

\newlist{enumsub}{enumerate}{3}
\setlist[enumsub]{ leftmargin=0.7cm, label*= \textbf{subcase} \bf \arabic*: }

\newlist{enumsubsub}{enumerate}{3}
\setlist[enumsubsub]{ leftmargin=0.5cm, label*= \textbf{subsubcase} \bf \arabic*: }

\newlist{mainitem}{itemize}{3}
\setlist[mainitem]{ leftmargin=0cm , label= {\bf Case} }


\newenvironment{subproof}[1][\proofname]{%
  \renewcommand{\qedsymbol}{$\blacksquare$}%
  \begin{proof}[#1]%
}{%
  \end{proof}%
}


\newenvironment{nstabbing}
  {\setlength{\topsep}{0pt}%
   \setlength{\partopsep}{0pt}%
   \tabbing}
  {\endtabbing} 





%%% Local Variables:
%%% mode: latex
%%% TeX-master: "main"
%%% End:

\usepackage{eucal}
\usepackage{url}
\usepackage{tikz}
\usepackage{amsfonts,amsmath}
\begin{document}

\title{Verifying Snapping Mechanism - Floating Point Implementation Version}
\author{Jiawen Liu}

\date{\today}

\maketitle
In order to verify an implementation of the snapping mechanism
\cite{mironov2012significance} differentially private, we will extend
the logic proposed in~\cite{barthe2016proving} and we will take
inspiration from the floating
point error semantics from
\cite{Ramananandro2016unified,Martel2006higher,Becker2018verified,Moscato2017Automatic}.

\section{Preliminary Definitions}
\begin{defn}
[Laplace mechanism \cite{dwork2006calibrating}]
Let $\epsilon > 0$. The Laplace mechanism  $\lapmech_{\epsilon}$: $\real \to \distr(\real)$ is defined by $\lapmech(t) = t + v$, where $v \in \real$ is drawn from the Laplace distribution $\laplace(\frac{1}{\epsilon})$.
\end{defn}
%
%
%

\section{Syntax - IMP}
\[\begin{array}{llll}
\mbox{Programs} & \prog & ::= & 
	%
     \varx = \expr ~|~ \varx \samplel \edistr
	%
	~|~ \prog ; \prog \\

\mbox{Expr.} & \expr & ::= & \rval ~|~  \fval
	%
	~|~ \varx  ~|~ \expr \bop \expr
	%
	~|~ \uop (\expr) \\
%
\mbox{Binary Operation} & \bop & ::= & + ~|~ - ~|~ \times ~|~ \div \\
%
\mbox{Unary Operation} & \uop & ::= & \ln ~|~ - ~|~ \round{\cdot} 
	%
	~|~ \clamp_B(\cdot) \\
%
\mbox{Value} & \valv & ::= & \rval ~|~  \fval \\
%
\mbox{Distribution} & \edistr & ::= & \uniform(0, 1) 
%
	~|~ \uniform\{-1, 1\}\\ 
%
\mbox{Error} & \err & ::= & (\rval, \rval) \\
%
\mbox{Env.} & \trsenv & ::= & \cdot ~|~ \trsenv[x \mapsto (\fval, \err)] 
%\\
\end{array}
\]

\mg{Jiawen, please write down a description of the grammar, otherwise
  it is difficult ot imagine the different components. Especially for expressions.}

We consider a simple imperative language $\prog$ with assignments, random sampling and sequencing, and simple arithmetic expressions $\expr$ with real number $\rval$, floating point number $\fval$, variable $\varx$, binary operations and unary operations. The errors $\err$ are represented by a pair of real number. The environment maps variable to a floating point number and an error.
%
%
%
\section{Semantics - IMP}
%
The semantics for expressions with relative
floating point computation error are shown in
Figure~\ref{fig_semantics_prog}.\mg{I changed the reference because it
was pointing to the rules for expressions} 
% These rules prove judgment
% of the form $\trsenv, \prog \trsto \trsenv'$, which can be read as:
% the
% programs $\prog$ with environment $\trsenv$ results in the environment
% $\trsenv'$  which contains error bound for all variables \mg{the
%   previous sentence reads funny.}. in
% $\trsenv'$. 
In all the rules, $\eta$ is the machine epsilon.

\mg{I have some questions about the rules: 
  \begin{itemize}
  \item I find confusing the form of judgments for the evaluation of
    expressions. The form is $\Theta, e\Rightarrow (e',
    (e_l,e_u))$. One thing I find confusing is why do we want to have
    e' on the right-hand side of the arrow. As far as I can tell, the
    only rule that changes the expression is the rule for variables,
    which performs the substitution you have in the environment. At
    this point, wouldn't it better to just remove e' and perform the
    substitution when needed? Perhaps it is just a matter of style but
    it seems that it could be clearer. Also, if we want to stick with
    the current formulation, I suspect that there is a series of
    typos (see below).  
\item why do you need the condition $c=fl(r)$ in the rule VAL-NEG?
I imagine that you need a condition which tell us when to apply the
rule VAL or the rule VAL-NEG or the rule VAL-EQ. At the moment r=fl(r)
and r is positive you can apply VAL and VAL-EQ, if it is negative you
can apply VAL-NEG and VAL-EQ, Is this what you want?
\item what is $f(D)$ and why the rule return it as both upper and
  lower bound?
\item the notation max,min in rules BOP and BOP-NEG is confusing. What
  does this express? I don't understand if the * is a meta-operation
  here or it is the language level operation. If it is the
  meta-operation, then I can imagine that you can compute the min and
  the max. If it is the language level operation, what does it mean
  taking the min or max of an expression? I see the same problem with
  the rules for the unary operations. 
\item what is $e$ in the first premise of the BOP rule? I suspect that
  you actually mean $e^1_1$, that is another possible expression that
  is the result of evaluating $e^1$. The second premise has the same
  problem. Also the right-hand side of the conclusion has the same
  problem. The same problem is also in rule BOP-NEG and in rules UOP
  and UOP-NEG.
\item I am concerned by the rule SAMPLE. Are you sure you want to give
  a sample based semantics? Why this is better than giving a
  distribution based semantics like the one we are using in class for
  pWhile? Also, why there is no error here?
  \end{itemize}
}
%
%
\begin{figure}
\begin{mathpar}
\inferrule*[right = var]
{
	\trsenv(\varx) 
	= (\fval, ( \ubar{\rval}, \bar{\rval} ))
}
{
	\trsenv, \varx
	\trsto
	(\fval, ( \ubar{\rval}, \bar{\rval} ))
}
% %
% \and
% %
% \inferrule*[right = var-non]
% {
% 	\varx \notin dom(\trsenv)
% }
% {
% 	\trsenv, \varx
% 	\trsto
% 	(\varx, (\varx, \varx ))
% }
%
\and
%
\inferrule*[right = val]
{
	\fval = \floaten(\rval)
	~~
	\fval \neq \rval
	\and
	\rval \geq 0
}
{
	\trsenv, \rval
	\trsto
	\big(\fval, 
	(\frac{\rval}{(1 + \eta)}, \rval(1 + \eta)) 
	\big)
}
%
\and
%
\inferrule*[right = val-neg]
{
	\fval = \floaten(\rval) 
	~~
	\fval \neq \rval
	\and
	\rval < 0
}
{
	\trsenv, \rval
	\trsto
	\big(\fval, (\rval(1 + \eta), \frac{\rval}{(1 + \eta)}) \big)
}
%
\and
%
\inferrule*[right = val-eq]
{
	\fval = \floaten(\rval)
	~~~~
	\fval = \rval
}
{
	\trsenv, \rval
	\trsto
	(\fval, (\rval, \rval) )
}
%
\and
%
\inferrule*[right = bop-pp]
{
	\trsenv, \expr_1 \trsto (\fval_1, (\ubar{\rval_1}, \bar{\rval_1}))
	~~~~
	\trsenv, \expr_2 \trsto (\fval_2, (\ubar{\rval_2}, \bar{\rval_2}))
	~~~~
	\fval_1 \geq 0
	~~~~
	\fval_2 \geq 0
	~~~~
	\fval = \floaten(\fval_1 * \fval_2)
	~~~~
	* \in \{\times, \div \}
}
{
    \trsenv, \expr_1 * \expr_2
    \trsto
    \big(
    \fval,
    (\frac{\ubar{\rval_1} * \ubar{\rval_2}}{(1 + \eta)}, 
        (\bar{\rval_1} * \bar{\rval_2})(1 + \eta))
    \big)
}
%
\and
%
\inferrule*[right = bop-nn]
{
	\trsenv, \expr_1 \trsto (\fval_1, (\ubar{\rval_1}, \bar{\rval_1}))
	~~~~
	\trsenv, \expr_2 \trsto (\fval_2, (\ubar{\rval_2}, \bar{\rval_2}))
	~~~~
	\fval_1 < 0
	~~~~
	\fval_2 < 0
	~~~~
	\fval = \floaten(\fval_1 * \fval_2)
	~~~~
	* \in \{\times, \div \}
}
{
    \trsenv, \expr_1 * \expr_2
    \trsto
    \big(
    \fval,
    (\frac{\bar{\rval_1} * \bar{\rval_2}}{(1 + \eta)}, 
        (\ubar{\rval_1} * \ubar{\rval_2})(1 + \eta))
    \big)
}
%
\and
%
\inferrule*[right = bop-pn]
{
	\trsenv, \expr_1 \trsto (\fval_1, (\ubar{\rval_1}, \bar{\rval_1}))
	~~~~
	\trsenv, \expr_2 \trsto (\fval_2, (\ubar{\rval_2}, \bar{\rval_2}))
	~~~~
	\fval_1 \geq 0
	~~~~
	\fval_2 < 0
	~~~~
	\fval = \floaten(\fval_1 * \fval_2)
	~~~~
	* \in \{\times, \div \}
}
{
    \trsenv, \expr_1 * \expr_2
    \trsto
    \big(
    \fval,
    ((\bar{\rval_1} * \ubar{\rval_2})(1 + \eta),
    \frac{\ubar{\rval_1} * \bar{\rval_2}}{(1 + \eta)})
    \big)
}
%
\and
%
\inferrule*[right = bop-np]
{
	\trsenv, \expr_1 \trsto (\fval_1, (\ubar{\rval_1}, \bar{\rval_1}))
	~~~~
	\trsenv, \expr_2 \trsto (\fval_2, (\ubar{\rval_2}, \bar{\rval_2}))
	~~~~
	\fval_1 < 0
	~~~~
	\fval_2 \geq 0
	~~~~
	\fval = \floaten(\fval_1 * \fval_2)
	~~~~
	* \in \{\times, \div \}
}
{
    \trsenv, \expr_1 * \expr_2
    \trsto
    \big(
    \fval,
    ((\ubar{\rval_1} * \bar{\rval_2})(1 + \eta),
    \frac{\bar{\rval_1} * \ubar{\rval_2}}{(1 + \eta)})
    \big)
}
%
\and
%
\inferrule*[right = bop-p]
{
	\trsenv, \expr_1 \trsto (\fval_1, (\ubar{\rval_1}, \bar{\rval_1}))
	~~~~
	\trsenv, \expr_2 \trsto (\fval_2, (\ubar{\rval_2}, \bar{\rval_2}))
	~~~~
	\fval = \floaten(\fval_1 * \fval_2)
	~~~~
	\fval \geq 0
	~~~~
	* \in \{+, - \}
}
{
    \trsenv, \expr_1 * \expr_2
    \trsto
    \big(
    \fval,
    (\frac{\ubar{\rval_1} * \ubar{\rval_2}}{(1 + \eta)}, 
        (\bar{\rval_1} * \bar{\rval_2})(1 + \eta))
    \big)
}
%
\and
%
\inferrule*[right = bop-n]
{
	\trsenv, \expr_1 \trsto (\fval_1, (\ubar{\rval_1}, \bar{\rval_1}))
	~~~~
	\trsenv, \expr_2 \trsto (\fval_2, (\ubar{\rval_2}, \bar{\rval_2}))
	~~~~
	\fval = \floaten(\fval_1 * \fval_2)
	~~~~
	\fval < 0
	~~~~
	* \in \{+, - \}
}
{
    \trsenv, \expr_1 * \expr_2
    \trsto
    \big(
    \fval,
    ((\ubar{\rval_1} * \ubar{\rval_2})(1 + \eta),
        \frac{\bar{\rval_1} * \bar{\rval_2}}{(1 + \eta)})
    \big)
}
%
\and
%
\inferrule*[right = uop-p]
{
	\trsenv, \expr \trsto (\fval', (\ubar{\rval}, \bar{\rval}))
	~~~~
	\fval = \floaten(\uop (\fval')) 
	~~~~
	\fval \geq 0
}
{
    \trsenv, \uop(\expr)
    \trsto
    \Big(\fval,
    \big(
    \frac{\uop(\ubar{\rval})}{(1 + \eta)}, 
    (\uop(\bar{\rval}))(1 + \eta)
    \big)
    \Big)
}
%
~~~~
%
\inferrule*[right = uop-n]
{
	\trsenv, \expr \trsto (\fval', (\ubar{\rval}, \bar{\rval}))
	~~~~
	\fval = \floaten(\uop (\fval')) 
	~~~~
	\fval < 0
}
{
    \trsenv, \uop(\expr)
    \trsto 
    \Big(\fval,
    \big(\uop(\ubar{\rval})(1 + \eta),
    \frac{\uop(\bar{\rval})}{(1 + \eta)}
    \big)
    \Big)
}
\end{mathpar}
\caption{Semantics of Transition for Expressions with Relative Floating Point Error}
\label{fig_imp_trans_semantics_exp}
\end{figure}
%
%
%
\begin{figure}
\begin{mathpar}
\inferrule*[right = asg]
{
	\trsenv, \expr \trsto (\fval, \err )
}
{
	\trsenv, \varx = \expr
	\trsto
	\trsenv[\varx \mapsto (\fval, \err )]
}
%
~~
%
\inferrule*[right = consq]
{
	\trsenv, \prog_1  \trsto \trsenv_1
	\and
	\trsenv_1, \prog_2  \trsto \trsenv_2
}
{
	\trsenv, \prog_1; \prog_2
	\trsto
	\trsenv_2
}
%
~~
%
\inferrule*[right = sample]
{
	 \fval \leftarrow \edistr^{\diamond}
	 % \and
	 % \rval_1 \leq \fval \leq \rval_2
}
{
	\trsenv, \varx \samplel \edistr
	\trsto
	\trsenv[\varx \mapsto (\fval, (\fval, \fval))]
}
\end{mathpar}
\caption{Semantics of Transition with Relative Floating Point Error Propagation for Programs}
\label{fig_semantics_prog}
\end{figure}


\begin{figure}
\begin{mathpar}
\inferrule*[right = rval]
{
	\floaten(\rval) = \fval
}
{
	\rval
	\fbigstep
	\fval
}
%
~~
%
\inferrule*[right = fval]
{
	\empty
}
{
	\fval
	\fbigstep
	\fval
}
%
~~
%
\inferrule*[right = fbop]
{
	\expr^1 \fbigstep \fval^1
	~~~
	\expr^2 \fbigstep \fval^2
	~~~
	\floaten(\fval^1 \bop \fval^2) = \fval
}
{
    \expr^1 \bop \expr^2 \fbigstep \fval
}
%
~~
%
\inferrule*[right = fuop]
{
	\expr \fbigstep \fval',
	~~~
	\floaten(\uop (\fval')) = \fval
}
{
    \uop(\expr) \fbigstep \fval
}
\end{mathpar}
\caption{Semantics of Evaluation in Floating Point Computation}
\label{fig_imp_real_semantics_exp}
\end{figure}

\begin{figure}
\begin{mathpar}
\inferrule*[right = rval]
{
	\empty
}
{
	\rval
	\rbigstep
	\rval
}
%
~~
%
\inferrule*[right = rval]
{
	\empty
}
{
	\fval
	\rbigstep
	\fval
}
%
~~
%
\inferrule*[right = rbop]
{
	\expr^1 \rbigstep \rval^1
	~~~
	\expr^2 \rbigstep \rval^2
	~~~
	\rval^1 \bop \rval^2 = \rval
}
{
    \expr^1 \bop \expr^2 \rbigstep \rval
}
%
~~
%
\inferrule*[right = ruop]
{
	\expr \rbigstep \rval',
	~~~
	\uop (\rval') = \rval
}
{
    \uop(\expr) \rbigstep \rval
}
\end{mathpar}
\caption{Semantics of Evaluation in Real Computation}
\label{fig_real_semantics_exp}
\end{figure}

\clearpage
\mg{I suggest to split the theorem proof by adding a lemma that states
what you want about expressions. This will be used in the case of
assignment in the theorem.}
\begin{defn}[Bounded Environment]
An environment $\trsenv$ is bounded if and only if that:
$\forall x \in dom(\trsenv)$ s.t. $\trsenv(x) =(\fval, (\ubar{\rval}, \bar{\rval}))$,
 then 
$\ubar{\rval} \leq \fval \leq \bar{\rval}$) 
\end{defn}
%
%
\begin{thm}[Expression Soundness]
\label{thm:expsound}
For any $\expr$, if there exists a transition 
$\trsenv, \expr \trsto (\fval, (\ubar{\rval}, \bar{\rval}))$ 
and $\trsenv$ is a bounded transaction environment
%
then:
%
$$\ubar{\rval} \leq \fval \leq \bar{\rval}.$$
%
\end{thm}
%
%
%
\begin{thm}[Program Soundness]
\label{thm:progsound}
For any $\prog$, if there exists a transition 
$\trsenv, \prog \trsto \trsenv'$ and $\trsenv$ is a bounded transaction environment 
then 
$\trsenv'$ is also a bounded environment.
\end{thm}
%
\begin{proof} of \textbf{Theorem \ref{thm:progsound}}.
%
\\
%
Induction on transition rule of $\prog$, by assumption, we know $\trsenv$ is a bounded environment $(\star)$.
\begin{itemize}
	\caseL{
	\[
	\inferrule*[right = consq]
	{
	\trsenv, \prog_1  \trsto \trsenv_1
	\and
	\trsenv_1, \prog_2  \trsto \trsenv_2
	}
	{
	\trsenv, \prog_1; \prog_2
	\trsto
	\trsenv_2
	}
	\]
	}
	We need to show $\trsenv_2$ is a bounded environment.\\
	%
	Since we know $\trsenv$ is a bounded environment by assumption $(\star)$, by induction hypothesis, we have:
	%
	\\
	$\trsenv_1$ and $\trsenv_2$ are all bounded environment. This case is proved.
%
	\caseL{\[
	\inferrule*[right = sample]
	{
		 \fval \leftarrow \edistr^{\diamond}
		 \and
	 	\rval_1 \leq \fval \leq \rval_2 ~ (\square)
	}
	{
		\trsenv, \varx \samplel \edistr
		\trsto
		\trsenv[\varx \mapsto (\fval, (\rval_1, \rval_2))]
	}
	\]}
	We need to show $\trsenv[\varx \mapsto (\fval, (\fval, \fval))]$ is a bounded environment.\\
	%
	Since we know $\trsenv$ is a bounded environment and $\rval_1 \leq \fval \leq \rval_2$ by assumption $(\star)$ and $(\square)$ .
	%
	\\
	%
	So we know $\trsenv[\varx \mapsto (\fval, (\fval, \fval))]$ is also a bounded environment.
%
	\caseL{\[
	\inferrule*[right = asg]
	{
		\trsenv, \expr \trsto (\fval, (\ubar{\rval}, \bar{\rval}) )
	}
	{
		\trsenv, \varx = \expr
		\trsto
		\trsenv[\varx \mapsto (\fval, (\ubar{\rval}, \bar{\rval}) )]
	}
	\]}
	We need to show: $\trsenv[\varx \mapsto (\expr, \err )]$ is a bounded environment.\\
	%
	By assumption $(\star)$ we know: $\trsenv$ is already a bounded environment. It is sufficient to show:\\
	%
	$\ubar{\rval} \leq \fval \leq \bar{\rval}$.\\
	%
	This is proved by the \textbf{Theorem \ref{thm:expsound}} (expression soundness).
\end{itemize}
\end{proof}
%
%
\begin{proof} of \textbf{Theorem \ref{thm:expsound}}.
%
\\
%
Induction on the transition rule of $\expr$, by assumption, we know $\trsenv$ is a bounded environment $(\star)$.
%
	\begin{itemize}
%
	\caseL{
	\[\inferrule*[right = var]
		{
			\trsenv(\varx) 
			= (\fval, ( \ubar{\rval}, \bar{\rval} ))
		}
		{
			\trsenv, \varx
			\trsto
			(\fval, ( \ubar{\rval}, \bar{\rval} ))
		}\]
		}
	By the assumption, we have $\forall \varx \in dom(\trsenv)$ s.t. $\trsenv(\varx) = (\fval, (\ubar{\rval}, \bar{\rval}))$, 
	$\ubar{\rval} \leq \fval \leq \bar{\rval}$.
	This case is proved.
	%
	%
	\caseL{
	\[\inferrule*[right = val]
		{
			\fval = \floaten(\rval)
			\and
			\fval \neq \rval
			\and
			\rval \geq 0
		}
		{
			\trsenv, \rval
			\trsto
			\big(\fval, 
			(\frac{\rval}{(1 + \eta)}, \rval(1 + \eta)) 
			\big)
		}
		\]
	}
	%
	By the definition of floating point rounding error and $\rval \geq 0$, we have:
	%
	$\frac{\rval}{(1 + \eta)}
	\leq \fval \leq
	\rval(1 + \eta)$
	%
	%
	\caseL{\[
	\inferrule*[right = val-neg]
	{
		\fval = \floaten(\rval) 
		\and
		\fval \neq \rval
		\and
		\rval < 0
	}
	{
		\trsenv, \rval
		\trsto
		\big(\fval, (\rval(1 + \eta), \frac{\rval}{(1 + \eta)}) \big)
	}
	\]}
	%
	By the definition of floating point rounding error and $\rval < 0$, we have:
	%
	$\rval(1 + \eta)
	\leq \fval \leq
	\frac{\rval}{(1 + \eta)}$
	%
	\caseL{\[
		\inferrule*[right = val-eq]
		{
			\fval = \floaten(\rval)
			~~~~
			\fval = \rval
		}
		{
			\trsenv, \rval
			\trsto
			(\fval, (\rval, \rval) )
		}
	\]}
	%
	It is trivial that $\rval \leq \fval = \floaten(\rval) = \rval \leq \rval$
	%
	\caseL{\[
	\inferrule*[right = bop-pp]
	{
		\trsenv, \expr_1 \trsto (\fval_1, (\ubar{\rval_1}, \bar{\rval_1})) ~ (\diamond)
		~~~
		\trsenv, \expr_2 \trsto (\fval_2, (\ubar{\rval_2}, \bar{\rval_2})) ~ (\triangle)
		~~~
		\fval_1 \geq 0
		~~~
		\fval_2 \geq 0
		~~~
		\fval = \floaten(\fval_1 * \fval_2)
		~~~
		* \in \{\times, \div \}
	}
	{
	    \trsenv, \expr_1 * \expr_2
	    \trsto
	    \big(
	    \fval,
	    (\frac{\ubar{\rval_1} * \ubar{\rval_2}}{(1 + \eta)}, 
	    (\bar{\rval_1} * \bar{\rval_2})(1 + \eta))
	    \big)
	}
	\]}
	%
	It is needed to show $\frac{\ubar{\rval_1} * \ubar{\rval_2}}{(1 + \eta)}
	\leq \fval \leq 
	(\bar{\rval_1} * \bar{\rval_2})(1 + \eta)$.\\
	%
	By induction hypothesis on $(\diamond)$ and $(\triangle)$, we have:\\
	%
	(1) $\ubar{\rval_1} \leq \fval_1 \leq \bar{\rval_1}$ holds. 
	%
	(2) $\ubar{\rval_2} \leq \fval_2 \leq \bar{\rval_2}$ holds.\\
	%
	Let $\bar{\rval'} = 
	\bar{\rval_1} * \bar{\rval_2}$ and 
	%
	$\ubar{\rval'} = \ubar{\rval_1} * \ubar{\rval_2}$.
	%
	\\
	%
	By (1), (2) and hypothesis: $\fval_1 \geq 0$, $\fval_2 \geq 0$ and $* \in \{\times, \div\}$, we have:
	%
	$\fval \geq 0
	\land
	\ubar{\rval'}
	\leq \fval_2 \bop \fval_1
	\leq \bar{\rval'}$.\\
	%
	Then by relative error of floating point rounding, we have:\\
	%
	$\frac{\ubar{\rval'}}{1 + \eta}
	\leq \floaten(\fval_2 \bop \fval_1) = \fval
	\leq (\bar{\rval'})(1 + \eta)$.
	%
	%
	This case is proved.
	%
	\caseL{\[
		\inferrule*[right = bop-nn]
		{
			\trsenv, \expr_1 \trsto (\fval_1, (\ubar{\rval_1}, \bar{\rval_1}))
			~~~~
			\trsenv, \expr_2 \trsto (\fval_2, (\ubar{\rval_2}, \bar{\rval_2}))
			~~~~
			\fval_1 < 0
			~~~~
			\fval_2 < 0
			~~~~
			\fval = \floaten(\fval_1 * \fval_2)
			~~~~
			* \in \{\times, \div \}
		}
		{
		    \trsenv, \expr_1 * \expr_2
		    \trsto
		    \big(
		    \fval,
		    (\frac{\bar{\rval_1} * \bar{\rval_2}}{(1 + \eta)}, 
		        (\ubar{\rval_1} * \ubar{\rval_2})(1 + \eta))
		    \big)
		}
	\]}
	%
	%
	It is needed to show 
	$\frac{\bar{\rval_1} * \bar{\rval_2}}{(1 + \eta)}
	\leq \fval \leq 
	(\ubar{\rval_1} * \ubar{\rval_2})(1 + \eta)$.\\
	%
	By induction hypothesis on $(\diamond)$ and $(\triangle)$, we have:\\
	%
	(1) $\ubar{\rval_1} \leq \fval_1 \leq \bar{\rval_1}$ holds. 
	%
	(2) $\ubar{\rval_2} \leq \fval_2 \leq \bar{\rval_2}$ holds.\\
	%
	Let $\bar{\rval'} = 
	\ubar{\rval_1} * \ubar{\rval_2}$ and 
	%
	$\ubar{\rval'} = \bar{\rval_1} * \bar{\rval_2}$.
	%
	\\
	%
	By (1), (2) and hypothesis: $\fval_1 < 0$, $\fval_2 < 0$ and $* \in \{\times, \div\}$, we have:
	%
	$\fval \geq 0
	\land
	\ubar{\rval'}
	\leq \fval_2 \bop \fval_1
	\leq \bar{\rval'}$.\\
	%
	Then by relative error of floating point rounding, we have:\\
	%
	$\frac{\ubar{\rval'}}{1 + \eta}
	\leq \floaten(\fval_2 \bop \fval_1) = \fval
	\leq (\bar{\rval'})(1 + \eta)$.
	%
	%
	This case is proved.
	%
	%
	\caseL{\[
	\inferrule*[right = bop-pn]
		{
			\trsenv, \expr_1 \trsto (\fval_1, (\ubar{\rval_1}, \bar{\rval_1}))
			~~~~
			\trsenv, \expr_2 \trsto (\fval_2, (\ubar{\rval_2}, \bar{\rval_2}))
			~~~~
			\fval_1 \geq 0
			~~~~
			\fval_2 < 0
			~~~~
			\fval = \floaten(\fval_1 * \fval_2)
			~~~~
			* \in \{\times, \div \}
		}
		{
		    \trsenv, \expr_1 * \expr_2
		    \trsto
		    \big(
		    \fval,
		    ((\bar{\rval_1} * \ubar{\rval_2})(1 + \eta),
		    \frac{\ubar{\rval_1} * \bar{\rval_2}}{(1 + \eta)})
		    \big)
		}
	\]}
	%
	%
	It is needed to show 
	$(\bar{\rval_1} * \ubar{\rval_2})(1 + \eta)
	\leq \fval \leq 
	\frac{\ubar{\rval_1} * \bar{\rval_2}}{(1 + \eta)}$.\\
	%
	By induction hypothesis on $(\diamond)$ and $(\triangle)$, we have:\\
	%
	(1) $\ubar{\rval_1} \leq \fval_1 \leq \bar{\rval_1}$ holds. 
	%
	(2) $\ubar{\rval_2} \leq \fval_2 \leq \bar{\rval_2}$ holds.\\
	%
	Let $\bar{\rval'} = 
	\ubar{\rval_1} * \bar{\rval_2}$
	and 
	%
	$\ubar{\rval'} = \bar{\rval_1} * \ubar{\rval_2}$.
	%
	\\
	%
	By (1), (2) and hypothesis: $\fval_1 \geq 0$, $\fval_2 < 0$ and $ * \in \{\times, \div\}$, we have:
	%
	$\fval < 0
	\land
	\ubar{\rval'}
	\leq \fval_2 \bop \fval_1
	\leq \bar{\rval'}$.\\
	%
	Then by relative error of floating point rounding, we have:\\
	%
	$\ubar{\rval'}(1 + \eta)
	\leq \floaten(\fval_2 \bop \fval_1) = \fval
	\leq \frac{\bar{\rval'}}{1 + \eta}$.
	%
	%
	This case is proved.
	%
	%
	\caseL{\[
	\inferrule*[right = bop-np]
		{
			\trsenv, \expr_1 \trsto (\fval_1, (\ubar{\rval_1}, \bar{\rval_1}))
			~~~~
			\trsenv, \expr_2 \trsto (\fval_2, (\ubar{\rval_2}, \bar{\rval_2}))
			~~~~
			\fval_1 < 0
			~~~~
			\fval_2 \geq 0
			~~~~
			\fval = \floaten(\fval_1 * \fval_2)
			~~~~
			* \in \{\times, \div \}
		}
		{
		    \trsenv, \expr_1 * \expr_2
		    \trsto
		    \big(
		    \fval,
		    ((\ubar{\rval_1} * \bar{\rval_2})(1 + \eta),
		    \frac{\bar{\rval_1} * \ubar{\rval_2}}{(1 + \eta)})
		    \big)
		}
	\]}
    % 
 	%
 	It is needed to show 
	$(\ubar{\rval_1} * \bar{\rval_2})(1 + \eta)
	\leq \fval \leq 
	\frac{\bar{\rval_1} * \ubar{\rval_2}}{(1 + \eta)}$.\\
	%
	By induction hypothesis on $(\diamond)$ and $(\triangle)$, we have:\\
	%
	(1) $\ubar{\rval_1} \leq \fval_1 \leq \bar{\rval_1}$ holds. 
	%
	(2) $\ubar{\rval_2} \leq \fval_2 \leq \bar{\rval_2}$ holds.\\
	%
	Let 
	$\bar{\rval'} = \bar{\rval_1} * \ubar{\rval_2}$
	and 
	%
	$\ubar{\rval'} = \ubar{\rval_1} * \bar{\rval_2}$.
	%
	\\
	%
	By (1), (2) and hypothesis: $\fval_1 < 0$, $\fval_2 \geq 0$ and $ * \in \{\times, \div\}$, we have:
	%
	$\fval < 0
	\land
	\ubar{\rval'}
	\leq \fval_2 \bop \fval_1
	\leq \bar{\rval'}$.\\
	%
	Then by relative error of floating point rounding, we have:\\
	%
	$\ubar{\rval'}(1 + \eta)
	\leq \floaten(\fval_2 \bop \fval_1) = \fval
	\leq \frac{\bar{\rval'}}{1 + \eta}$.
	%
	%
	This case is proved.
    %
    %
    %
    \caseL{\[
	\inferrule*[right = bop-p]
	{
		\trsenv, \expr_1 \trsto (\fval_1, (\ubar{\rval_1}, \bar{\rval_1}))
		~~~~
		\trsenv, \expr_2 \trsto (\fval_2, (\ubar{\rval_2}, \bar{\rval_2}))
		~~~~
		\fval = \floaten(\fval_1 * \fval_2)
		~~~~
		\fval \geq 0
		~~~~
		* \in \{+, - \}
	}
	{
	    \trsenv, \expr_1 * \expr_2
	    \trsto
	    \big(
	    \fval,
	    (\frac{\ubar{\rval_1} * \ubar{\rval_2}}{(1 + \eta)}, 
	    (\bar{\rval_1} * \bar{\rval_2})(1 + \eta))
	    \big)
	}
    \]}
    %
    %
 	It is needed to show 
	$\frac{\ubar{\rval_1} * \ubar{\rval_2}}{(1 + \eta)}
	\leq \fval \leq 
	(\bar{\rval_1} * \bar{\rval_2})(1 + \eta)$.\\
	%
	By induction hypothesis on $(\diamond)$ and $(\triangle)$, we have:\\
	%
	(1) $\ubar{\rval_1} \leq \fval_1 \leq \bar{\rval_1}$ holds. 
	%
	(2) $\ubar{\rval_2} \leq \fval_2 \leq \bar{\rval_2}$ holds.\\
	%
	Let 
	$\bar{\rval'} = \bar{\rval_1} * \bar{\rval_2}$
	and 
	%
	$\ubar{\rval'} = \ubar{\rval_1} * \ubar{\rval_2}$.
	%
	By (1), (2) and hypothesis: $ * \in \{+, -\}$, we have:
	%
	$\ubar{\rval'}
	\leq \fval_2 \bop \fval_1
	\leq \bar{\rval'}$.\\
	%
	Then by hypothesis: $\fval \geq 0$ and relative error of floating point rounding, we have:\\
	%
	$\frac{\ubar{\rval'}}{1 + \eta}
	\leq \floaten(\fval_2 \bop \fval_1) = \fval
	\leq \bar{\rval'}(1 + \eta)$.
	%
	This case is proved.
	%
	%
	%
    \caseL{\[
    \inferrule*[right = bop-n]
		{
			\trsenv, \expr_1 \trsto (\fval_1, (\ubar{\rval_1}, \bar{\rval_1}))
			~~~~
			\trsenv, \expr_2 \trsto (\fval_2, (\ubar{\rval_2}, \bar{\rval_2}))
			~~~~
			\fval = \floaten(\fval_1 * \fval_2)
			~~~~
			\fval < 0
			~~~~
			* \in \{+, - \}
		}
		{
		    \trsenv, \expr_1 * \expr_2
		    \trsto
		    \big(
		    \fval,
		    ((\ubar{\rval_1} * \ubar{\rval_2})(1 + \eta),
		    \frac{\bar{\rval_1} * \bar{\rval_2}}{(1 + \eta)})
		    \big)
		}
    \]}
    %
    %
 	It is needed to show 
	$(\ubar{\rval_1} * \ubar{\rval_2})(1 + \eta)
	\leq \fval \leq 
	\frac{\bar{\rval_1} * \bar{\rval_2}}{(1 + \eta)}$.\\
	%
	By induction hypothesis on $(\diamond)$ and $(\triangle)$, we have:\\
	%
	(1) $\ubar{\rval_1} \leq \fval_1 \leq \bar{\rval_1}$ holds. 
	%
	(2) $\ubar{\rval_2} \leq \fval_2 \leq \bar{\rval_2}$ holds.\\
	%
	Let 
	$\bar{\rval'} = \ubar{\rval_1} * \ubar{\rval_2}$
	and 
	%
	$\ubar{\rval'} = \bar{\rval_1} * \bar{\rval_2}$.
	%
	By (1), (2) and hypothesis: $ * \in \{+, -\}$, we have:
	%
	$\ubar{\rval'}
	\leq \fval_2 \bop \fval_1
	\leq \bar{\rval'}$.\\
	%
	Then by hypothesis: $\fval \geq 0$ and relative error of floating point rounding, we have:\\
	%
	$\ubar{\rval'}(1 + \eta)
	\leq \floaten(\fval_2 \bop \fval_1) = \fval
	\leq \frac{\bar{\rval'}}{1 + \eta}$.
	%
	This case is proved.
	%
	%
    %
	\caseL
	{
	\[
	\inferrule*[right = uop-p]
		{
			\trsenv, \expr 
			\trsto (\expr, \ubar{\rval}, \bar{\rval})
			~(\diamond)
			\and
			\uop (\expr) \geq 0
			~ (\square)
		}
		{
		    \trsenv, \uop(\expr)
		    \trsto
		    \big(\uop(\expr),
		    \frac{\uop(\ubar{\rval})}{(1 + \eta)}, 
		    (\uop(\bar{\rval}))(1 + \eta)
		    \big)
		}
	\]
	}
	%
	We need to show: 
	%
	for $\uop(\expr) \fbigstep \fval$, 
	$\frac{\uop(\ubar{\rval})}{(1 + \eta)} \rbigstep \ubar{\rval}$ 
	and $\uop(\bar{\rval})(1 + \eta) \rbigstep \bar{\rval}$,
	the $\ubar{\rval} \leq \fval \leq \bar{\rval}$ holds.\\
	%
	By induction hypothesis on $(\diamond)$, we have:\\
	%
	(1) for $\expr \fbigstep \fval'$, 
	$\ubar{\rval} \rbigstep \ubar{\rval'}$ 
	and $\bar{\rval} \rbigstep \bar{\rval'}$,
	the $\ubar{\rval'} \leq \fval \leq \bar{\rval'}$ holds.\\
	%
	%
	By (1) and monotone of unary operations, we have:
	$\uop(\ubar{\rval'})
	\leq \uop(\fval')
	\leq \uop(\bar{\rval'})$.\\
	%
	By hypothesis $(\square)$ and relative error of floating point rounding, we have:\\
	%
	$\frac{\uop(\ubar{\rval'})}{1 + \eta}
	\leq \floaten(\uop(\fval'))
	\leq \uop(\bar{\rval'})(1 + \eta)$.\\
	%
	By evaluation rule FBOP and RBOP, we have:\\
	%
	$\uop(\fval') \fbigstep \floaten(\uop(\fval'))$, 
	%
	$\frac{\uop(\ubar{\rval})}{1 + \eta} 
	\rbigstep \frac{\uop(\ubar{\rval'})}{1 + \eta}$ 
	%
	and $\uop(\bar{\rval})(1 + \eta) 
	\rbigstep \uop(\bar{\rval'})(1 + \eta)$.\\
	%
	This case is proved.
	%
	\subcaseL
	{
	\[
	\inferrule*[right = uop-neg]
		{
			\trsenv, \expr \trsto (\expr, \ubar{\rval}, \bar{\rval})
			\and
			\uop (\expr) < 0
		}
		{
		    \trsenv, \uop(\expr)
		    \trsto 
		    \big(\uop(\expr),
		    (\uop(\ubar{\rval}))(1 + \eta),
		    \frac{\uop(\bar{\rval})}{(1 + \eta)}
		    \big)
		}
	\]
	}
	%
	We need to show: 
	%
	for $\uop(\expr) \fbigstep \fval$, 
	$\uop(\ubar{\rval})(1 + \eta) \rbigstep \ubar{\rval}$
	and $\frac{\uop(\bar{\rval})}{(1 + \eta)} \rbigstep \bar{\rval}$,
	the $\ubar{\rval} \leq \fval \leq \bar{\rval}$ holds.\\
	%
	By induction hypothesis on $(\diamond)$, we have:\\
	%
	(1) for $\expr \fbigstep \fval'$, 
	$\ubar{\rval} \rbigstep \ubar{\rval'}$ 
	and $\bar{\rval} \rbigstep \bar{\rval'}$,
	the $\ubar{\rval'} \leq \fval \leq \bar{\rval'}$ holds.\\
	%
	%
	By (1) and monotone of unary operations, we have:
	$\uop(\ubar{\rval'})
	\leq \uop(\fval')
	\leq \uop(\bar{\rval'})$.\\
	%
	By hypothesis $(\square)$ and relative error of floating point rounding, we have:\\
	%
	$\uop(\ubar{\rval'})(1 + \eta)
	\leq \floaten(\uop(\fval'))
	\leq \frac{\uop(\bar{\rval'})}{1 + \eta}$.\\
	%
	By evaluation rule FBOP and RBOP, we have:\\
	%
	$\uop(\fval') \fbigstep \floaten(\uop(\fval'))$, 
	%
	$\uop(\ubar{\rval})(1 + \eta) 
	\rbigstep \uop(\ubar{\rval'})(1 + \eta)$
	%
	and $\frac{\uop(\bar{\rval})}{1 + \eta} 
	\rbigstep \frac{\uop(\bar{\rval'})}{1 + \eta}$ .\\
	%
	Let $\fval = \floaten(\uop(\fval'))$,
	$\ubar{\rval} = \uop(\ubar{\rval'})(1 + \eta)$ and $\bar{\rval} = \frac{\uop(\bar{\rval'})}{1 + \eta}$, this case is proved.
	%
\end{itemize}
%
\end{proof}

\newpage
\section{Snapping Mechanism}

\begin{defn}
[$\snap(a) : A \to \distr(\real)$]
Given privacy parameter $\epsilon$, the Snapping mechanism $\snap(a)$ is defined as:
\[
	U \samplel \uniform(0,1); S \samplel \uniform\{-1, 1\};
	x = \clamp_B \big(
	\round{f(a) + \frac{1}{\epsilon} \times S \times \ln (U)}_{\Lambda}
	\big)
\]
where $F$ is a primitive query function over input database $a \in A$, $\epsilon$ is the privacy budget, $B$ is the clamping bound and $\Lambda$ is the rounding argument satisfying $\lambda = 2^k$ where $2^k$ is the smallest power of 2 greater or equal to the $\frac{1}{\epsilon}$.
%
% \\
% %
% Given $U = u, S = s$, let $\expr_{\snap'} = f(a) + s \times \ln (u) \div \epsilon$,
% $\expr_{\snap''} = \clamp_B \big( \round{\rval_{\snap'} }_{\Lambda} \big)$
% %
% \\
% %
% Let $\snap'(a)$ be the same as $\snap(a)$ given $U = u, S = s$ without rounding and clamping steps, i.e., $\snap'(a): y = \expr_{\snap'}$,
% and $\snap''(a): z = \expr_{\snap''}$.
\end{defn}

\begin{lem}
[ZeroBoundL]
\label{lem:zeroboundl}
For any input $a$:
\begin{enumerate}
\item 
if 
$\clamp_B \big(
	\round{f(a) + \frac{1}{\epsilon} \times S \times \ln (U)}_{\Lambda}
	\big)
	\fbigstep -B$ or $B$, then $U \in (0, \fvarR)$ for some $\fvarR$.
	%
\item
if
$\clamp_B \big(
	\round{f(a) + \frac{1}{\epsilon} \times S \times \ln (U)}_{\Lambda}
	\big)
	\rbigstep -B$ or $B$, then $U \in (0, \rvarR)$ for some $\rvarR$.
\end{enumerate}
\end{lem}

\begin{lem}
[ZeroBoundR]
\label{lem:zeroboundr}
For any input $a$:
\begin{enumerate}
\item 
if 
$\clamp_B \big(
	\round{f(a) + \frac{1}{\epsilon} \times S \times \ln (U)}_{\Lambda}
	\big)
	\fbigstep -B$ or $B$ and $U \in (\fvarL, \fvarR)$ for some $\fvarL, \fvarR$.
	%
	\\
	%
	 Let $b$ be the largest number rounded by $\Lambda$ that is smaller than $B$, 
	 then:
	 %
	 \\
	 %
	$f(a) + \frac{1}{\epsilon} \times S \times \ln (U) \fbigstep -b - \frac{\Lambda}{2}$ or $b + \frac{\Lambda}{2}$.
%
\item
	%
	%
if
$\clamp_B \big(
	\round{f(a) + \frac{1}{\epsilon} \times S \times \ln (U)}_{\Lambda}
	\big)
	\rbigstep -B$ or $B$ and $U \in (\rvarL, \rvarR)$ for some $\rvarL, \rvarR$.
	%
	\\
	%
	 Let $b$ be the largest number rounded by $\Lambda$ that is smaller than $B$, 
	 then:
	 %
	 \\
	 %
	$f(a) + \frac{1}{\epsilon} \times S \times \ln (U) \rbigstep -b - \frac{\Lambda}{2}$ or $b + \frac{\Lambda}{2}$.
\end{enumerate}
\end{lem}

\begin{lem}[sign]
\label{lem:sign}
For any input $a$ and output $\varx$ where
$\clamp_B \big(
	\round{f(a) + \frac{1}{\epsilon} \times S \times \ln (U)}_{\Lambda}
	\big)
	\fbigstep \varx$ 
or $\clamp_B \big(
	\round{f(a) + \frac{1}{\epsilon} \times S \times \ln (U)}_{\Lambda}
	\big)
	\rbigstep \varx$ and $\varx < f(a)$,
%
\begin{enumerate}
	\item if $\varx < f(a)$, then $S = 1$;
	\item if $\varx = f(a)$, then $S = 1$ or $ -1$;
	\item if $\varx > f(a)$, then $S = -1$.
\end{enumerate}
\end{lem}
% \begin{defn}
% [$\snap(a) : A \to \distr(\real)$]
% Given privacy parameter $\epsilon$, the floating point implemented
% Snapping mechanism $\snap(a)$ is defined as (where all parameters are defined the same as above):
% \[
% 	u_{\mathbb{F}} \xleftarrow{\$} \edistr;
% 	s_{\mathbb{F}} \samplel \{-1, 1\};
% 	z = \clamp_B \big(
% 	\round{f(a) \oplus s \otimes \oln (u) \odiv \epsilon}_{\Lambda}
% 	\big)
% \]
% Let $\snap'(a)$ be the same as $\snap(a)$ without rounding and clamping steps given $u, s$.
% \end{defn}



\newpage
\section{Main Theorem}

\begin{thm}
[The $\snap$ mechanism is $(\epsilon + 23 B \epsilon \eta, 0)-$differentially private]
%
Consider the $\snap(a)$ defined as before. For any privacy parameter $\epsilon$ representable in floating point computation, for all possible outputs $\varx$ in floating point computation and pairs of adjacent input $a$ and $a'$ satisfying $|f(a) - f(a')| \leq 1$, the privacy loss between $\snap(a)$ and $\snap(a')$ is bounded by $e^{\epsilon + 23 B \epsilon \eta}$.
\end{thm}

\begin{proof}
%
%
Consider:
\\
1. any %
output of $\snap(a)$ in floating point computation is $\varx$,
%
\\
2. any pair of adjacent database $a'$ and $a$ where $|f(a) - f(a')| \leq 1$
%
\\
3. and any parameter $\epsilon$ is representable in floating point computation.
%
\\
Without loss of generalization, we assume $f(a) + 1 = f(a') ~ (\diamond)$.
%
\\
%
Consider the $\snap(a)$ and $\snap(a')$ output the same result $x$ under floating point and real computation.
%
\\
%
Let $(\rvarL, \rvarR)$, $(\rvarL', \rvarR')$ be the range where 
$\forall u \in (\rvarL, \rvarR)$ and some $s$, 
$\forall u' \in (\rvarL', \rvarR')$ and some $s'$, s.t.:
%
$$f(a) + \frac{1}{\epsilon} \times s \times \ln (u) \rbigstep \varx; 
~~~~~~~~~~~~~~~~~~
f(a') + \frac{1}{\epsilon} \times s' \times \ln (u') \rbigstep \varx.$$
%
We have $\Pr[\snap(a) = \varx] = \frac{1}{2} (\rvarR - \rvarL)$ 
and $\Pr[\snap(a') = \varx] = \frac{1}{2} (\rvarR' - \rvarL')$.
%
\\
%
Since the $\snap(a)$ in real computation is $(\epsilon, 0)-$DP, we can get:
\[
	e^{-\epsilon} \leq \frac{\Pr[\snap(a)]}{\Pr[\snap(a')]}
	= \frac{\rvarR - \rvarL}{\rvarR' - \rvarL'} \leq e^{\epsilon}
\]
%
\\
%
Let $(\fvarL, \fvarR)$, $(\fvarL', \fvarR')$ be the range where 
$\forall u \in (\fvarL, \fvarR)$ and some $s$, 
$\forall u' \in (\fvarL', \fvarR')$ and some $s'$, s.t.:
%
$$f(a) + \frac{1}{\epsilon} \times s \times \ln (u) \fbigstep \varx; 
~~~~~~~~~~~~~~~~~~
f(a') + \frac{1}{\epsilon} \times s' \times \ln (u') \fbigstep \varx.$$
%
To show the privacy loss of $\snap$ mechanism in floating point computation is bounded by $\epsilon + 23 B \epsilon \eta$, it’s sufficent to show:
%
\\
%
$|\fvarR - \fvarL|$ and $|\fvarR' - \fvarL'|$
is bounded by $h(|{\rvarR} - {\rvarL}|)$, $g(|{\rvarR} - {\rvarL}|)$ and $h(|{\rvarR'} - {\rvarL'}|)$, $g(|{\rvarR'} - {\rvarL'}|)$,
%
s.t.:
%
\[
	-(\epsilon + 23 B \epsilon \eta)
	\leq \ln( \frac{h(|{\rvarR} - {\rvarL}|)}{g(|{\rvarR'} - {\rvarL'}|)} )
	\leq \epsilon + 23 B \epsilon \eta.
\]

%
Induction on the outputspace of $\snap(a)$ mechanism, we have following cases:
	%
	\begin{itemize}
		%
		% \item[\textbf{case}] $\boldsymbol{x = -B}$
		\caseL{$\boldsymbol{x = -B}$}
		%
		Let $b$ be the largest number rounded by $\Lambda$ that is smaller than $B$, $b' = b + \Lambda / 2$.
		% %
		% \\
		% %
		% Let $L$ and $R$ be the range where $\forall u \in (L, R)$ and $s = 1$, s.t.
		% $\expr_{\snap''} \rbigstep \varx$.
		% %
		% \\
		% %
		% Let $l$ and $r$ be the range where $\forall u \in (l, r)$ and $s = 1$, s.t.
		% $\expr_{\snap''} \fbigstep \varx$.
		%
		\\
		%
		By Lemma \ref{lem:sign}, \ref{lem:zeroboundr} and \ref{lem:zeroboundl} we know $s = 1$, 
		$U \in (0, \fvarR)$ for some $\fvarR$
		, $U \in (0, \rvarR)$ for some $\rvarR$ s.t.:.
		%
		$$f(a) + \frac{1}{\epsilon} \times \ln(\fvarR) \fbigstep -b' ~~~ (1)
		~~~~~~~~~~~~~~~~
		f(a) + \frac{1}{\epsilon} \times \ln(\rvarR) \rbigstep -b' ~~~ (2)
		$$
		\\
		%
		So we can get $\fvarL = \rvarL = 0$.
		%
		\\
		%
		In order to bound the $\fvarR$ by $\rvarR$, let $\tilde{\fvarR}$ be the real representation of $\fvarR~(\star)$ . we have the derivation as following:
		\\
		% \begin{figure}
{\scriptsize
\begin{mathpar}
	\inferrule*[right=bop]
	{
	\inferrule*[right=bop]
	{
		\inferrule*[right = uop]
		{
			\inferrule*[right = val-eq]
			{
				\tilde{\fvarR} ~~~ \text{representable}
			}
			{
				\trsenv, \tilde{\fvarR} 
				\trsto
				(\fvarR,
				(\tilde{\fvarR}, \tilde{\fvarR}))
			}
		}
		{
				\trsenv,
				\ln(\tilde{\fvarR}) 
				\trsto
				(\ln(\fvarR), \ln(\tilde{\fvarR})(1 + \eta),
				\frac{\ln(\tilde{\fvarR})}{(1 + \eta)})
			}
			\and
			\inferrule*[right = val-eq]
			{
			\frac{1}{\epsilon}~ \text{representable}
			}
			{
				\frac{1}{\epsilon} \trsto
				(\frac{1}{\epsilon},
				(\frac{1}{\epsilon},
			\frac{1}{\epsilon}))	
			}
		}
			{
					\trsenv,
					\frac{1}{\epsilon} \times \ln(\tilde{\fvarR}) 
					\trsto
					(\frac{1}{\epsilon} \times \ln(R),
					((\frac{1}{\epsilon} \times \ln(\tilde{\fvarR}))(1 + \eta)^2,
					%
					\frac{\frac{1}{\epsilon} \times \ln(\tilde{\fvarR})}{(1 + \eta)^2}))
				}
					\and
				\inferrule*[right=val-eq]
				{f(a) ~~ \text{representable}}
				{	[],
				f(a) \trsto
				(f(a),(f(a),f(a)))
				}
			}
				{
				\trsenv,
				f(a) + \frac{1}{\epsilon} \times \ln(\tilde{\fvarR})
				\trsto
				\bigg(
				f(a) + \frac{1}{\epsilon} \times \ln(\fvarR),
						%
				\big( (f(a) + 
				(\frac{1}{\epsilon} \times \ln(\tilde{\fvarR}))
				(1 + \eta)^2)
				{(1 + \eta)},
				%
				\frac{(
				f(a) + \frac{\frac{1}{\epsilon} 
				\times \ln(\tilde{\fvarR})}
				{(1 + \eta)^2}
				)}
				{(1 + \eta)}
				 \big)
				\bigg)
				}
		\end{mathpar}	
}		%
		%
		By Soundness Theorem and $(1)$, we have:
		% 
		%
		\[
		(f(a) + 
				(\frac{1}{\epsilon} \times \ln(\tilde{\fvarR}))
				(1 + \eta)^2)
				{(1 + \eta)}
		\leq
		-b'
		\leq
		\frac{(
				f(a) + \frac{\frac{1}{\epsilon} 
				\times \ln(\tilde{\fvarR})}
				{(1 + \eta)^2}
				)}
				{(1 + \eta)}
				 \big)
				\bigg)
		\]
		Then we can get:
		%
		\[
		e^{\epsilon 
		\big( (-b'(1 + \eta) - f(a)) (1 + \eta)^2) \big)}
		\leq
		\tilde{\fvarR}
		\leq
		e^{\epsilon 
		\frac{(\frac{-b'}{1 + \eta} - f(a))}{(1 + \eta)^2}}
		\]
		%
		By $(\star)$ we know: $\tilde{\fvarR} = \fvarR$, so we have:
		%
		\[
		e^{\epsilon 
		\big( (-b'(1 + \eta) - f(a)) (1 + \eta)^2) \big)}
		\leq
		\fvarR
		\leq
		e^{\epsilon 
		\frac{(\frac{-b'}{1 + \eta} - f(a))}{(1 + \eta)^2}}
		\]
		%
		\\		
		%
		In the same way, we get:
		%
		%
		\[
		e^{\epsilon 
		\big((-b'(1 + \eta) - f(a'))(1 + \eta)^2 \big)}
		\leq
		\fvarR'
		%
		\leq
		e^{\epsilon 
		(\frac{(\frac{-b'}{1 + \eta} - f(a'))}{(1 + \eta)^2})}
		\]
		%		
		Then we have that the $|\fvarR - \fvarL| = |\fvarR - 0|$ is bounded by 
		$e^{\epsilon 
		\frac{(\frac{-b'}{1 + \eta} - f(a))}{(1 + \eta)^2}}$ 
		and $e^{\epsilon 
		\frac{(\frac{-b'}{1 + \eta} - f(a))}{(1 + \eta)^2}}$;
		%
		\\
		and $|\fvarR' - \fvarL'| = |\fvarR' - 0|$ is bounded by 
		$e^{\epsilon 
		\frac{(\frac{-b'}{1 + \eta} - f(a'))}{(1 + \eta)^2}}$ 
		and $e^{\epsilon 
		\frac{(\frac{-b'}{1 + \eta} - f(a'))}{(1 + \eta)^2}}$;
		

		So the privacy loss in this case is bounded by:
		%
		\[
		\begin{array}{ll}
		\frac
		{\frac{1}{2}(e^{\epsilon 
		\frac{(\frac{-b'}{1 + \eta} - f(a))}{(1 + \eta)^2}})}
		{\frac{1}{2}(e^{\epsilon 
		\big((-b'(1 + \eta) - f(a'))(1 + \eta)^2 \big)})}
		& = e^{\epsilon
		\bigg(
		\frac{(\frac{-b'}{1 + \eta} - f(a))}{(1 + \eta)^2}
		-
		\big((-b'(1 + \eta) - f(a'))(1 + \eta)^2 \big)
		\bigg)}\\
		& = e^{\epsilon
		\bigg(
		\frac{-b'}{(1 + \eta)^3} - \frac{f(a)}{(1 + \eta)^2}
		-
		x(1 + \eta)^3 + f(a')(1 + \eta)^2 
		\bigg)} ~ (\triangle)
		\end{array}
		\]
		%
		Since $ (1 + \eta)^3 > 1 + 3\eta$,  $\frac{1}{(1 + \eta)^3} < \frac{1}{1 + 3\eta} $, $(1 + \eta)^2 < 1 + 2.1\eta$ and $\frac{1}{(1 + \eta)^2} > 1 - 2 \eta$, we have:
		%
		\[
		\begin{array}{ll}
		e^0
		<
		~~~ (\triangle) & < e^{\epsilon \big( 
		\frac{9\eta + 6}{1 + 3\eta} b'
		+ 4.1 \eta f(a)
		+ (1 + 2.1\eta) 
		\big)}\\
		%
		& < e^{\epsilon(10.1 \eta B + 1 + 2.1\eta)}
		\end{array}
		\]
		%
		%
		%
		%
		\caseL{$\boldsymbol{x \in (-B, \round{f(a)}_{\Lambda})}$}
		%
		\subcaseL{$\boldsymbol{\round{f(a)}_{\Lambda} \leq 0 \lor \bigg( \round{f(a)}_{\Lambda} > 0 \land x \in (-B, 0) \bigg) } $}
		%
		Let $y_1 = x - (\frac{\Lambda}{2})$, $y_2 = x + (\frac{\Lambda}{2})$, we know $y_1 < 0$, $y_2 < 0$.
		%
		\\
		%
		Let $L = e^{\epsilon(y_1 - f(a))}$ and $R = e^{\epsilon(y_2 - f(a))}$, we have: $\forall u \in (L, R)$:
		$\expr_{\snap''} \rbigstep \varx$.
		%
		\\
		%
		Let $l$ and $r$ be the range where $\forall u \in (l, r)$ and $s = 1$, s.t.
		$\expr_{\snap''} \fbigstep \varx$.
		%
		\\
		%
		So we know: $\ubar{L} < l < \bar{L}$, $\ubar{R} < r < \bar{R}$ s.t.:
		%
		$$f(a) + \frac{1}{\epsilon} \times \ln(l) \fbigstep y_1
		\land
		f(a) + \frac{1}{\epsilon} \times \ln(r) \fbigstep y_2.$$
		%
		The transition from $R$ to $r$ given the transition environment 
		$\trsenv = [U \mapsto (R, (\ubar{R}, \bar{R})), S \mapsto (1, (1, 1))]$ is shown as following:
		%
		\begin{mathpar}
		\inferrule[ln]
		{
			\inferrule*[right = val-eq]
			{
				\empty
			}
			{
				\trsenv, R 
				\trsto
				(R, (\ubar{R}, \bar{R})
			}
		}
		{
			\inferrule[op]
			{
				\trsenv, \ln(U) 
				\trsto
				(\ln(R) ,
				(\ln(\ubar{R})(1 + \eta),
				\frac{\ln(\bar{R})}{(1 + \eta)}))
			}
			{
				\inferrule[op]
				{
					\trsenv, \frac{1}{\epsilon} \times \ln(U) 
					\trsto
					\big(
					\frac{1}{\epsilon} \times \ln(R)
					,
					((\frac{1}{\epsilon} \times \ln(\ubar{R}))(1 + \eta)^2,
					%
					\frac{\frac{1}{\epsilon} \times \ln(\bar{R})}{(1 + \eta)^2})
					\big)
				}
				{
					\inferrule[id]
					{
					\trsenv,f(a) + \frac{1}{\epsilon} \times \ln(U)
						\trsto
						\Big(
						f(a) + \frac{1}{\epsilon} \times \ln(R)
						,
						\bigg(
						\big( f(a) + 
						(\frac{1}{\epsilon} \times \ln(\ubar{R}))
						(1 + \eta)^2 \big)
						{(1 + \eta)},
						%
						\frac{(
						f(a) + \frac{\frac{1}{\epsilon} \times \ln(\bar{R})}
						{(1 + \eta)^2}
						)}
						{(1 + \eta)}
						\bigg)
						\Big)
					}
					{
						\trsenv, \snap'(a)
						\trsto
						\trsenv
						[y \mapsto 
						(f(a) + \frac{1}{\epsilon} \times \ln(R),
						(\expr^1,
						%
						\expr^2))]
					}
				}
			}
		}
		\end{mathpar}
		From soundness theorem, we have  $\expr^1 \leq y_2 \leq \expr^2$, where we can get:
		%
		\\ 
		$\ubar{R} = e^{\epsilon 
				\big( (y_1(1 + \eta) - f(a)) (1 + \eta)^2) \big)}$ and
		%
		$\bar{R} = e^{\epsilon 
				\frac{(\frac{y_1}{1 + \eta} - f(a))}{(1 + \eta)^2}}$.
		%
		The transition from $L$ to $l$ given the transition environment 
		$\trsenv = [U \mapsto (L, (\ubar{L}, \bar{L})), S \mapsto (1, (1, 1))]$ is shown as following:
		%
		\begin{mathpar}
		\inferrule
		{
		\dots
		}
		{
		 \trsenv, \snap'(a)
		 \trsto
		 \trsenv[
		 z \mapsto(
		 f(a) + (\frac{1}{\epsilon} \times \ln(\ubar{L})
		 ,
		 \big(
		 \frac{f(a) + 
		 (\frac{1}{\epsilon} \times \ln(\ubar{L}))
		 (1 + \eta)^2}
		 {1 + \eta},
		 %
		 (
		 f(a) + \frac{\frac{1}{\epsilon} \times \ln(\bar{ L})}
		 {(1 + \eta)^2}
		 )(1 + \eta)
		 )
		 \big)
		 ]
		}
		\end{mathpar}
		%
		From soundness theorem, we have  $\err_1 \leq y_2 \leq \err_2$.\\
		%
		%
		Taking the lower bound , we have:
		$\ubar{L} = e^{\epsilon 
				\big( (y_2(1 + \eta) - f(a)) (1 + \eta)^2) \big)}$.\\
		%
		Taking the upper bound, we have: 
		$\bar{L} = e^{\epsilon 
				\frac{(\frac{y_2}{1 + \eta} - f(a))}{(1 + \eta)^2}}$.

		In the same way, we have the bound of $l, r$ for adjacent data set $a'$:

		$$\ubar{R'} = e^{\epsilon 
				\big( (y_1(1 + \eta) - f(a')) (1 + \eta)^2) \big)},  ~
		\bar{R'} = e^{\epsilon 
				\frac{(\frac{y_1}{1 + \eta} - f(a'))}{(1 + \eta)^2}}.$$
		%
		%
		$$ 
		\ubar{L'} = e^{\epsilon 
				\big( (y_2(1 + \eta) - f(a')) (1 + \eta)^2) \big)}, ~ 
		\bar{L'} = e^{\epsilon 
				\frac{(\frac{y_2}{1 + \eta} - f(a'))}{(1 + \eta)^2}}$$

		Then, we have the privacy loss is bounded by:
		\[
		\frac{|\bar{R} - \ubar{L}|}{|\ubar{R'} - \bar{L'}|}.		
		\]
		%
		We also have:
		\[
		\begin{array}{ll}
		\frac{\bar{R}}{R} 
		& = e^{\epsilon
		\big(\frac{y_1}{(1 + \eta)^3} - \frac{f(a)}{(1 + \eta) ^2} 
		- y_1 + f(a) \big)}
		 \leq e^{\epsilon
		\big(- \frac{3 \eta}{1 +3 \eta}y_1 + 2 \eta f(a) \big)}
		 \leq e^{\epsilon
		\big(\frac{3 \eta}{1 +3 \eta}B + 2 \eta B \big)}
		\leq e^{5 \epsilon B \eta}\\
%
		\frac{\ubar{L}}{L} 
		& = e^{\epsilon
		\big({y_2}{(1 + \eta)^3} - {f(a)}{(1 + \eta) ^2} 
		- y_2 + f(a) \big)}
		 \geq e^{\epsilon
		\big({3 \eta}y_1 - 2 \eta f(a) \big)}
		\geq e^{-5 \epsilon B \eta}\\
		\end{array}
		\]
		%
		Then, we can derive:
		%
		\[
		\begin{array}{ll}
		|\bar{R} - \ubar{L}| 
		& \leq e^{5 \epsilon B \eta}R - e^{-5 \epsilon B \eta} L \\
		& = L \big(  e^{ \Lambda\epsilon + 5 \epsilon B \eta} 
		- e^{-5 \epsilon B \eta} \big)\\
		& = L \big( e^{ \Lambda\epsilon} e^{5 \epsilon B \eta}
		-e^{5 \epsilon B \eta}
		+  e^{5 \epsilon B \eta}
		- e^{-5 \epsilon B \eta} \big)\\
		& = L \big( e^{ \Lambda\epsilon} e^{5 \epsilon B \eta}
		- e^{5 \epsilon B \eta}
		+  \frac{1}{(e^{ \Lambda\epsilon} - 1)}
		(e^{ \Lambda\epsilon} - 1)e^{5 \epsilon B \eta}
		- e^{-5 \epsilon B \eta} \big)\\
		& \leq L \big( e^{ \Lambda\epsilon} e^{5 \epsilon B \eta}
		- e^{5 \epsilon B \eta}
		+  \frac{1}{(e - 1)}
		(e^{ \Lambda\epsilon} - 1)e^{5 \epsilon B \eta}
		- e^{-5 \epsilon B \eta} \big) ~( by 1 \leq \Lambda \epsilon < 2)\\ 
		& = L  \frac{e}{(e - 1)} \big( e^{ \Lambda\epsilon} e^{5 \epsilon B \eta}
		- e^{5 \epsilon B \eta}
		- e^{-5 \epsilon B \eta} \big)\\
		& < L  \frac{e}{(e - 1)} \big( e^{ \Lambda\epsilon} e^{5 \epsilon B \eta}
		- e^{5 \epsilon B \eta} \big)\\
		& = L (e^{ \Lambda\epsilon} -  1) e^{\ln(\frac{e}{(e - 1)}) + 5 \epsilon B \eta}\\
		& < L (e^{ \Lambda\epsilon} -  1)e^{11 \epsilon B \eta}~ (by ~ ( \frac{1}{\epsilon} < B < 2^{42} \frac{1}{\epsilon} )) \\
		& = (R - L)e^{11 \epsilon B \eta}
		\end{array}
		\]
		In the same way, we can derive:
		\[
		|\ubar{R} - \bar{L}| > e^{-5 \epsilon B\eta} R - e^{5 \epsilon B\eta} L
		> (R - L)e^{- 12 \epsilon B \eta}
		\]
		Then we have:
		\[
		\frac{|\bar{R} - \ubar{L}|}{|\ubar{R'} - \bar{L'}|}
		< e^{(23 \epsilon B \eta + \epsilon)}.		
		\]
%
		\subcaseL{$\boldsymbol{\round{f(a)}_{\Lambda} > 0 \land x \in (0, \round{f(a)}_{\Lambda}) } $}
		%

		Let $y_1 = x - (\frac{\Lambda}{2})$, $y_2 = x + (\frac{\Lambda}{2})$, we know $y_1 > 0$, $y_2 > 0$.
		%
		\\
		%
		Let $L = e^{\epsilon(y_1 - f(a))}$ and $R = e^{\epsilon(y_2 - f(a))}$, we have: $\forall u \in (L, R)$:
		$\expr_{\snap''} \rbigstep \varx$.
		%
		\\
		%
		Let $l$ and $r$ be the range where $\forall u \in (l, r)$ and $s = 1$, s.t.
		$\expr_{\snap''} \fbigstep \varx$.
		%
		\\
		%
		So we know: $\ubar{L} < l < \bar{L}$, $\ubar{R} < r < \bar{R}$ s.t.:
		%
		$$f(a) + \frac{1}{\epsilon} \times \ln(l) \fbigstep y_1
		\land
		f(a) + \frac{1}{\epsilon} \times \ln(r) \fbigstep y_2.$$
		%

		The transition from $L$ to $l$ given the transition environment 
		$\trsenv = [U \mapsto (L, (\ubar{L}, \bar{L})), S \mapsto (1, (1, 1))]$ is shown as following:
		% in Figure. \ref{fig_der_snap2}
		% \begin{figure}
		\begin{mathpar}
		\inferrule
		{
			\trsenv, U 
			\trsto
			(
			L,
			(\ubar{L}, \bar{L})
			)
		}
		{
			\inferrule
			{
				\trsenv, \ln(U) 
				\trsto
				(\ln(L),
				(\ln(\ubar{L})(1 + \eta),
				\frac{\ln(\bar{L})}{(1 + \eta)}))
			}
			{
				\inferrule
				{
					\trsenv, \frac{1}{\epsilon} \times \ln(U) 
					\trsto
				\big(
				\frac{1}{\epsilon} \times \ln(L) 
					,
					((\frac{1}{\epsilon} \times \ln(\ubar{L}))(1 + \eta)^2,
					%
					\frac{\frac{1}{\epsilon} \times \ln(\bar{L})}{(1 + \eta)^2})
					\big)
				}
				{
					\inferrule
					{
						\trsenv, f(a) + \frac{1}{\epsilon} \times \ln(U) 
						\trsto
						\big(
						f(a) + \frac{1}{\epsilon} \times \ln(l)
						,
						(
						\frac{f(a) + 
						(\frac{1}{\epsilon} \times \ln(\ubar{L}))
						(1 + \eta)^2}
						{1 + \eta},
						%
						(
						f(a) + \frac{\frac{1}{\epsilon} \times \ln(\bar{L})}
						{(1 + \eta)^2}
						)(1 + \eta)
						)
						\big)
					}
					{
					\trsenv, \snap'(a)
					\trsto
					\trsenv[
					y \mapsto 
					(f(a) + \frac{1}{\epsilon} \times \ln(L)
					,
					(
					\err_1,
					%
					\err_2
					))
					}
				}
			}
		}
		\end{mathpar}
		%
		%
		From soundness theorem, we have  $\err_1 \leq y_1 \leq \err_2$, then we can get:
		%
		\\
		$\ubar{L} = e^{(y_1 / (1 + \eta) - f(a))(1 + \eta)^2\epsilon}$ and
		$\bar{L} = e^{(y_1 (1 + \eta) - f(a))\epsilon/(1 + \eta)^2}$.
		%
		%
		The transition from $R$ to $r$ given the transition environment 
		$\trsenv = [U \mapsto (R, (\ubar{R}, \bar{R})), S \mapsto (1, (1, 1))]$ is shown as following:
%
%
		\begin{mathpar}
		\inferrule
		{
			\dots
		}
		{
   		\trsenv, \snap'(a)
   		\trsto
   		\trsenv[y \mapsto 
   		\big(
   		f(a) + \frac{1}{\epsilon} \times \ln(R)
   		,
   		(
   		\frac{f(a) + 
   		(\frac{1}{\epsilon} \times \ln(\ubar{R}))
   		(1 + \eta)^2}
   		{1 + \eta},
   		%
   		(
   		f(a) + \frac{\frac{1}{\epsilon} \times \ln(\bar{R})}
   		{(1 + \eta)^2}
   		)(1 + \eta)
   		)\big)
   		]
		}
		\end{mathpar}
		%
		From soundness theorem, we have  $\err_1 \leq y_2 \leq \err_2$.
		%
\\
		%
		Taking the lower bound (i.e. $\err_1 = y_2$), we have:
		$\ubar{R} = e^{(y_2 / (1 + \eta) - f(a))(1 + \eta)^2\epsilon}$.
		Taking the upper bound (i.e. $\err_2 = y_1$), we have:
		$\bar{R} = e^{(y_2 (1 + \eta) - f(a))\epsilon/(1 + \eta)^2}$.
		%

		%
		For the adjacent input $a$, we first have the same setting as input $a$ for $R'$ and $L'$.
		Then the transition from $L'$ to $l'$ given the transition environment 
		$\trsenv = [U \mapsto (L', (\ubar{L'}, \bar{L'})), S \mapsto (1, (1, 1))]$ is shown as following:
		\begin{mathpar}
		\inferrule
		{
			\dots
		}
		{
			\trsenv, \snap'(a')
			\trsto
			\trsenv[ y \mapsto
			\big(f(a) + \frac{1}{\epsilon} \times \ln(L')
			,
			(
			\frac{f(a') + 
			(\frac{1}{\epsilon} \times \ln(\ubar{L'}))
			(1 + \eta)^2}
			{1 + \eta},
			%
			(
			f(a') + \frac{\frac{1}{\epsilon} \times \ln(\bar{L'})}
			{(1 + \eta)^2}
			)(1 + \eta)
			)\big)
			]
		}
		\end{mathpar}
		%
		From soundness theorem, we have  $\err_1 \leq y_2 \leq \err_2$.
		%

		Taking the lower bound (i.e. $\err_1 = y_1$), we get:
		$\ubar{L'} = e^{(y_1 / (1 + \eta) - f(a'))(1 + \eta)^2\epsilon}$.
		%
		Taking the upper bound (i.e. $\err_2 = y_1$), we get:
		$\bar{L'} = e^{(y_1 (1 + \eta) - f(a'))\epsilon/(1 + \eta)^2}$.
		%
		%
		The transition from $R'$ to $r'$ given the transition environment 
		$\trsenv = [U \mapsto (R', (\ubar{R'}, \bar{R'})), S \mapsto (1, (1, 1))]$ is shown as following:
		\begin{mathpar}
		\inferrule
		{
			\dots
		}
		{
			\trsenv, \snap'(a')
			\trsto
			\trsenv[y \mapsto
			\big(f(a) + \frac{1}{\epsilon} \times \ln(R') 
			,
			(
			\frac{f(a') + 
			(\frac{1}{\epsilon} \times \ln(\ubar{R'}))
			(1 + \eta)^2}
			{1 + \eta},
			%
			(
			f(a') + \frac{\frac{1}{\epsilon} \times \ln(\bar{R'})}
			{(1 + \eta)^2}
			)(1 + \eta)
			) \big)
			]
		}
		\end{mathpar}
		From soundness theorem, we have  $\err_1 \leq y_2 \leq \err_2$.
		%
\\
		%
		Taking the lower bound (i.e. $\err_1 = y_2$), we have:
		$\ubar{R'} = e^{(y_2 / (1 + \eta) - f(a'))(1 + \eta)^2\epsilon}$.
		Taking the upper bound (i.e. $\err_2 = y_1$), we have:
		$\bar{R'} = e^{(y_2 (1 + \eta) - f(a'))\epsilon/(1 + \eta)^2}$.
		%
		\\
		%
		%
		%
		We have the privacy loss is bounded by:
		%
		\[
		\frac{|\bar{R} - \ubar{L}|}{|\ubar{R'} - \bar{L'}|}
		\]
		%
\\
		%
		Since the following bound can be proved by using $1 - 2\eta < (1 + \eta)^2 < 1 + 2.1\eta, y_1 > -B, y_2 > -B $ and simple approximation:
		\[
		\bar{R} - \ubar{L} < (R - L) e^{(5B\eta\epsilon)}, 
		\ubar{R'} - \bar{L'} > (R'  - L') e^{-7B\eta \epsilon}
		\]

		We also have the $\snap(a)$ is $\epsilon$-dp:
		\[
		\frac{|R - L|}{|R' - L'|} = e^{\epsilon}
		\]
		So we can get:
		\[
		\frac{|\bar{R} - \ubar{L}|}{|\ubar{R'} - \bar{L'}|}
		< \frac{|R - L|}{|R' - L'|} e^{(12B\eta\epsilon)}
		= e^{(1 + 12B\eta)\epsilon}
		\]		
%
%
		\subcaseL{$\boldsymbol{\round{f(a)}_{\Lambda} > 0 \land x = 0 } $}

		Let $y_1 = x - (\frac{\Lambda}{2})$, $y_2 = x + (\frac{\Lambda}{2})$, we know $y_1 < 0$, $y_2 > 0$.
		\\
		%
		Let $L = e^{\epsilon(y_1 - f(a))}$ and $R = e^{\epsilon(y_2 - f(a))}$, we have: $\forall u \in (L, R)$:
		$\expr_{\snap''} \rbigstep \varx$.
		%
		\\
		%
		Let $l$ and $r$ be the range where $\forall u \in (l, r)$ and $s = 1$, s.t.
		$\expr_{\snap''} \fbigstep \varx$.
		%
		\\
		%
		So we know: $\ubar{L} < l < \bar{L}$, $\ubar{R} < r < \bar{R}$ s.t.:
		%
		$$f(a) + \frac{1}{\epsilon} \times \ln(l) \fbigstep y_1
		\land
		f(a) + \frac{1}{\epsilon} \times \ln(r) \fbigstep y_2.$$
		%
	%
		The transition from $L$ to $l$ given the transition environment 
		$\trsenv = [U \mapsto (L, (\ubar{L}, \bar{L})), S \mapsto (1, (1, 1))]$ is shown as following:
		%
		%
		\begin{mathpar}
		\inferrule
		{
		\dots
		}
		{
		\trsenv, \snap'(a)
		\trsto
		\trsenv[y \mapsto
		\big(
		f(a) + 
		\frac{1}{\epsilon} \times \ln(\ubar{L})
		,
		(
		\frac{f(a) + 
		(\frac{1}{\epsilon} \times \ln(\ubar{L}))
		(1 + \eta)^2}
		{1 + \eta},
		%
		(
		f(a) + \frac{\frac{1}{\epsilon} \times \ln(\bar{L})}
		{(1 + \eta)^2}
		)(1 + \eta)
		) \big)
		]
		}
		\end{mathpar}
		%
		From soundness theorem, we have  $\err_1 \leq y_2 \leq \err_2$.\\
		%
		%
		Taking the lower bound , we have:
		$\ubar{L} = e^{\epsilon 
				\big( (y_2(1 + \eta) - f(a)) (1 + \eta)^2) \big)}$.\\
		%
		Taking the upper bound, we have: 
		$\bar{L} = e^{\epsilon 
				\frac{(\frac{y_2}{1 + \eta} - f(a))}{(1 + \eta)^2}}$.  
		%
		%
		The transition from $R$ to $r$ given the transition environment 
		$\trsenv = [U \mapsto (R, (\ubar{R}, \bar{R})), S \mapsto (1, (1, 1))]$ is shown as following:
		\begin{mathpar}
		\inferrule
		{
		 \dots
		}
		{
		\trsenv, \snap'(a)
		\trsto
		\trsenv[y \mapsto
		\big(
		f(a) + 
		\frac{1}{\epsilon} \times \ln(\ubar{R})
		,
		(
		\frac{f(a) + 
		(\frac{1}{\epsilon} \times \ln(\ubar{R}))
		(1 + \eta)^2}
		{1 + \eta},
		%
		(
		f(a) + \frac{\frac{1}{\epsilon} \times \ln(\bar{R})}
		{(1 + \eta)^2}
		)(1 + \eta)
		) \big)
		]
		}
	   \end{mathpar}
		%
		From soundness theorem, we have  $\err_1 \leq y_2 \leq \err_2$.
		%
		%
		Taking the lower bound (i.e. $\err_1 = y_2$), we have:
		$\ubar{R} = e^{(y_2 / (1 + \eta) - f(a))(1 + \eta)^2\epsilon}$.
		Taking the upper bound (i.e. $\err_2 = y_1$), we have:
		$\bar{R} = e^{(y_2 (1 + \eta) - f(a))\epsilon/(1 + \eta)^2}$.
		%
		Using the bound we proved before, we have the folloing bound on $|\bar{R} - \ubar{L}|$ and $|\ubar{R} - \bar{L}|$:
		%
		\[
		\begin{array}{ll}
		\bar{R} - \ubar{L} 
		& < 
		e^{(2B\eta \epsilon)}R - e^{-5B\eta \epsilon} L < 
		(R - L) e^{6B\eta \epsilon}\\
		\ubar{R} - \bar{L}
		& > e^{(-3B\eta \epsilon)}R - e^{5B\eta \epsilon} L > (R - L) e^{-8B\eta \epsilon},
		\end{array}
		\]

		and privacy loss is bounded by:
		%
		\[
		\frac{|\bar{R} - \ubar{L}|}{|\ubar{R'} - \bar{L'}|}
		< e^{14B\eta \epsilon + \epsilon}
		\]

		%
		\caseL{$\boldsymbol{x = \round{f(a)}_{\Lambda}}$}
		%
		%
		This case can also be split into 3 subcases by: $\round{f(a)}_{\Lambda} < 0$, $\round{f(a)}_{\Lambda} = 0$ and $\round{f(a)}_{\Lambda} > 0$. 
		%
		Without loss of generalization, we consider the worst case where the error propagate in the same direction, i.e. $\round{f(a)}_{\Lambda} < 0$.\\
		%
		From this assumption, let $y_1 = x - \frac{\Lambda}{2}$, $y_2 = x + \frac{\Lambda}{2}$, we know $y_1 < 0$, $y_2 < 0$.
		%
		Since $f(a) + 1 = f(a')$, we also have $\round{f(a)} < \round{f(a')}$.
		So, we know $s$ can only be $1$ for input $a'$ but $s$ can be $1$ or $-1$ for input $a$.
		%
		\\
		%
		%
		For input $a$,
		Let $L = e^{\epsilon(y_1 - f(a))}$ and $R = e^{\epsilon(y_2 - f(a))}$, we have: $\forall u \in (L, R)$:
		$\expr_{\snap''} \rbigstep \varx$.
		%
		\\
		%
		Let $l$ and $r$ be the range where $\forall u \in (l, r)$ and $s = 1$, s.t.
		$\expr_{\snap''} \fbigstep \varx$.
		%
		\\
		%
		So we know: $\ubar{L} < l < \bar{L}$, $\ubar{R} < r < \bar{R}$ s.t.:
		%
		$$f(a) + \frac{1}{\epsilon} \times \ln(l) \fbigstep y_1
		\land
		f(a) + \frac{1}{\epsilon} \times \ln(r) \fbigstep y_2.$$
		%
		%
		Induction on $s$, we have when $s = 1$:
		%
		\\
		%
		The transition from $R$ to $r$ given the transition environment 
		$\trsenv = [U \mapsto (R_{+}, (\ubar{R_{+}}, \bar{R_{+}})), S \mapsto (1, (1, 1))]$ is shown as following:
		%
		\begin{mathpar}
		\inferrule
		{
		 \trsenv, U
		 \trsto
		 R_{+}, ( \ubar{R_{+}}, \bar{R_{+}} )
		}
		{
		 \inferrule
		 {
		  \trsenv, \ln(U)
		  \trsto
		  \big(\ln (R_{+}), 
		  (\ln(\ubar{R_{+}})(1 + \eta), \frac{\ln(\bar{R_{+}})}{1+\eta})
		  \big)
		 }
		 {
		  \inferrule
		  {
		  \trsenv,  \frac{1}{\epsilon}\ln(U)
		   \trsto
		   \bigg(\frac{1}{\epsilon} \times \ln (R_{+}), 
		   \big(
		   \frac{1}{\epsilon}\ln(\ubar{R_{+}})(1 + \eta)^2, 
		   \frac{1}{\epsilon}\frac{\ln(\bar{R_{+}})}{(1+\eta)^2}
		   \big) \bigg)
		  }
		  {
		   \inferrule
		   {
		    \trsenv, f(a) + \frac{1}{\epsilon}\ln(U)
			\trsto
			\bigg(f(a) + \frac{1}{\epsilon} \times \ln (R_{+}), 
			\big(
			(f(a) + \frac{1}{\epsilon}\ln(\ubar{R_{+}})(1 + \eta)^2)(1 + \eta), 
			(f(a) + \frac{1}{\epsilon}\frac{\ln(\bar{R_{+}})}{(1+\eta)^2}) / (1 + \eta) \big)
			\bigg)
		   }
		   {
		   \trsenv, \snap'(a) \trsto 
		   \trsenv[y \mapsto
		   (f(a) + \frac{1}{\epsilon} \times \ln (R_{+}), 
		   (\err_1, \err_2))
		   ]
		   }
		  }
		 }
		}
		\end{mathpar}
		%
		%
		From soundness theorem, we have  $\err_1 \leq y_2 \leq \err_2$. Then we can get following bounds for $r$:\\
		%
		%
		$\ubar{R_+} = e^{\epsilon 
				\big( (y_2(1 + \eta) - f(a)) (1 + \eta)^2) \big)}$, 
		%
		$\bar{R_+} = e^{\epsilon 
				\frac{(\frac{y_2}{1 + \eta} - f(a))}{(1 + \eta)^2}}$.  
		%
		\\
		%
		Since $y_2 = \round{f(a)} + \frac{\Lambda}{2}$, we have $e^{\epsilon 
				\big( (y_2 - f(a))) \big)} > 1$, so actually we know $R = r = 1$.
		%
		% 
		\\
		%
		We can also derive the bound for $l$ in the same way as:\\
		$\ubar{L_+} = e^{\epsilon 
				\big( (y_1(1 + \eta) - f(a)) (1 + \eta)^2) \big)}$, 
		%
		$\bar{L_+} = e^{\epsilon 
				\frac{(\frac{y_1}{1 + \eta} - f(a))}{(1 + \eta)^2}}$.
		%

		% 
		When $s = -1$, we can derive following bounds in the same way for $l$ and $r$:\\
		%
		$\ubar{L_-} = e^{\epsilon 
				\big( (f(a) - y_2(1 + \eta)) (1 + \eta)^2) \big)}$,
		%
		$\bar{L_-} = e^{\epsilon 
				\frac{(f(a) - \frac{y_2}{1 + \eta})}{(1 + \eta)^2}}$.\\
		%
		% 
		$\ubar{R_2} = e^{\epsilon 
				\big( (f(a) - y_1(1 + \eta)) (1 + \eta)^2) \big)}$, 
		%
		$\bar{R_2} = e^{\epsilon 
				\frac{(f(a) - \frac{y_1}{1 + \eta})}{(1 + \eta)^2}}$.\\
		%
		Since $y_1 = \round{f(a)} - \frac{\Lambda}{2}$, 
		%
		we have $e^{\epsilon \big( (f(a) - y_1)) \big)} > 1$, so actually we know $R' = r' = 1$.
		%
		\\
		%
		For input $a'$, we have only one case where $s = 1$, the following bound can be derived:
		%
		\\
		%
		$\ubar{R'} = e^{\epsilon 
				\big( (y_2(1 + \eta) - f(a')) (1 + \eta)^2) \big)}$, 
		%
		$\bar{R'} = e^{\epsilon 
				\frac{(\frac{y_2}{1 + \eta} - f(a'))}{(1 + \eta)^2}}$.  
		% 
		\\
		%
		$\ubar{L'} = e^{\epsilon 
				\big( (y_1(1 + \eta) - f(a')) (1 + \eta)^2) \big)}$, 
		%
		$\bar{L'} = e^{\epsilon 
				\frac{(\frac{y_1}{1 + \eta} - f(a'))}{(1 + \eta)^2}}$.

		% \begin{mathpar}
		% \end{mathpar}

		We have following bounds on their ratios:
		%
		\[
		\frac{\ubar{R_+}}{R_+} = e^{\epsilon 
		\big(
		(1 + \eta)^3 y_2 - (1 + \eta)^2f(a) - y_2 + f(a)
		\big)}
		> e^{-3\epsilon B \eta},
		\frac{\bar{R_+}}{R_+} = e^{\epsilon 
		\big(
		frac{y_2}{(1 + \eta)^3} - \frac{f(a)}{(1 + \eta)^2} - y_2 + f(a)
		\big)}
		< e^{3\epsilon B \eta},
		\]
		%
		The same bound for $L_+$ by substituting $y_2$ with $y_1$, and similar bound for $L', R'$.
		%
		\[
		\frac{\ubar{R'}}{R} = e^{\epsilon 
		\big(
		(1 + \eta)^2f(a) - (1 + \eta)^3 y_2 - f(a) + y_2
		\big)}
		> e^{-2\epsilon B \eta},
		%
		\frac{\bar{R'}}{R'} = e^{\epsilon 
		\big(
		\frac{f(a)}{(1 + \eta)^2} - frac{y_2}{(1 + \eta)^3} - f(a) + y_2
		\big)}
		< e^{2\epsilon B \eta},
		\]
		%
		Using the bound on their ratios, we can get following bounds on $|\bar{R_+} - \ubar{L_+}|$ and $|\ubar{R'} - \bar{L'}|$:
		%
		\[
		|\bar{R_+} - \ubar{L_+}| < e^{3\epsilon B \eta} R - e^{-3\epsilon B \eta}L < (R - L) e^{7\epsilon B \eta},
		%
		|\ubar{R'} - \bar{L'}| > e^{-2\epsilon B \eta} R - e^{2\epsilon B \eta}L > (R' - L') e^{-5\epsilon B \eta}
		\]
		%
		Then we have the following bounds on privacy loss:
		%
		\[
		\frac{2 - (\ubar{L_+} + \ubar{L_-})}{\ubar{R'} - \bar{L'}}
		< \frac{\bar{R_+} - \ubar{L_+}}{\ubar{R'} - \bar{L'}}
		< \frac{e^{7\epsilon B \eta}(R_+ - L_+)}
		{e^{-5\epsilon B \eta} (R' - L')}
		= e^{12\epsilon B \eta + \epsilon}
		\]
		%
		%
		%
		\caseL{$\boldsymbol{x \in (\round{f(a)}_{\Lambda}, \round{f(a')}_{\Lambda})}$}
		%
		Since the output set $(\round{f(a)}_{\Lambda}, \round{f(a')}_{\Lambda})$ is empty when $\Lambda \geq 1$, so we consider the situation where $\Lambda < 1$.
		%
		There are two subcases in this case : $x >0$ and $x < 0$.
		%
		Without loss of generalization, we consider the worst case where error propagate in the same direction, i.e., $\round{f(a')}_{\Lambda}) < 0$. The bounds derived for $l, r$ and $l', r'$ under input $a$ and $a'$ are as follows:
		%
		\\
		%
		For input $a$:
		%
		\\
		%
		$\ubar{R} = e^{\epsilon 
				\big( (f(a) - y_2(1 + \eta)) (1 + \eta)^2) \big)}$, 
		%
		$\bar{R} = e^{\epsilon 
				\frac{(f(a) - \frac{y_2}{1 + \eta})}{(1 + \eta)^2}}$.  
		% 
		\\
		%
		$\ubar{L} = e^{\epsilon 
				\big( (f(a) - y_1(1 + \eta)) (1 + \eta)^2) \big)}$, 
		%
		$\bar{L} = e^{\epsilon 
				\frac{f(a) - \frac{y_1}{1 + \eta}}{(1 + \eta)^2}}$.
		%
		\\
		For input $a'$:
		%
		\\
		% 
		$\ubar{R'} = e^{\epsilon 
				\big( (y_2(1 + \eta) - f(a')) (1 + \eta)^2) \big)}$, 
		%
		$\bar{R'} = e^{\epsilon 
				\frac{ \frac{y_2}{1 + \eta} - f(a') }{(1 + \eta)^2}}$.
		%
		\\
		%
		%
		$\ubar{L'} = e^{\epsilon 
				\big( (y_1(1 + \eta) - f(a') ) (1 + \eta)^2) \big)}$,
		%
		$\bar{L'} = e^{\epsilon 
				\frac{\frac{y_1}{1 + \eta} - f(a') }{(1 + \eta)^2}}$.
		%
		\\
		%
		The bounds on their ratio are as follows:
		%
		\[
		\frac{\ubar{R}}{R} > e^{-5B\eta \epsilon}, 
		~ \frac{\bar{R}}{R} < e^{5B\eta \epsilon};
		~~~~
		\frac{\ubar{R'}}{R'} > e^{-5B\eta \epsilon}, 
		~ \frac{\bar{R'}}{R'} < e^{5B\eta \epsilon}.
		\]
		%
		And the bounds on $|\ubar{R} - \bar{L}|$ and $|\bar{R'} - \ubar{L'}|$ are as follows:
		%
		\[
		|\ubar{R} - \bar{L}| > e^{-12 B\eta \epsilon}|R - L|, 
		~ |\bar{R'} - \ubar{L'}| < e^{11 B\eta \epsilon}|R' - L'|
		\]
		%
		So we have the privacy loss is bounded by:
		%
		\[
		\frac{|\ubar{R} - \bar{L}|}{|\bar{R'} - \ubar{L'}|}
		> \frac{e^{-12 B\eta \epsilon}|R - L|}{e^{11 B\eta \epsilon}|R' - L'|}
		= e^{-23B \eta \epsilon - \epsilon}
		\]
		%
		%
		%
		\caseL{$\boldsymbol{x = \round{f(a')}_{\Lambda}}$}
		%
		%
		%
		This case is symmetric with the case where $\boldsymbol{x = \round{f(a')}_{\Lambda}}$.
		%
		It can also be split into 3 subcases by: $\round{f(a')}_{\Lambda} < 0$, $\round{f(a')}_{\Lambda} = 0$ and $\round{f(a')}_{\Lambda} > 0$. 
		%
		Without loss of generalization, we consider the worst case where the error propagate in the same direction, i.e. $\round{f(a')}_{\Lambda} < 0$.\\
		%
		From this assumption, let $y_1 = x - (\frac{\Lambda}{2})$, $y_2 = x + (\frac{\Lambda}{2})$, we know $y_1 < 0$, $y_2 < 0$.
		%
		Since $f(a) + 1 = f(a')$, we also have $\round{f(a)} < \round{f(a')} <0$.
		So, we know $s$ can only be $-1$ for input $a$ but $s$ can be $1$ or $-1$ for input $a'$.
		%
		\\
		%
		For input $a'$, 
		Let $S = s$, $L' = e^{\epsilon(y_1 - f(a'))}$ and $R' = e^{\epsilon(y_2 - f(a))}$, we have $\forall u \in (L', R')$:
		$\expr_{\snap''} \rbigstep \varx$.
		%
		\\
		%
		Let $l'$ and $r'$ be the range where $\forall u \in (l', r')$ and $S = s$, s.t.
		$\expr_{\snap''} \fbigstep \varx$.
		%
		\\
		%
		We know: $\ubar{L'} < l' < \bar{L'}$, $\ubar{R'} < r < \bar{R'}$ s.t.:
		%
		$$f(a) + \frac{1}{\epsilon} \times \ln (l) \fbigstep y_1
		\land
		f(a) + \frac{1}{\epsilon} \times \ln (r) \fbigstep y_2.$$
		%
		Induction on s, we have:
		When $s = 1$.
		%
		\\
		%
		The transition from $R'$ to $r'$ given the transition environment 
		$\trsenv = [U \mapsto (R'_{+}, (\ubar{R'_{+}}, \bar{R'_{+}})), S \mapsto (1, (1, 1))]$ is shown as following:
		%
		%
		\begin{mathpar}
		\inferrule
		{
		 \trsenv, U
		 \trsto
		 R_+, ( R_+', R_+' )
		}
		{
		 \inferrule
		 {
		  \trsenv, \ln(U)
		  \trsto
		  \big(\ln (R_+), 
		  (\ln(R_+')(1 + \eta), \frac{\ln(R_+')}{1+\eta}) \big)
		 }
		 {
		  \inferrule
		  {
		   \trsenv, \frac{1}{\epsilon}\ln(U)
		   \trsto
		   \frac{1}{\epsilon} \times \ln (R_+), 
		   \big(
		   \frac{1}{\epsilon}\ln(R_+')(1 + \eta)^2, 
		   \frac{1}{\epsilon}\frac{\ln(R_+')}{(1+\eta)^2}
		   \big)
		  }
		  {
		   \inferrule
		   {
		    \trsenv, f(a) + \frac{1}{\epsilon}\ln(U)
			\trsto
			\bigg(
			f(a) + \frac{1}{\epsilon} \times \ln (R_+), 
			\big(
			(f(a) + \frac{1}{\epsilon}\ln(R_+')(1 + \eta)^2)(1 + \eta), 
			(f(a) + \frac{1}{\epsilon}\frac{\ln(R_+')}{(1+\eta)^2}) / (1 + \eta)
			\big) \bigg)
		   }
		   {
		   \trsenv, \snap'(a) \trsto 
		   \trsenv[y \mapsto 
		   (f(a) + \frac{1}{\epsilon} \times \ln (R_+), 
		   	(\err_1, \err_2))]
		   }
		  }
		 }
		}
		\end{mathpar}
		%
		%
		From soundness theorem, we have  $\err_1 \leq y_2 \leq \err_2$. Then we can get following bounds for $r$:\\
		%
		%
		$\ubar{R_+'} = e^{\epsilon 
				\big( (y_2(1 + \eta) - f(a')) (1 + \eta)^2) \big)}$, 
		%
		$\bar{R_+'} = e^{\epsilon 
				\frac{(\frac{y_2}{1 + \eta} - f(a'))}{(1 + \eta)^2}}$.  
		%
		\\
		%
		Since $y_2 = \round{f(a)} + \frac{\Lambda}{2}$, we have 
		$e^{\epsilon 
				\big( (y_2 - f(a))) \big)} > 1$, so actually we know $R_+' = r_+' = 1$.
		%
		% 
		\\
		%
		We can also derive the bound for $l$ in the same way as:\\
		$\ubar{L_+'} = e^{\epsilon 
				\big( (y_1(1 + \eta) - f(a')) (1 + \eta)^2) \big)}$, 
		%
		$\bar{L_+'} = e^{\epsilon 
				\frac{(\frac{y_1}{1 + \eta} - f(a'))}{(1 + \eta)^2}}$.
		%

		% 
		When $s = -1$, we can derive following bounds in the same way for $l$ and $r$:\\
		%
		$\ubar{L_-'} = e^{\epsilon 
				\big( (f(a') - y_2(1 + \eta)) (1 + \eta)^2) \big)}$,
		%
		$\bar{L_-'} = e^{\epsilon 
				\frac{(f(a') - \frac{y_2}{1 + \eta})}{(1 + \eta)^2}}$.\\
		%
		% 
		$\ubar{R_-'} = e^{\epsilon 
				\big( (f(a') - y_1(1 + \eta)) (1 + \eta)^2) \big)}$, 
		%
		$\bar{R_-'} = e^{\epsilon 
				\frac{(f(a') - \frac{y_1}{1 + \eta})}{(1 + \eta)^2}}$.\\
		%
		Since $y_1 = \round{f(a')} - \frac{\Lambda}{2}$, 
		%
		we have $e^{\epsilon \big( (f(a') - y_1)) \big)} > 1$, so actually we know $R_-' = r_-' = 1$.
		%
		\\
		%
		For input $a$, we have only one case where $s = -1$, the following bound can be derived:
		%
		\\
		%
		$\ubar{R} = e^{\epsilon 
				\big( f(a) - (y_2(1 + \eta)) (1 + \eta)^2) \big)}$, 
		%
		$\bar{R} = e^{\epsilon 
				\frac{(f(a) - \frac{y_2}{1 + \eta})}{(1 + \eta)^2}}$.  
		% 
		\\
		%
		$\ubar{L} = e^{\epsilon 
				\big( (f(a) - y_1(1 + \eta)) (1 + \eta)^2) \big)}$, 
		%
		$\bar{L} = e^{\epsilon 
				\frac{f(a) - \frac{y_1}{1 + \eta}}{(1 + \eta)^2}}$.

		% \begin{mathpar}
		% \end{mathpar}

		We have following bounds on their ratios:
		%
		\[
		\frac{\ubar{R_+'}}{R_+'} = e^{\epsilon 
		\big(
		(1 + \eta)^3 y_2 - (1 + \eta)^2f(a) - y_2 + f(a)
		\big)}
		> e^{-3\epsilon B \eta},
		\frac{\bar{R_+'}}{R_+'} = e^{\epsilon 
		\big(
		frac{y_2}{(1 + \eta)^3} - \frac{f(a)}{(1 + \eta)^2} - y_2 + f(a)
		\big)}
		< e^{3\epsilon B \eta},
		\]
		%
		The same bound for $L_+'$ by substituting $y_2$ with $y_1$, and similar bound for $L, R$.
		%
		\[
		\frac{\ubar{R}}{R} = e^{\epsilon 
		\big(
		(1 + \eta)^2f(a) - (1 + \eta)^3 y_2 - f(a) + y_2
		\big)}
		> e^{-2\epsilon B \eta},
		%
		\frac{\bar{R}}{R} = e^{\epsilon 
		\big(
		\frac{f(a)}{(1 + \eta)^2} - frac{y_2}{(1 + \eta)^3} - f(a) + y_2
		\big)}
		< e^{2\epsilon B \eta},
		\]
		%
		Using the bound on their ratios, we can get following bounds on $|\bar{R_-'} - \ubar{L_-'}|$ and $|\ubar{R} - \bar{L}|$:
		%
		\[
		|\bar{R_-'} - \ubar{L_-'}| < e^{3\epsilon B \eta} R - e^{-3\epsilon B \eta}L < (R_-' - L_-') e^{7\epsilon B \eta},
		%
		|\ubar{R} - \bar{L}| > e^{-2\epsilon B \eta} R - e^{2\epsilon B \eta}L > (R - L) e^{-5\epsilon B \eta}
		\]
		%
		Then we have the following bounds on privacy loss:
		%
		\[
		\frac{\ubar{R} - \bar{L}}{2 - (\ubar{L_+'} + \ubar{L_-'})}
		> \frac{\ubar{R} - \bar{L}}{\bar{R_-'} - \ubar{L_-'}}
		> \frac{e^{-5\epsilon B \eta} (R - L)}
		{e^{7\epsilon B \eta}(R_-' - L_-')}
		= e^{-12\epsilon B \eta - \epsilon}
		\]
		%
		%
		%
		\caseL{$\boldsymbol{x \in  (\round{f(a')}_{\Lambda}, B)}$}
		%
		This case can also be split into 3 subcases symmetric with the case where $\boldsymbol{x \in  (-B, \round{f(a)}_{\Lambda})}$:
		%
		%
		%
		\subcaseL{$\boldsymbol{\round{f(a')}_{\Lambda} > 0 \lor \round{f(a')}_{\Lambda} < 0 \land x \in  (0, B)}$}
		%
		%
		let $y_1 = x - \frac{\Lambda}{2}$, $y_2 = x + \frac{\Lambda}{2}$, we have $y_1, y_2 > 0$. The bounds derived for $l, r$ and $l', r'$ under input $a$ and $a'$ in this case are as follows:
		%
		\\
		For input $a'$:
		%
		\\
		% 
		$\ubar{R'} = e^{\epsilon 
				\big( (f(a') - \frac{y_2}{1 + \eta}) (1 + \eta)^2) \big)}$, 
		%
		$\bar{R'} = e^{\epsilon 
				\frac{(f(a') - y_2(1 + \eta))}{(1 + \eta)^2}}$.
		%
		\\
		%
		%
		$\ubar{L'} = e^{\epsilon 
				\big( (f(a') - \frac{y_1}{1 + \eta})) (1 + \eta)^2) \big)}$,
		%
		$\bar{L'} = e^{\epsilon 
				\frac{(f(a') - y_1(1 + \eta)}{(1 + \eta)^2}}$.
		%
		\\
		%
		For input $a$:
		%
		\\
		%
		$\ubar{R} = e^{\epsilon 
				\big( (f(a) - \frac{y_2}{1 + \eta}) (1 + \eta)^2) \big)}$, 
		%
		$\bar{R} = e^{\epsilon 
				\frac{(f(a) - y_2(1 + \eta))}{(1 + \eta)^2}}$.  
		% 
		\\
		%
		$\ubar{L} = e^{\epsilon 
				\big( (f(a) - \frac{y_1}{1 + \eta}) (1 + \eta)^2) \big)}$, 
		%
		$\bar{L} = e^{\epsilon 
				\frac{f(a) - y_1(1 + \eta)}{(1 + \eta)^2}}$.
		%
		%
		The bounds on their ratio are as follows:
		%
		\[
		\frac{\ubar{R}}{R} > e^{-3B\eta \epsilon}, 
		~ \frac{\bar{R}}{R} < e^{3B\eta \epsilon}
		\]
		%
		And the bounds on $|\ubar{R} - \bar{L}|$ and $|\bar{R'} - \ubar{L'}|$ are as follows:
		%
		\[
		|\ubar{R} - \bar{L}| > e^{-7 B\eta \epsilon}|R - L|, 
		~ |\bar{R'} - \ubar{L'}| < e^{7 B\eta \epsilon}|R' - L'|
		\]
		%
		So we have the privacy loss is bounded by:
		%
		\[
		\frac{|\ubar{R} - \bar{L}|}{|\bar{R'} - \ubar{L'}|}
		> \frac{e^{-7 B\eta \epsilon}|R - L|}{e^{7 B\eta \epsilon}|R' - L'|}
		= e^{-14B \eta \epsilon - \epsilon}
		\]
		%
		\subcaseL{$\boldsymbol{\round{f(a')}_{\Lambda} < 0 \land x \in  (\round{f(a')}_{\Lambda}, 0)}$}
		%		
		let $y_1 = x - \frac{\Lambda}{2}$, $y_2 = x - \frac{\Lambda}{2}$, we have $y_1, y_2 < 0$. The bounds derived for $l, r$ in this case are as follows:
		%
		\\
		For input $a'$:
		%
		\\
		% 
		$\ubar{R'} = e^{\epsilon 
				\big( (f(a') - y_2(1 + \eta)) (1 + \eta)^2) \big)}$, 
		%
		$\bar{R'} = e^{\epsilon 
				\frac{(f(a') - \frac{y_2}{1 + \eta})}{(1 + \eta)^2}}$.
		%
		\\
		%
		%
		$\ubar{L'} = e^{\epsilon 
				\big( (f(a') - y_1(1 + \eta) ) (1 + \eta)^2) \big)}$,
		%
		$\bar{L'} = e^{\epsilon 
				\frac{(f(a') - \frac{y_1}{1 + \eta}}{(1 + \eta)^2}}$.
		%
		\\
		%
		For input $a$:
		%
		\\
		%
		$\ubar{R} = e^{\epsilon 
				\big( (f(a) - y_2(1 + \eta)) (1 + \eta)^2) \big)}$, 
		%
		$\bar{R} = e^{\epsilon 
				\frac{(f(a) - \frac{y_2}{1 + \eta})}{(1 + \eta)^2}}$.  
		% 
		\\
		%
		$\ubar{L} = e^{\epsilon 
				\big( (f(a) - y_1(1 + \eta)) (1 + \eta)^2) \big)}$, 
		%
		$\bar{L} = e^{\epsilon 
				\frac{f(a) - \frac{y_1}{1 + \eta}}{(1 + \eta)^2}}$.
		%
		\\
		%
		The bounds on their ratio are as follows:
		%
		\[
		\frac{\ubar{R}}{R} > e^{-5B\eta \epsilon}, 
		~ \frac{\bar{R}}{R} < e^{5B\eta \epsilon}
		\]
		%
		And the bounds on $|\ubar{R} - \bar{L}|$ and $|\bar{R'} - \ubar{L'}|$ are as follows:
		%
		\[
		|\ubar{R} - \bar{L}| > e^{-12 B\eta \epsilon}|R - L|, 
		~ |\bar{R'} - \ubar{L'}| < e^{11 B\eta \epsilon}|R' - L'|
		\]
		%
		So we have the privacy loss is bounded by:
		%
		\[
		\frac{|\ubar{R} - \bar{L}|}{|\bar{R'} - \ubar{L'}|}
		> \frac{e^{-12 B\eta \epsilon}|R - L|}{e^{11 B\eta \epsilon}|R' - L'|}
		= e^{-23B \eta \epsilon - \epsilon}
		\]
		%
		%
		\subcaseL{$\boldsymbol{\round{f(a')}_{\Lambda} < 0 \land x = 0}$}
		%
		let $y_1 = x - \frac{\Lambda}{2}$, $y_2 = x - \frac{\Lambda}{2}$, we have $y_1 < 0$ and $y_2 > 0$. The bounds derived for $l, r$ in this case are as follows:
		%
		\\
		For input $a'$:
		%
		\\
		% 
		$\ubar{R'} = e^{\epsilon 
				\big( (f(a') - \frac{y_2}{1 + \eta}) (1 + \eta)^2) \big)}$, 
		%
		$\bar{R'} = e^{\epsilon 
				\frac{(f(a') - y_2(1 + \eta))}{(1 + \eta)^2}}$.
		%
		\\
		%
		%
		$\ubar{L'} = e^{\epsilon 
				\big( (f(a') - y_1(1 + \eta)) (1 + \eta)^2) \big)}$, 
		%
		$\bar{L'} = e^{\epsilon 
				\frac{f(a') - \frac{y_1}{1 + \eta}}{(1 + \eta)^2}}$.
		%
		\\
		%
		For input $a$:
		%
		\\
		%
		$\ubar{R} = e^{\epsilon 
				\big( (f(a) - \frac{y_2}{1 + \eta}) (1 + \eta)^2) \big)}$, 
		%
		$\bar{R} = e^{\epsilon 
				\frac{(f(a) - y_2(1 + \eta))}{(1 + \eta)^2}}$.  
		% 
		\\
		%
		$\ubar{L} = e^{\epsilon 
				\big( (f(a) - y_1(1 + \eta)) (1 + \eta)^2) \big)}$, 
		%
		$\bar{L} = e^{\epsilon 
				\frac{f(a) - \frac{y_1}{1 + \eta}}{(1 + \eta)^2}}$.
		%
		\\
		%
		The bounds on their ratio are as follows:
		%
		\[
		\frac{\ubar{R}}{R} > e^{-3B\eta \epsilon}, 
		~ \frac{\bar{R}}{R} < e^{3B\eta \epsilon}
		\frac{\ubar{L}}{L} > e^{-5B\eta \epsilon}, 
		~ \frac{\bar{L}}{L} < e^{5B\eta \epsilon}
		\]
		%
		And the bounds on $|\ubar{R} - \bar{L}|$ and $|\bar{R'} - \ubar{L'}|$ are as follows:
		%
		\[
		|\ubar{R} - \bar{L}| > e^{-8 B\eta \epsilon}|R - L|, 
		~ |\bar{R'} - \ubar{L'}| < e^{8 B\eta \epsilon}|R' - L'|
		\]
		%
		So we have the privacy loss is bounded by:
		%
		\[
		\frac{|\ubar{R} - \bar{L}|}{|\bar{R'} - \ubar{L'}|}
		> \frac{e^{-8 B\eta \epsilon}|R - L|}{e^{8 B\eta \epsilon}|R' - L'|}
		= e^{-16B \eta \epsilon - \epsilon}
		\]
		%
		%
		%
		\caseL{$\boldsymbol{x = B}$}
		%
		%
		We know $s = -1$, $L = l = 0$ and $R = b$, so we only need to estimate the right side range $r$ in this case. The bounds derived for $r, r'$ are as following:
		\[
		\begin{array}{c}
		\ubar{R} = e^{\epsilon 
		\big( (f(a) -  \frac{x}{1 + \eta}) (1 + \eta)^2) \big)},
		%
		\bar{R} = e^{\epsilon 
		\frac{(f(a) - x(1 + \eta))}{(1 + \eta)^2}}
		\\
		%
		\ubar{R'} = e^{\epsilon 
		\big( (f(a') -  \frac{x}{1 + \eta}) (1 + \eta)^2) \big)},
		%
		\bar{R'} = e^{\epsilon 
		\frac{(f(a') - x(1 + \eta))}{(1 + \eta)^2}}
		\end{array}
		\]
		%		
		The privacy loss of $\snap(a)$ in this case is bounded by:
		%
		\[
		\begin{array}{ll}
		\frac
		{\frac{1}{2}(\ubar{R} - 0)}
		{\frac{1}{2}(\bar{R'} - 0)}
		& = e^{\epsilon
		\bigg(
		\big( (f(a) -  \frac{x}{1 + \eta}) (1 + \eta)^2) \big)
		-
		\frac{(f(a') - x(1 + \eta))}{(1 + \eta)^2}
		\bigg)}\\
		& = e^{\epsilon
		\bigg(
		f(a)(1 + \eta)^2 - x(1 + \eta) 
		- \frac{f(a)}{(1 + \eta)^2} + \frac{x}{(1 + \eta)}   
		\bigg)} ~ (\star)
		\end{array}
		\]
		%
		Since $ 1 + 2.1\eta > (1 + \eta)^2 > 1 + 2\eta$ and $\frac{1}{(1 + \eta)^2} > 1 - 2 \eta$, we have:
		%
		\[
		\begin{array}{ll}
		(\star) & > e^{\epsilon \big(
		(1 + 2\eta) f(a) - \frac{\eta(\eta + 2)}{1 + \eta} x
		- \frac{1}{1 + 2\eta}(f(a) + 1)
		\big)}\\
		%
		& = e^{\epsilon\big(
		\frac{4\eta(\eta + 1)}{1 + 2\eta} f(a) 
		- \frac{\eta(\eta + 2)}{1 + \eta} x
		- \frac{1}{1 + 2\eta}		
		\big)}\\
		%
		& > e^{\epsilon\big( -B \eta
		\frac{4(\eta + 1)}{1 + 2\eta} + \frac{(\eta + 2)}{1 + \eta} x - 1	
		\big)}\\
%
		& > e^{\epsilon(-6 \eta B - 1)}
		\end{array}
		\]
	%	
	\end{itemize}



\end{proof}

\newpage
\section{Syntax - Functional}
Following are the syntax of our system:
%
\[\begin{array}{llll}
\mbox{Expr.} & \expr & ::= & \varx 
	~|~ \rval ~|~ \fval
	%
	~|~ F(\data) ~|~ \expr \bop \expr
	%
	~|~ \uop (\expr) ~|~ \elet ~ \varx \samplel \edistr ~ \ein ~ \expr
	~|~ \elet ~ \varx = \expr_1 ~ \ein ~ \expr_2 \\
%
\mbox{Binary Operation} & \bop & ::= & + ~|~ - ~|~ \times ~|~ \div \\
%
\mbox{Unary Operation} & \uop & ::= & \ln ~|~ - ~|~ \round{\cdot} 
	%
	~|~ \clamp_B(\cdot) \\
%
\mbox{Value} & \valv & ::= & \rval ~|~  \fval \\
%
\mbox{Distribution} & \edistr & ::= & \uniform ~|~ \bernoulli \\ 
%
\mbox{Error} & \err & ::= & (\expr, \expr)
%
\end{array}
\]


\newpage
\section{Semantics - Functional}

The transition semantics with relative floating point computation error are shown in Figure. \ref{fig_func_trans_semantics_exp}. The semantics are $\expr \trsto (\err)$, which means a real expression $\expr$ can be transited in floating point computation with error bound $\err$, $\eta$ is the machine epsilon.

We assume the SAMPLE and F(D) semantics for floating point and real computation are the same. $\edistr \bigstep_{\$} \valv$ represents $\valv$ is sampled from the distribution $\edistr$.

\begin{figure}
\begin{mathpar}
\inferrule*[right = val]
{
	\rval \geq 0
}
{
	\rval
	\trsto
	\big( \frac{\rval}{(1 + \eta)}, \rval(1 + \eta) \big)
}
%
\and
%
\inferrule*[right = val-neg]
{
	\rval < 0
}
{
	\rval
	\trsto
	\big( \rval(1 + \eta), \frac{\rval}{(1 + \eta)} \big)
}
%
\and
%
\inferrule*[right = val-eq]
{
	\rval = \floaten(\rval)
}
{
	\rval
	\trsto
	( \rval, \rval )
}
%
\and
%
\inferrule*[right = f(d)]
{
	\empty
}
{
	f(D)
	\trsto
	( f(D), f(D) )
}
%
\and
%
\inferrule*[right = sample]
{
	\empty
}
{
	\samplel \edistr
	\trsto
	( \samplel \edistr, \samplel \edistr )
}
%
\and
%
\inferrule*[right = bop]
{
	\expr^1 \trsto (\ubar{\rval^1}, \bar{\rval^1})
	\and
	\expr^2 \trsto (\ubar{\rval^2}, \bar{\rval^2})
	\and
	\expr^1 * \expr^2 \geq 0
}
{
    \expr^1 * \expr^2
    \trsto 
    \big(
    \frac{\ubar{\rval^1} * \ubar{\rval^2}}{(1 + \eta)}, 
    (\bar{\rval^1} * \bar{\rval^2})(1 + \eta)
    \big)
}
%
\inferrule*[right = bop-neg]
{
	\expr^1 \trsto  (\ubar{\rval^1}, \bar{\rval^1})
	\and
	\expr^2 \trsto  (\ubar{\rval^2}, \bar{\rval^2})
	\and
	\expr^1 * \expr^2 < 0
}
{
    \expr^1 * \expr^2
    \trsto 
    \big(
    (\bar{\rval^1} * \bar{\rval^2})(1 + \eta),
    \frac{\ubar{\rval^1} * \ubar{\rval^2}}{(1 + \eta)}
    \big)
}%
\and
%
\inferrule*[right = uop]
{
	\expr \trsto (\ubar{\rval}, \bar{\rval})
	\and
	\uop (\expr) \geq 0
}
{
    \uop(\expr)
    \trsto 
    \big(
    \frac{\uop(\ubar{\rval})}{(1 + \eta)}, 
    (\uop(\bar{\rval}))(1 + \eta)
    \big)
}
%
\inferrule*[right = uop-neg]
{
	\expr \trsto  (\ubar{\rval}, \bar{\rval})
	\and
	\uop (\expr) < 0
}
{
    \uop(\expr)
    \trsto 
    \big(
    (\uop(\ubar{\rval}))(1 + \eta),
    \frac{\uop(\bar{\rval})}{(1 + \eta)}
    \big)
}
\end{mathpar}
\caption{Semantics of Transition for Expressions with Relative Floating Point Error}
\label{fig_func_trans_semantics_exp}
\end{figure}



\begin{figure}
\begin{mathpar}
\inferrule*[right = fval]
{
	\floaten(\rval) = \fval
}
{
	\rval
	\bigstep
	\fval
}
%
\and
%
\inferrule*[right = fbop]
{
	\expr^1 \bigstep \fval^1
	\and
	\expr^2 \bigstep \fval^2
	\and
	\floaten(\fval^1 \bop \fval^2) = \fval
}
{
    \expr^1 \bop \expr^2 \bigstep \fval
}
%
\and
%
\inferrule*[right = fuop]
{
	\expr \bigstep \fval',
	\and
	\floaten(\uop (\fval')) = \fval
}
{
    \uop(\expr) \bigstep \fval
}
\end{mathpar}
\caption{Semantics of Evaluation in Floating Point Computation}
\label{fig_fl_semantics_exp}
\end{figure}

\begin{figure}
\begin{mathpar}
\inferrule*[right = rval]
{
	\empty
}
{
	\rval
	\bigstep
	\rval
}
%
\and
%
\inferrule*[right = rbop]
{
	\expr^1 \bigstep \rval^1
	\and
	\expr^2 \bigstep \rval^2
	\and
	\rval^1 \bop \rval^2 = \rval
}
{
    \expr^1 \bop \expr^2 \bigstep \rval
}
%
\and
%
\inferrule*[right = ruop]
{
	\expr \bigstep \rval',
	\and
	\uop (\rval') = \rval
}
{
    \uop(\expr) \bigstep \rval
}
%
\and
%
\inferrule*[right = sample]
{
	 \fval \leftarrow \edistr^{\diamond}
}
{
	\samplel \edistr \bigstep \fval
}
%
\and
%
\inferrule*[right = f(d)]
{
	f(D) = \fval
}
{
	f(D) \bigstep \fval
}
\end{mathpar}
\caption{Semantics of Evaluation in Real Computation}
\label{fig_func_real_semantics_exp}
\end{figure}


\begin{thm}[Soundness Theorem]
Given $\expr$ where the transition 
$\expr \trsto (\ubar{\rval}, \bar{\rval})$ holds, 
then if $\expr$ evaluates to $c$ in floating point computation and 
$\ubar{\rval}$ and $\bar{\rval}$ evaluates to $\ubar{r}$ and $\bar{r}$ in real computation, we have: 
\[
\ubar{r} \leq c \leq \bar{r}
% ~~ \land ~~ 
% \frac{r^1}{r^2} \leq \frac{c}{r} \leq \frac{r^2}{r^1}.
\]
\end{thm}
% \begin{proof}
% Induction on $\expr$, we have following cases:

% \end{proof}

\newpage
\bibliographystyle{plain}
\bibliography{verifysnap.bib}



\end{document}















