\documentclass[a4paper,11pt]{article}
\usepackage[utf8]{inputenc}
%

\usepackage[utf8]{inputenc}%Packages
\usepackage[T1]{fontenc}
\usepackage{fourier} 
\usepackage[english]{babel} 
\usepackage{amsmath,amsfonts,amsthm} 
\usepackage{lscape}
\usepackage{geometry}
\usepackage{amsmath}
\usepackage{algorithm}
\usepackage{algorithmic}
\usepackage{amssymb}
\usepackage{amsfonts}
\usepackage{times}
\usepackage{bm}
\usepackage{mathtools}
\usepackage{ stmaryrd }
\usepackage{ amssymb }
\usepackage{ textcomp }
\usepackage[normalem]{ulem}
% For derivation rules
\usepackage{mathpartir}
\usepackage{color}
\usepackage{a4wide}

\usepackage{stmaryrd}
\SetSymbolFont{stmry}{bold}{U}{stmry}{m}{n}

\newcommand{\distr}{\mathsf{Distr}}
\newcommand{\uniform}{\mathsf{unif}}
\newcommand{\pdf}{\mathsf{pdf}}
\newcommand{\snap}{\mathsf{Snap}}
\newcommand{\fsnap}{\mathsf{Snap}_{\mathbb{F}}}
\newcommand{\rsnap}{\mathsf{Snap}_{\mathbb{R}}}


\newcommand{\pr}[2]{\underset{#1}{\mathsf{Pr}}[#2]}
\newcommand{\projl}{\pi_1}
\newcommand{\projr}{\pi_2}
\newcommand{\supp}{\mathsf{supp}}
\newcommand{\clamp}{\mathsf{clamp}}
\newcommand{\real}{\mathbb{R}}
\newcommand{\samplel}{\xleftarrow{\$}}
\newcommand{\psup}{\mathsf{Sup}}
\newcommand{\sign}{\mathsf{sign}}

\newcommand{\lapmech}{\mathcal{L}}
\newcommand{\laplace}{\mathsf{laplce}}
\newcommand{\round}[1]{\lfloor #1 \rceil}


%for syntax:

%for programs:
\newcommand{\prog}{p}
\newcommand{\fprog}{p_{\mathbb{F}}}
\newcommand{\rprog}{p_{\mathbb{R}}}
\newcommand{\ret}{\mathsf{return}}



%expression
\newcommand{\expr}{e}
\newcommand{\fexpr}{\expr_{\mathbb{F}}}
\newcommand{\rexpr}{\expr_{\mathbb{R}}}

\newcommand{\elet}{\kw{let}}

\newcommand{\ein}{\kw{in}}

%for smaples:
\newcommand{\bernoulli}{\kw{bernoulli}}

%values
\newcommand{\fval}{c}
\newcommand{\rval}{r}
\newcommand{\valv}{v}
\newcommand{\data}{D}

%variables
\newcommand{\varx}{x}

\newcommand{\fvarx}{x}
\newcommand{\rvarx}{X}


\newcommand{\term}{t}
\newcommand{\etrue}{\kw{true}}
\newcommand{\efalse}{\kw{false}}
% \newcommand{\eflconst}{c}
% \newcommand{\erlconst}{r}
\newcommand{\precision}{\eta}
\newcommand{\floaten}{\kw{fl}}

\newcommand{\err}{err}
\newcommand{\condition}{\Phi}
\newcommand{\edistr}{\mu}

\newcommand{\fbigstep}{\Downarrow^{\mathbb{F}}}
\newcommand{\rbigstep}{\Downarrow^{\mathbb{R}}}

\newcommand{\bigstep}{\Downarrow}
\newcommand{\trsto}{\Rightarrow}


%for environments
\newcommand{\trsenv}{\Theta}

\newcommand{\evlenv}{\Gamma}

\newcommand{\fevlenv}{\Gamma^{\mathbb{F}}}

\newcommand{\revlenv}{\Gamma^{\mathbb{R}}}



\usepackage{stackengine} 

% For Operations
%binary operations
\newcommand{\bop}{*}
\newcommand{\obop}{\stackMath\mathbin{\stackinset{c}{0ex}{c}{0ex}{\text{\footnotesize{$\bop$}}}{\bigcirc}}}

\newcommand{\oexp}{\stackMath\mathbin{\stackinset{c}{0ex}{c}{0ex}{\text{\footnotesize{$\mathsf{e}$}}}{\bigcirc}}}

\newcommand{\oln}{\stackMath\mathbin{\stackinset{c}{0ex}{c}{0ex}{\text{\footnotesize{$\mathsf{ln}$}}}{\bigcirc}}}

\newcommand{\odiv}{\stackMath\mathbin{\stackinset{c}{0ex}{c}{0ex}{\text{\footnotesize{$\div$}}}{\bigcirc}}}
\newcommand{\ubar}[1]{\text{\b{$#1$}}}

%unary operations
\newcommand{\uop}{\circ}
\newcommand{\ouop}{\stackMath\mathbin{\stackinset{c}{0ex}{c}{0ex}{\text{\footnotesize{$\uop$}}}{\bigcirc}}}





\newcommand{\diam}{{\color{red}\diamond}}
\newcommand{\dagg}{{\color{blue}\dagger}}
\let\oldstar\star
\renewcommand{\star}{\oldstar}

\newcommand{\im}[1]{\ensuremath{#1}}

\newcommand{\kw}[1]{\im{\mathtt{#1}}}


\newcommand{\set}[1]{\im{\{{#1}\}}}

\newcommand{\mmax}{\ensuremath{\mathsf{max}}}

%%%%%%%%%%%%%%%%%%%%%%%%%%%%%%%%%%%%%%%%%%%%%%%%%%%%%%%%
% Comments
\newcommand{\omitthis}[1]{}

% Misc.
\newcommand{\etal}{\textit{et al.}}
\newcommand{\bump}{\hspace{3.5pt}}

% Text fonts
\newcommand{\tbf}[1]{\textbf{#1}}
%\newcommand{\trm}[1]{\textrm{#1}}

% Math fonts
\newcommand{\mbb}[1]{\mathbb{#1}}
\newcommand{\mbf}[1]{\mathbf{#1}}
\newcommand{\mrm}[1]{\mathrm{#1}}
\newcommand{\mtt}[1]{\mathtt{#1}}
\newcommand{\mcal}[1]{\mathcal{#1}}
\newcommand{\mfrak}[1]{\mathfrak{#1}}
\newcommand{\msf}[1]{\mathsf{#1}}
\newcommand{\mscr}[1]{\mathscr{#1}}

% Text mode
\newenvironment{nop}{}{}

% Math mode
\newenvironment{sdisplaymath}{
\begin{nop}\small\begin{displaymath}}{
\end{displaymath}\end{nop}\ignorespacesafterend}
\newenvironment{fdisplaymath}{
\begin{nop}\footnotesize\begin{displaymath}}{
\end{displaymath}\end{nop}\ignorespacesafterend}
\newenvironment{smathpar}{
\begin{nop}\small\begin{mathpar}}{
\end{mathpar}\end{nop}\ignorespacesafterend}
\newenvironment{fmathpar}{
\begin{nop}\footnotesize\begin{mathpar}}{
\end{mathpar}\end{nop}\ignorespacesafterend}
\newenvironment{alignS}{
\begin{nop}\begin{align}}{
\end{align}\end{nop}\ignorespacesafterend}
\newenvironment{salignS}{
\begin{nop}\small\begin{align}}{
\end{align}\end{nop}\ignorespacesafterend}
\newenvironment{falignS}{
\begin{nop}\footnotesize\begin{align*}}{
\end{align}\end{nop}\ignorespacesafterend}

% Stack formatting
\newenvironment{stackAux}[2]{%
\setlength{\arraycolsep}{0pt}
\begin{array}[#1]{#2}}{
\end{array}}
\newenvironment{stackCC}{
\begin{stackAux}{c}{c}}{\end{stackAux}}
\newenvironment{stackCL}{
\begin{stackAux}{c}{l}}{\end{stackAux}}
\newenvironment{stackTL}{
\begin{stackAux}{t}{l}}{\end{stackAux}}
\newenvironment{stackTR}{
\begin{stackAux}{t}{r}}{\end{stackAux}}
\newenvironment{stackBC}{
\begin{stackAux}{b}{c}}{\end{stackAux}}
\newenvironment{stackBL}{
\begin{stackAux}{b}{l}}{\end{stackAux}}

%APPENDIX
\newcommand{\caseL}[1]{\item[\textbf{case}] \textbf{#1}\newline}
\newcommand{\subcaseL}[1]{\item[\textbf{subcase}] \textbf{#1}\newline}

\newcommand{\todo}[1]{{\footnotesize \color{red}\textbf{[[ #1 ]]}}}


%% \makeatletter
%% \newcommand\definitionname{Lemma}
%% \newcommand\listdefinitionname{Proofs of Lemmas and Theorems}
%% \newcommand\listofdefinitions{%
%%   \section*{\listdefinitionname}\@starttoc{def}}
%% \makeatother



\newtheoremstyle{athm}{\topsep}{\topsep}%
      {\upshape}%         Body font
      {}%         Indent amount (empty = no indent, \parindent = para indent)
      {\bfseries}% Thm head font
      {}%        Punctuation after thm head
      {.8em}%     Space after thm head (\newline = linebreak)
      {\thmname{#1}\thmnumber{ #2}\thmnote{~\,(#3)}
% \addcontentsline{Lemma}{Lemma}
%   {\protect\numberline{\thechapter.\thelemma}#1}
      % \ifstrempty{#3}%
      {\addcontentsline{def}{section}{#1~#2~#3}}%
      % {\addcontentsline{def}{subsection}{\theathm~#3}}
\newline}%         Thm head spec

 \theoremstyle{athm}


% \newtheoremstyle{break}
%   {\topsep}{\topsep}%
%   {\itshape}{}%
%   {\bfseries}{}%
%   {\newline}{}%
% \theoremstyle{break}

%There are some problems with llncs documentcalss, so commenting these out until i find a solution
\newtheorem{thm}{Theorem}

%\spnewtheorem{thm1}[theorem]{Theorem}{\bfseries}{\upshape}
%\newenvironment{Theorem}[1][]{\begin{thm1}\iffirstargument[#1]\fi\quad\\}{\end{thm1}}

 \newtheorem{lem}[thm]{Lemma}
 \newtheorem{conjec}{Conjecture}
 \newtheorem{corr}[thm]{Corollary}
 \newtheorem{defn}{Definition}
 \newtheorem{prop}[thm]{Proposition}
 \newtheorem{assm}[thm]{Assumption}

\newtheorem{Eg}[thm]{Example}
\newtheorem{hypothesis}[thm]{Hypothesis}
\newtheorem{motivation}{Motivation}

% BNF symbols
\newcommand{\bnfalt}{{\bf \,\,\mid\,\,}}
\newcommand{\bnfdef}{{\bf ::=~}}

%% Highlighting
\newcommand{\hlm}[1]{\mbox{\hl{$#1$}}}

%% Provenance modes
\newcommand{\modifrcationProvenance}{{\bf MP}}
\newcommand{\updateProvenance}{{\bf UP}}

%Lemmas
\newcommand{\lemref}[1]{Lemma \ref{#1}} %name and number
\newcommand{\thmref}[1]{Theorem \ref{#1}} %name and number

\renewcommand{\labelenumii}{\theenumii}
\renewcommand{\theenumii}{\theenumi.\arabic{enumii}.}

\usepackage{enumitem}
\setenumerate{listparindent=\parindent}

\newlist{enumih}{enumerate}{3}
\setlist[enumih]{label=\alph*),before=\raggedright, topsep=1ex, parsep=0pt,  itemsep=1pt }

\newlist{enumconc}{enumerate}{3}
\setlist[enumconc]{leftmargin=0.5cm, label*= \arabic*.  , topsep=1ex, parsep=0pt,  itemsep=3pt }

\newlist{enumsub}{enumerate}{3}
\setlist[enumsub]{ leftmargin=0.7cm, label*= \textbf{subcase} \bf \arabic*: }

\newlist{enumsubsub}{enumerate}{3}
\setlist[enumsubsub]{ leftmargin=0.5cm, label*= \textbf{subsubcase} \bf \arabic*: }

\newlist{mainitem}{itemize}{3}
\setlist[mainitem]{ leftmargin=0cm , label= {\bf Case} }


\newenvironment{subproof}[1][\proofname]{%
  \renewcommand{\qedsymbol}{$\blacksquare$}%
  \begin{proof}[#1]%
}{%
  \end{proof}%
}


\newenvironment{nstabbing}
  {\setlength{\topsep}{0pt}%
   \setlength{\partopsep}{0pt}%
   \tabbing}
  {\endtabbing} 





%%% Local Variables:
%%% mode: latex
%%% TeX-master: "main"
%%% End:

\usepackage{eucal}
\usepackage{url}

\begin{document}

\title{Verifying Snapping Mechanism}

\maketitle
In order to verify the differential privacy proeprty of snapping mechanism, we follow the logic rules designed from \cite{barthe2016proving}.

Some new rules are added into this logic in Figure \ref{logic_rule} following with correctness proof. Then we formalized the snapping mechanism and verified its differential privacy property under these logic rules.

\section{Logic Rules}
\begin{defn}
[Laplce mechanism \cite{dwork2006calibrating}]
Let $\epsilon > 0$. The Laplace mechanism  $\lapmech_{\epsilon}$: $\real \to \distr(\real)$ is defined by $\lapmech(t) = t + v$, where $v \in \real$ is drawn from the Laplace distribution $\laplace(\frac{1}{\epsilon})$.
\end{defn}

\begin{defn}
Let $\epsilon \leq 0$. The $\epsilon${\text -DP divergence} $\Delta_{\epsilon}(\mu_1, \mu_2)$ between two sub-distributions $\mu_1 \in \distr(U)$, $\mu_2 \in \distr(U)$ is defined as:
\[	
	\underset{E \in U}{sup}\Big(\pr{x \leftarrow \mu_1}{x \in E} - \exp(\epsilon) \pr{x \leftarrow \mu_2}{x \in  E}]\Big)
\]

\end{defn}

\begin{defn}
[$(\epsilon, \delta)$ - lifting \cite{barthe2016proving}]
Two sub-distributions $\mu_1 \in \distr(U_1)$, $\mu_2 \in \distr(U_2)$are related by the $(\epsilon, \delta)$ - dilation lifting of $\Psi \subseteq U_1 \times U_2$, written $\mu_1 \Psi^{\#(\epsilon, \delta)} \mu_2$, if there exist two witness sub-distributions $\mu_L \in \distr(U_1 \times U_2)$ and $\mu_R \in \distr(U_1, U_2)$ s.t.:
\begin{enumerate}
	\item $\projl(\mu_L) = \mu_1$ and $\projr(\mu_R) = \mu_2$;
	\item $\supp(\mu_L) \subseteq \Psi$ and $\supp(\mu_R) \subseteq \Psi$; and
	\item $\Delta_{\epsilon}(\mu_L, \mu_R) \leq \delta$.
\end{enumerate}
\end{defn}

The logic rules we are using in our work is presented in Figure \ref{logic_rule}. The correctness of rules is proved in Theorem \ref{corr_axunif} and Theorem \ref{corr_axnull}

\begin{figure}
\begin{mathpar}
\inferrule*[right = AxUnif]
{
}
{
	\vdash 
	u_1 \samplel \mu 
	\sim_{\epsilon, 0} 
	u_2 \samplel \mu 
	: \top \Rightarrow  e^{-\epsilon} u_2 \leq u_1 \leq e^{\epsilon} u_2
}
\and
\inferrule*[right = LapGen]
{
}
{
	\vdash 
	y_1 \samplel \lapmech_{\epsilon}(e_1) 
	\sim_{k' \cdot \epsilon, 0} 
	y_2 \samplel \lapmech_{\epsilon}(e_2)
	: | k + e_1 - e_2| \leq k'  \Rightarrow  y_1 + k = y_2
}
\and
\inferrule*[right = LapNull]
{
}
{
	\vdash 
	y_1 \samplel \lapmech_{\epsilon}(e_1) 
	\sim_{0, 0} 
	y_2 \samplel \lapmech_{\epsilon}(e_2)
	: \top  \Rightarrow  y_1 - y_2 = e_1 - e_2
}
\and
\inferrule*[right = AxNull]
{
}
{
	\vdash 
	y_1 \samplel  \mu
	\sim_{0, 0} 
	y_2 \samplel \mu
	: \top  \Rightarrow  y_1 = y_2
}
\and
\inferrule*[right = Comp]
{
p_1 \sim_{k, 0} p_2 : \Phi_1 \Rightarrow \Phi'_1
\\
c_1 \sim_{k', 0} c_2 : \Phi'_1 \Rightarrow \Phi_2
}
{
	\vdash 
	p_1; c_1  
	\sim_{k + k', 0} 
	p_2; c_2
	: \Phi_1  \Rightarrow  \Phi_2
}
\end{mathpar}
\caption{Logic Rules from \cite{barthe2016proving}}
\label{logic_rule}
\end{figure}


\begin{thm}
\label{corr_axunif}
Let $\mu_1 \in \distr(\real)$, $\mu_2 \in \distr(\real)$ are defined:
\[
	{\mu_1}(x) = \uniform(x)
\]
\[
	{\mu_2}(y) = {\uniform}(y)
\]
where $\uniform$ is uniform distribution over $[0, 1)$ whoes $\pdf.$ is defined as:
\[
	\pdf_{\uniform}(x) = 
	\begin{cases}
	1 & x \in [0, 1)\\
	0       & o.w.
	\end{cases}.
\]
Then, $\mu_1 \Psi^{\#(\epsilon, 0)} \mu_2$, where
\[
	\Psi = \{(x,y) \in \real \times \real | x \cdot e^{-\epsilon} \leq y \leq x \cdot e^{\epsilon} \}
\]
\end{thm}

\begin{thm}
\label{corr_axnull}
For any distributions $\mu_1 \in \distr(\real)$, $\mu_2 \in \distr(\real)$, $\mu_1 \Psi^{\#(0, 0)} \mu_2$, where
\[
	\Psi = \{(x,y) \in \real \times \real | x = y \}
\]
\end{thm}


\begin{proof}[Proof of Theorem \ref{corr_axunif}]

Existing $\mu_L, \mu_R \in \distr(\real \times \real)$:
\[
	{\mu_L}(x, y) = 
	\begin{cases}
	{\uniform}(x) & x \cdot e^{-\epsilon} = y \land x \in [0, 1)\\
	0       & o.w.
	\end{cases}
	\\
	{\mu_R}(x, y) = 
	\begin{cases}
	{\uniform}(y) & x \cdot e^{-\epsilon} = y \land y \in [0, 1)\\
	0       & o.w.
	\end{cases}.
\]


Their $\pdf.$ are defined:
\[
	\pdf_{\mu_L}(x, y) = 
	\begin{cases}
	\pdf_{\uniform}(x) & x \cdot e^{-\epsilon} = y \land x \in [0, 1)\\
	0       & o.w.
	\end{cases}
\]
\[
	\pdf_{\mu_R}(x, y) = 
	\begin{cases}
	\pdf_{\uniform}(y) & x \cdot e^{-\epsilon} = y \land y \in [0, 1)\\
	0       & o.w.
	\end{cases}.
\]
\begin{itemize}
	\item $\supp(\mu_L) \in \Psi \land \supp(\mu_R) \in \Psi$

	\begin{itemize}
		\item $\supp(\mu_L) \subseteq \Psi$ 

		By definition of the $\pdf$ of $\mu_L$, we have: $\pr{(x,y) \samplel \mu_L}{(x,y) \notin \Psi} = 0$.

		Then we can derive $\supp(\mu_L) \in \Psi$

		\item $\supp(\mu_R) \subseteq \Psi$

		By definition of the $\pdf$ of $\mu_R$, we have: $\pr{(x,y) \samplel \mu_R}{(x,y) \notin \Psi} = 0$.

		Then we can derive $\supp(\mu_L) \in \Psi$

	\end{itemize}		


	\item $\projl(\mu_L) = \mu_1 \land \pi_2(\mu_R) = \mu_2$
	
	\begin{itemize}
		\item $\projl(\mu_L) = \mu_1$ 

		% Equivalent to show $\pdf_{\projl(\mu_L)}  = \pdf_{\mu_1}$.

		By definition of the $\projl$ and $\pdf$ of $\mu_L$, we have $\forall x \in \real$:
		\[
			\pdf_{\projl(\mu_L)}(x) = 
			\begin{cases}
			\int_{y}\pdf_{\uniform}(x) & (x,y) \in \Psi \land x \in [0, 1)\\
			0       & o.w.
			\end{cases} 
			= 
			\begin{cases}
			\pdf_{\uniform}(x) & x \in [0, 1)\\
			0       & o.w.
			\end{cases}
			=
			\pdf_{\mu_1}(x)
		\]

		\item $\projl(\mu_R) = \mu_2$ 

		Equivalent to show$\pdf_{\projr(\mu_R)}  = \pdf_{\mu_2}$.

		By definition of the $\projr$ and $\pdf$ of $\mu_R$, we have $\forall y \in \real$:
		\[
			\pdf_{\projr(\mu_R)}(y) = 
			\begin{cases}
			\int_{x}\pdf_{\uniform}(y) & (x,y) \in \Psi \land y \in [0, 1)\\
			0       & o.w.
			\end{cases} 
			= 
			\begin{cases}
			\pdf_{\uniform}(y) & y \in [0, 1)\\
			0       & o.w.
			\end{cases}
			=
			\pdf_{\mu_2}(y)
		\]
	\end{itemize}	

	\item $\Delta_{\epsilon}(\mu_L, \mu_R) \leq 0$

	By definition of $\epsilon-$DP divergence, we have:
	 \[
	 \begin{array}{ll}
	 \Delta_{\epsilon}(\mu_L, \mu_R) 
	 & = \underset{S}{\psup}
	 \Big(
	 \pr{(x,y) \samplel \mu_L}{(x,y) \in S} - e^{\epsilon} \pr{(x,y) \samplel \mu_R}{(x,y) \in S}
	 \Big) \\
	 & =\underset{S}{\psup}
	 \Big(
	 \int_{(x,y) \in S} \pdf_{\mu_L}(x, y) - e^{\epsilon} \int_{(x,y) \in S} \pdf_{\mu_R}(x, y)
	 \Big)	 
	 \end{array}
	 \]
	 \begin{itemize}
	 	\item[{\bf case}] $S \subseteq \{(x, y) | x \in [0, 1) \land x \cdot e^{-\epsilon} = y\}$:\\
		 \[
		 \begin{array}{ll}
		 \Delta_{\epsilon}(\mu_L, \mu_R) 
		 & = 
		 \int_{(x,y) \in S} \pdf_{\uniform}(x) - e^{\epsilon} \int_{(x,y) \in S} \pdf_{\uniform}(y)\\
		 & = 
		 \int_{(x,y) \in S} \pdf_{\uniform}(x) - e^{\epsilon} \int_{(x,y) \in S} \pdf_{\uniform}(x * e^{-\epsilon})\\ 
		 & = 
		 \int_{(x,y) \in S} \pdf_{\uniform}(x) - e^{\epsilon}* e^{-\epsilon} \int_{(x,y) \in S} \pdf_{\uniform}(x )\\
		 & = 0 
		 \end{array}
		 \]
	 	\item[{\bf case}] $S \subseteq \{(x, y) | x \in [1, e^{\epsilon}) \land x \cdot e^{-\epsilon} = y\}$:\\
		 \[
		 \begin{array}{ll}
		 \Delta_{\epsilon}(\mu_L, \mu_R) 
		 & = 
		 0 - e^{\epsilon} \int_{(x,y) \in S} \pdf_{\uniform}(y)\\
		 & <  0 
		 \end{array}
		 \]
	 	\item[{\bf case}] o.w.\\
		 \[
		 \begin{array}{ll}
		 \Delta_{\epsilon}(\mu_L, \mu_R) 
		 & = 0 - 0 =  0 
		 \end{array}
		 \]	 	

	 \end{itemize}

\end{itemize}
\end{proof}

\section{Formalization}
\begin{defn}
[$\snap(a) : A \to \distr(B)$]
The ideal Snapping mechanism $\snap( a)$ is defined as:
\[
	u \xleftarrow{\$} \mu; y = \frac{\ln (u)}{\epsilon}; s \samplel \{-1, 1\}; z = s * y; x = f(a); w = x + z; w' = \lfloor w \rfloor_{\Lambda}; r = \clamp_B (w')
\]
where $f$ is the query function over input $a \in A$, $\epsilon$ is the privacy budget, $B$ is the clampping bound and $\Lambda$ is the rounding argument satisfying $\lambda = 2^k$ where $2^k$ is the smallest power of 2 greater or equal to the $\frac{1}{\epsilon}$.
\end{defn}

\begin{thm}
The $\snap$ mechanism is differentially praivate by following derivation in Figure \ref{derivation_snap}.
\end{thm}

\begin{figure}
\begin{mathpar}
\inferrule*[right = AxUnif]
{
}
{
	u_1 \samplel \mu \sim_{\epsilon, 0} u_2 \samplel \mu : \top \Rightarrow  e^{-\epsilon} u_2 \leq u_1 \leq e^{\epsilon} u_2
}
\and
\inferrule*[right = AxNull]
{
}
{
	y_1 = \frac{\ln(u_1)}{\epsilon} \sim_{0, 0} 
	y_2 = \frac{\ln(u_2)}{\epsilon} : e^{-\epsilon} u_2 \leq u_1 \leq e^{\epsilon} u_2  \Rightarrow y_2 - 1 \leq y_1 \leq 1 + y_2
}
\and
\inferrule*[right = AxNull]
{
}
{
	s_1 \samplel \{ -1, 1\} \sim_{0, 0} s_2 \samplel \{ -1, 1\} : \top \Rightarrow s_1 = s_2
}
\and
\inferrule*[right = AxNull]
{
}
{
	z_1 = s_1 * y_1 \sim_{0, 0} z_2 = s_2 * y_2 : s_1 = s_2 \land y_2 - 1 \leq y_1 \leq 1 + y_2  \Rightarrow | z_1 - z_2 | \leq 1
}
\and
\inferrule*[right = AxNull]
{
}
{
	x_1 = f(a_1) \sim_{0, 0} x_2 = f(a_2) : a_1 = a_2 + 1 \land f(a_1) = f(a_2) + 1 \Rightarrow x_1 = x_2 + 1 
}
\and
\inferrule*[right = AxNull]
{
}
{
	w_1 = x_1 + z_1 \sim_{0, 0} w_2 = x_2 + z_2 : x_1 = x_2 + 1 \land | z_1 - z_2 | \leq 1  \land -2 \leq k \leq 0   \Rightarrow w_1 + k = w_2
}
\and
\inferrule*[right = AxNull]
{
}
{
	w'_1 = \lfloor w_1 \rfloor_{\Lambda} 
	\sim_{0, 0} w'_2 = \lfloor w_2 \rfloor_{\Lambda} : w_1 + k = w_2 \land -2 \leq k \leq 0  \Rightarrow w'_1 + k = w'_2
}
\and
\inferrule*[right = AxNull]
{
}
{
	r_1 = \clamp_B (w'_1) 
	\sim_{0, 0} r_2 = \clamp_B (w'_2)
	: w'_1 + k = w'_2  \land -2 \leq k \leq 0  \Rightarrow r_1 + k = r_2 
}
\and
\inferrule*[right = Comp]
{
\cdots
}
{
	r_1 = \snap(a_1) 
	\sim_{\epsilon, 0} r_2 = \snap(a_1) 
	: a_1 = a_2 + 1 \land f(a_1) = f(a_2) + 1  \land -2 \leq k \leq 0  \Rightarrow r_1 + k = r_2 
}
\end{mathpar}
\caption{Coupling Derivation of two $\snap$ mechanisms: $\snap(a_1)$, $\snap(a_2)$}
\label{derivation_snap}
\end{figure}


\begin{defn}
[$\epsilon$ - dilation].

Let $\epsilon \geq 0$. The $\epsilon$-dilation $D_{\epsilon}(\mu_1, \mu_2)$ between two sub-distributions $\mu_1 \in \distr(U)$, $\mu_2 \in \distr(U)$is defined as:
\[	
	\underset{E \in U}{sup}\Big(\pr{x \leftarrow \mu_1}{x \in E} - \exp(\epsilon) \pr{x \leftarrow \mu_2}{x \in \exp(-\epsilon) \cdot E}]\Big)
\]
\end{defn}

\begin{prop}
[($\epsilon, \delta$)-differential privacy]
For every pair of sub-distributions $\mu_1 \in \distr(U)$, $\mu_2 \in \distr(U)$, s.t. 
\[
D_{\epsilon}(\mu_1, \mu_2) \leq \delta,
\]
The snapping mechanism $\snap(\mu, a) : \distr(U) \to A \to \distr(B)$ is $(\epsilon, \delta)$ - differentially private w.r.t. an adjacency relation $\Phi$ for every two adjacent inputs a, a’ and $\mu_1, \mu_2$
\end{prop}

\begin{proof}
Followed directly by unfolding the $\snap$ mechanism.
\[
	\begin{array}{lll}
	\pr{x \leftarrow \snap(\mu_1, a)}{x = e} 
	& = & \pr
			{u \leftarrow \mu_1}
			{	\lfloor 
				f(a) + \frac{ S \cdot \log(u)}{\epsilon} 
				\rfloor_{\Lambda} = e
			}\\
	& = & \pr
			{ u \leftarrow \mu_1}
		   	{ u \in [
		   		\frac{\exp((e - \frac{\Lambda}{2} - f(a)) \epsilon )}{S},
		   		\frac{\exp((e + \frac{\Lambda}{2} - f(a)) \epsilon )}{S})
		   	}\\
	& \leq & \exp(\epsilon)
			\pr
			{ u \leftarrow \mu_2}
		   	{ u \in \exp(-\epsilon)[
		   		\frac{\exp((e - \frac{\Lambda}{2} - f(a)) \epsilon )}{S},
		   		\frac{\exp((e + \frac{\Lambda}{2} - f(a)) \epsilon )}{S})
		   	}\\
	& = & \exp(\epsilon)
			\pr
			{u \leftarrow \mu_2}
			{	\lfloor 
				f(a') + \frac{ S \cdot \log(u)}{\epsilon} 
				\rfloor_{\Lambda} = e
			}\\
	& = & \exp(\epsilon)
			\pr{x \leftarrow \snap(\mu_2, a')}{x = e} 
	\end{array}
\]
\end{proof}



\newpage
\bibliographystyle{plain}
\bibliography{verifysnap.bib}


\end{document}















