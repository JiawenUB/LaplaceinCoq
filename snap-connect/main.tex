\documentclass[a4paper,11pt]{article}
\usepackage[utf8]{inputenc}
%

\usepackage[utf8]{inputenc}%Packages
\usepackage[T1]{fontenc}
\usepackage{fourier} 
\usepackage[english]{babel} 
\usepackage{amsmath,amsfonts,amsthm} 
\usepackage{lscape}
\usepackage{geometry}
\usepackage{amsmath}
\usepackage{algorithm}
\usepackage{algorithmic}
\usepackage{amssymb}
\usepackage{amsfonts}
\usepackage{times}
\usepackage{bm}
\usepackage{mathtools}
\usepackage{ stmaryrd }
\usepackage{ amssymb }
\usepackage{ textcomp }
\usepackage[normalem]{ulem}
% For derivation rules
\usepackage{mathpartir}
\usepackage{color}
\usepackage{a4wide}

\usepackage{stmaryrd}
\SetSymbolFont{stmry}{bold}{U}{stmry}{m}{n}

\newcommand{\distr}{\mathsf{Distr}}
\newcommand{\uniform}{\mathsf{unif}}
\newcommand{\pdf}{\mathsf{pdf}}
\newcommand{\snap}{\mathsf{Snap}}
\newcommand{\fsnap}{\mathsf{Snap}_{\mathbb{F}}}
\newcommand{\rsnap}{\mathsf{Snap}_{\mathbb{R}}}


\newcommand{\pr}[2]{\underset{#1}{\mathsf{Pr}}[#2]}
\newcommand{\projl}{\pi_1}
\newcommand{\projr}{\pi_2}
\newcommand{\supp}{\mathsf{supp}}
\newcommand{\clamp}{\mathsf{clamp}}
\newcommand{\real}{\mathbb{R}}
\newcommand{\samplel}{\xleftarrow{\$}}
\newcommand{\psup}{\mathsf{Sup}}
\newcommand{\sign}{\mathsf{sign}}

\newcommand{\lapmech}{\mathcal{L}}
\newcommand{\laplace}{\mathsf{laplce}}
\newcommand{\round}[1]{\lfloor #1 \rceil}


%for syntax:

%for programs:
\newcommand{\prog}{p}
\newcommand{\fprog}{p_{\mathbb{F}}}
\newcommand{\rprog}{p_{\mathbb{R}}}
\newcommand{\ret}{\mathsf{return}}



%expression
\newcommand{\expr}{e}
\newcommand{\fexpr}{\expr_{\mathbb{F}}}
\newcommand{\rexpr}{\expr_{\mathbb{R}}}

\newcommand{\elet}{\kw{let}}

\newcommand{\ein}{\kw{in}}

%for smaples:
\newcommand{\bernoulli}{\kw{bernoulli}}

%values
\newcommand{\fval}{c}
\newcommand{\rval}{r}
\newcommand{\valv}{v}
\newcommand{\data}{D}

%variables
\newcommand{\varx}{x}

\newcommand{\fvarx}{x}
\newcommand{\rvarx}{X}


\newcommand{\term}{t}
\newcommand{\etrue}{\kw{true}}
\newcommand{\efalse}{\kw{false}}
% \newcommand{\eflconst}{c}
% \newcommand{\erlconst}{r}
\newcommand{\precision}{\eta}
\newcommand{\floaten}{\kw{fl}}

\newcommand{\err}{err}
\newcommand{\condition}{\Phi}
\newcommand{\edistr}{\mu}

\newcommand{\fbigstep}{\Downarrow^{\mathbb{F}}}
\newcommand{\rbigstep}{\Downarrow^{\mathbb{R}}}

\newcommand{\bigstep}{\Downarrow}
\newcommand{\trsto}{\Rightarrow}


%for environments
\newcommand{\trsenv}{\Theta}

\newcommand{\evlenv}{\Gamma}

\newcommand{\fevlenv}{\Gamma^{\mathbb{F}}}

\newcommand{\revlenv}{\Gamma^{\mathbb{R}}}



\usepackage{stackengine} 

% For Operations
%binary operations
\newcommand{\bop}{*}
\newcommand{\obop}{\stackMath\mathbin{\stackinset{c}{0ex}{c}{0ex}{\text{\footnotesize{$\bop$}}}{\bigcirc}}}

\newcommand{\oexp}{\stackMath\mathbin{\stackinset{c}{0ex}{c}{0ex}{\text{\footnotesize{$\mathsf{e}$}}}{\bigcirc}}}

\newcommand{\oln}{\stackMath\mathbin{\stackinset{c}{0ex}{c}{0ex}{\text{\footnotesize{$\mathsf{ln}$}}}{\bigcirc}}}

\newcommand{\odiv}{\stackMath\mathbin{\stackinset{c}{0ex}{c}{0ex}{\text{\footnotesize{$\div$}}}{\bigcirc}}}
\newcommand{\ubar}[1]{\text{\b{$#1$}}}

%unary operations
\newcommand{\uop}{\circ}
\newcommand{\ouop}{\stackMath\mathbin{\stackinset{c}{0ex}{c}{0ex}{\text{\footnotesize{$\uop$}}}{\bigcirc}}}





\newcommand{\diam}{{\color{red}\diamond}}
\newcommand{\dagg}{{\color{blue}\dagger}}
\let\oldstar\star
\renewcommand{\star}{\oldstar}

\newcommand{\im}[1]{\ensuremath{#1}}

\newcommand{\kw}[1]{\im{\mathtt{#1}}}


\newcommand{\set}[1]{\im{\{{#1}\}}}

\newcommand{\mmax}{\ensuremath{\mathsf{max}}}

%%%%%%%%%%%%%%%%%%%%%%%%%%%%%%%%%%%%%%%%%%%%%%%%%%%%%%%%
% Comments
\newcommand{\omitthis}[1]{}

% Misc.
\newcommand{\etal}{\textit{et al.}}
\newcommand{\bump}{\hspace{3.5pt}}

% Text fonts
\newcommand{\tbf}[1]{\textbf{#1}}
%\newcommand{\trm}[1]{\textrm{#1}}

% Math fonts
\newcommand{\mbb}[1]{\mathbb{#1}}
\newcommand{\mbf}[1]{\mathbf{#1}}
\newcommand{\mrm}[1]{\mathrm{#1}}
\newcommand{\mtt}[1]{\mathtt{#1}}
\newcommand{\mcal}[1]{\mathcal{#1}}
\newcommand{\mfrak}[1]{\mathfrak{#1}}
\newcommand{\msf}[1]{\mathsf{#1}}
\newcommand{\mscr}[1]{\mathscr{#1}}

% Text mode
\newenvironment{nop}{}{}

% Math mode
\newenvironment{sdisplaymath}{
\begin{nop}\small\begin{displaymath}}{
\end{displaymath}\end{nop}\ignorespacesafterend}
\newenvironment{fdisplaymath}{
\begin{nop}\footnotesize\begin{displaymath}}{
\end{displaymath}\end{nop}\ignorespacesafterend}
\newenvironment{smathpar}{
\begin{nop}\small\begin{mathpar}}{
\end{mathpar}\end{nop}\ignorespacesafterend}
\newenvironment{fmathpar}{
\begin{nop}\footnotesize\begin{mathpar}}{
\end{mathpar}\end{nop}\ignorespacesafterend}
\newenvironment{alignS}{
\begin{nop}\begin{align}}{
\end{align}\end{nop}\ignorespacesafterend}
\newenvironment{salignS}{
\begin{nop}\small\begin{align}}{
\end{align}\end{nop}\ignorespacesafterend}
\newenvironment{falignS}{
\begin{nop}\footnotesize\begin{align*}}{
\end{align}\end{nop}\ignorespacesafterend}

% Stack formatting
\newenvironment{stackAux}[2]{%
\setlength{\arraycolsep}{0pt}
\begin{array}[#1]{#2}}{
\end{array}}
\newenvironment{stackCC}{
\begin{stackAux}{c}{c}}{\end{stackAux}}
\newenvironment{stackCL}{
\begin{stackAux}{c}{l}}{\end{stackAux}}
\newenvironment{stackTL}{
\begin{stackAux}{t}{l}}{\end{stackAux}}
\newenvironment{stackTR}{
\begin{stackAux}{t}{r}}{\end{stackAux}}
\newenvironment{stackBC}{
\begin{stackAux}{b}{c}}{\end{stackAux}}
\newenvironment{stackBL}{
\begin{stackAux}{b}{l}}{\end{stackAux}}

%APPENDIX
\newcommand{\caseL}[1]{\item[\textbf{case}] \textbf{#1}\newline}
\newcommand{\subcaseL}[1]{\item[\textbf{subcase}] \textbf{#1}\newline}

\newcommand{\todo}[1]{{\footnotesize \color{red}\textbf{[[ #1 ]]}}}


%% \makeatletter
%% \newcommand\definitionname{Lemma}
%% \newcommand\listdefinitionname{Proofs of Lemmas and Theorems}
%% \newcommand\listofdefinitions{%
%%   \section*{\listdefinitionname}\@starttoc{def}}
%% \makeatother



\newtheoremstyle{athm}{\topsep}{\topsep}%
      {\upshape}%         Body font
      {}%         Indent amount (empty = no indent, \parindent = para indent)
      {\bfseries}% Thm head font
      {}%        Punctuation after thm head
      {.8em}%     Space after thm head (\newline = linebreak)
      {\thmname{#1}\thmnumber{ #2}\thmnote{~\,(#3)}
% \addcontentsline{Lemma}{Lemma}
%   {\protect\numberline{\thechapter.\thelemma}#1}
      % \ifstrempty{#3}%
      {\addcontentsline{def}{section}{#1~#2~#3}}%
      % {\addcontentsline{def}{subsection}{\theathm~#3}}
\newline}%         Thm head spec

 \theoremstyle{athm}


% \newtheoremstyle{break}
%   {\topsep}{\topsep}%
%   {\itshape}{}%
%   {\bfseries}{}%
%   {\newline}{}%
% \theoremstyle{break}

%There are some problems with llncs documentcalss, so commenting these out until i find a solution
\newtheorem{thm}{Theorem}

%\spnewtheorem{thm1}[theorem]{Theorem}{\bfseries}{\upshape}
%\newenvironment{Theorem}[1][]{\begin{thm1}\iffirstargument[#1]\fi\quad\\}{\end{thm1}}

 \newtheorem{lem}[thm]{Lemma}
 \newtheorem{conjec}{Conjecture}
 \newtheorem{corr}[thm]{Corollary}
 \newtheorem{defn}{Definition}
 \newtheorem{prop}[thm]{Proposition}
 \newtheorem{assm}[thm]{Assumption}

\newtheorem{Eg}[thm]{Example}
\newtheorem{hypothesis}[thm]{Hypothesis}
\newtheorem{motivation}{Motivation}

% BNF symbols
\newcommand{\bnfalt}{{\bf \,\,\mid\,\,}}
\newcommand{\bnfdef}{{\bf ::=~}}

%% Highlighting
\newcommand{\hlm}[1]{\mbox{\hl{$#1$}}}

%% Provenance modes
\newcommand{\modifrcationProvenance}{{\bf MP}}
\newcommand{\updateProvenance}{{\bf UP}}

%Lemmas
\newcommand{\lemref}[1]{Lemma \ref{#1}} %name and number
\newcommand{\thmref}[1]{Theorem \ref{#1}} %name and number

\renewcommand{\labelenumii}{\theenumii}
\renewcommand{\theenumii}{\theenumi.\arabic{enumii}.}

\usepackage{enumitem}
\setenumerate{listparindent=\parindent}

\newlist{enumih}{enumerate}{3}
\setlist[enumih]{label=\alph*),before=\raggedright, topsep=1ex, parsep=0pt,  itemsep=1pt }

\newlist{enumconc}{enumerate}{3}
\setlist[enumconc]{leftmargin=0.5cm, label*= \arabic*.  , topsep=1ex, parsep=0pt,  itemsep=3pt }

\newlist{enumsub}{enumerate}{3}
\setlist[enumsub]{ leftmargin=0.7cm, label*= \textbf{subcase} \bf \arabic*: }

\newlist{enumsubsub}{enumerate}{3}
\setlist[enumsubsub]{ leftmargin=0.5cm, label*= \textbf{subsubcase} \bf \arabic*: }

\newlist{mainitem}{itemize}{3}
\setlist[mainitem]{ leftmargin=0cm , label= {\bf Case} }


\newenvironment{subproof}[1][\proofname]{%
  \renewcommand{\qedsymbol}{$\blacksquare$}%
  \begin{proof}[#1]%
}{%
  \end{proof}%
}


\newenvironment{nstabbing}
  {\setlength{\topsep}{0pt}%
   \setlength{\partopsep}{0pt}%
   \tabbing}
  {\endtabbing} 





%%% Local Variables:
%%% mode: latex
%%% TeX-master: "main"
%%% End:

\usepackage{eucal}
\usepackage{url}
\usepackage{tikz}
\usepackage{amsfonts,amsmath}
\begin{document}

\title{Verifying Snapping Mechanism - Connecting Ideal and Flopt}
\author{}

\date{}

\maketitle
In order to verify the differential privacy proeprty of
the snapping mechanism \cite{mironov2012significance},
we follow the logic rules designed from
\cite{barthe2016proving} and connecting 
it with the floating point computation semantics.

\section{
Connecting Coupling Logic
%
with Floating Point Computation
}
%
\begin{figure}[h]
\begin{mathpar}
\inferrule*[right = FloptUnif]
{
\rvalu^1 = \floaten(\fvalu^1)
\and
\rvalu^2 = \floaten(\fvalu^2)
}
{
	\vdash 
	\fvalu^1 \samplel \mu
	\trsto (e^{-k_1}\rvalu^1, e^{k_1}\rvalu^1)
	\sim_{(k_1 + k_2)+\epsilon, 0} 
	\fvalu^2 \samplel \mu \trsto (e^{-k_2}\rvalu^2, e^{k_2}\rvalu^2)
	: \top \Rightarrow  e^{-\epsilon} \rvalu^2 \leq \rvalu^1 \leq e^{\epsilon} \rvalu^2
}
\end{mathpar}
\caption{Rules of Connecting Coupling Logic with Floating Point Semantics}
\label{logic_rule}
\end{figure}

This rule is interprated as:
when $\fvalu^1$ is sampled from distribution $\mu$ and
can be transited to $\rvalu^1$ in snap-flopt transition semantics
with error range $(e^{-k_1}\rvalu^1, e^{k_1}\rvalu^1)$;
and $\fvalu^2$ is sampled from distribution $\mu$ and
%
can be transited to $\rvalu^2$ in snap-flopt transition semantics 
with error range $(e^{-k_2}\rvalu^2, e^{k_2}\rvalu^2)$,
if we have the postcondition
$e^{-\epsilon} \rvalu^2 \leq \rvalu^1 \leq e^{\epsilon} \rvalu^2$,
then the privacy loss in the floating point semantics
are $(k_1 + k_2 + \epsilon)$.

\paragraph{Semantics. - Non-deterministic}
\[
	\begin{array}{rcl}
	\sem{\varx \samplel \edistr}_{\trsenv}
	& \in &  \big\{\trsenv[\varx \mapsto \rvalu] | 
	\fvalu \leftarrow \edistr^{\diamond} \land \rvalu = \floaten(\fvalu)\big\}\\
	\sem{\varx = \expr}_{\trsenv}
	& \in &  \big\{\trsenv[\varx \mapsto \rvalu] | 
	\expr \trsto (\rvalv, err)\big\}\\
	\sem{\prog_1; \prog_2}_{\trsenv}
	& \in &  \big\{\trsenv_2 | 
	\sem{\prog_1}_{\trsenv} = \trsenv_1 \land
	\sem{\prog_2}_{\trsenv_1} = \trsenv_2\big\},
	\end{array}
\]
where $\fvalu \leftarrow \edistr^{\diamond}$ represents a value $\fvalu$ is sampled from the $\edistr$ and
$\expr \trsto (\rvalv, err)$ represents a the expression $\expr$
is transited to $\rvalv$ with error bound $err$ in floating point transition semantics.

\paragraph{Semantics. - Probabilistic}
\[
	\begin{array}{rcl}
	\sem{\varx \samplel \edistr}_{\trsenv}
	& = & \unit{\trsenv[\varx \mapsto \edistr]}\\
	\sem{\varx = \expr}_{\trsenv}
	& = & \unit{\trsenv[\varx \mapsto \rvalv] | 
	\expr \trsto (\rvalv, err)}\\
	\sem{\prog_1; \prog_2}_{\trsenv}
	& = & let ~~
	\trsenv_1 = \sem{\prog_1}_{\trsenv} ~~ in ~~
	\sem{\prog_2}_{\trsenv_1},
	\end{array}
\]
where $\expr \trsto (\rvalv, err)$ represents a the expression $\expr$
is transited to $\rvalv$ with error bound $err$ in floating point transition semantics and $\unit{a}$ is defined as a unit distribution with probabilistic $1$ returning $a$.

\begin{thm}[The $\snap$ mechanism is 
$\epsilon-$differentially private]
\end{thm}

\begin{figure}[h]
\begin{mathpar}
\inferrule*[right = AxUnif]
{
\rvalu^1 = \floaten(\fvalu^1)
\and
\rvalu^2 = \floaten(\fvalu^2)
}
{
	\vdash
	\fvalu^1 \samplel \mu
	\trsto ( e^{-12B \epsilon \eta} \rvalu^1,
	e^{12B \epsilon \eta}\rvalu^1)
	\sim_{24B \epsilon \eta +\epsilon, 0}
	\fvalu^2 \samplel \mu
	\trsto ( e^{-12B \epsilon \eta} \rvalu^2,
	e^{12B \epsilon \eta} \rvalu^2)
	: \top \Rightarrow  e^{-\epsilon} \rvalu^2
	\leq \rvalu^1 \leq e^{\epsilon} \rvalu^2
}
\and
\inferrule*[right = AxNull]
{
}
{
	y_1 = f(a_1) + \frac{1}{\epsilon} \times s \times \ln(\fvalu^1)
	\sim_{ 24B \epsilon \eta + \epsilon, 0 }
	y_2 = f(a_2) + \frac{1}{\epsilon} \times s \times \ln(\fvalu^2)
	:
	\\
	e^{-\epsilon} \rvalu^2 \leq \rvalu^1 \leq e^{\epsilon} \rvalu^2 \land
	a_1 = a_2 + 1 \land f(a_1) = f(a_2) + 1
	\land |k + f(a_1) - f(a_2) | \leq 1 
	\Rightarrow y_1 + k = y_2 
}
\and
\inferrule*[right = Comp]
{
\cdots
}
{
	\fvalu^1 \samplel \mu;
	r_1 = \clamp(\round{f(a_1) + \frac{1}{\epsilon} \times s \times \ln(\fvalu^1)}_{\Lambda})
	\sim_{\epsilon, 0}
	\fvalu^2 \samplel \mu;
	r_1 = \clamp(\round{f(a_2) + \frac{1}{\epsilon} \times s \times \ln(\fvalu^2)}_{\Lambda})
	: 
	\\
	a_1 = a_2 + 1 \land f(a_1) = f(a_2) + 1 \land 
	|k + f(a_1) - f(a_2) | \leq 1
	\Rightarrow r_1 + k = r_2 
}
\end{mathpar}
\caption{Coupling Derivation of two $\snap$ mechanisms:
$\snap(a_1)$, $\snap(a_2)$}
\label{derivation_snap}
\end{figure}



\newpage
\bibliographystyle{plain}
\bibliography{verifysnap.bib}

\end{document}



